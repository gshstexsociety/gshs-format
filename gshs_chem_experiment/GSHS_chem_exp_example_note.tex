\documentclass[a4paper,10pt]{article}
\usepackage[left=25mm,right=25mm,top=30mm,bottom=30mm]{geometry}
\usepackage{tocloft}
\cftpagenumbersoff{section}
\cftpagenumbersoff{subsection}
\cftpagenumbersoff{subsubsection}
\RequirePackage[nonfrench,hangul]{kotex}
\usepackage[ensec]{dhucs-sectsty}
\renewcommand\thesection{\arabic{section}}
\renewcommand\thesubsection{\arabic{section}.\arabic{subsection}}
\usepackage{chngcntr}
\renewcommand\thesubsubsection{%
	\arabic{section}.\arabic{subsection}.\arabic{subsubsection}%
}
\usepackage{titlesec}
\titleformat{\paragraph}
{\normalfont\normalsize\bfseries}{\theparagraph.}{1em}{}
\titleformat{\subparagraph}
{\normalfont\normalsize\bfseries}{\thesubparagraph.}{1em}{}
\setcounter{tocdepth}{2}
\titlelabel{\thetitle.\quad}
\usepackage{ulem}
\usepackage[version=4]{mhchem}
\usepackage{tikz}
\mhchemoptions{arrows=pgf}
\usepackage{siunitx}
\sisetup{inter-unit-product = \ensuremath{ { } \cdot { } } }
\DeclareSIUnit\atmosphere{atm}
\DeclareSIUnit\Molar{\textsc{m}}
\DeclareSIUnit\molar{\mole\per\litre}
\usepackage{amsmath}
\usepackage{graphicx,xcolor}
\usepackage{amssymb}
\usepackage{indentfirst}
\setlength\parindent{1.0em}
\newcommand{\unit}[2]{$\num{#1}\,{\mbox{#2}}$}
\newcommand{\unitrange}[3]{$\num{#1}\mbox{ - }\num{#2}\,\mbox{#3}$}
\begin{document}
	\title{일반화학실험 보고서 작성 도움말}	
	\author{14088 이주찬}%내용 추가·수정하신 분은 추가로 이름 적어주세요.
	\date{2016년 2월 7일 마지막으로 수정}
	\maketitle
	\tableofcontents
	\section{개요}
	일반화학실험 보고서 양식을 바탕으로 보고서를 작성할 때 참고할 점 및 
	유의할 점 등을 설명한 문서입니다. 일단 한 번 읽은 뒤 \LaTeX 으로 보고서 
	작성을 하시는 것을 권장합니다.
	일반화학실험(이하 일화실) 보고서를 작성할 때 기본적으로 지켜야 할 점 및 
	현 양식에서 특별히 알아두어야 할 점 등을 같이 설명합니다.
	
	계속해서 내용 추가 예정입니다.
	
	\section{수식 관련}
	
	\subsection{화학식}
	화학식을 쓸 때 흔히 범하는 오류로 원소 기호를 기울여 쓰는 것이
	있습니다(이는 한/글 수식 편집기 기본 설정이 기울임이기 때문인 것으로
	파악됩니다. 참고로 한/글 수식 편집기에서는 rm을 쓰면 로만(곧게 선) 
	글꼴을, it를 쓰면 기울인 글꼴을 쓸 수 있습니다).
	IUPAC 권장을 따르면 원소 기호나 물질의 상태 등은 곧게 씁니다.
	
	\LaTeX 에서 일반적인 수식이 아닌 화학식을 작성할 때에는
	\textbackslash math가 아닌 \textbackslash ce를 사용하시면 됩니다.
	예시로, \ce{H_{2}O(l)}를 쓰고 싶다면
	\textbackslash ce\{H\_\{2\}O(l)\}라고 쓰시면 됩니다.
	수식 내에서 화학식이 나올 때도 수식 안에서
	\textbackslash ce\{\} 명령을 사용하시면 됩니다.
	기타 자세한 사용 방법은 mhchem 패키지의 package documentation을
	보시면 됩니다. 화살표 등 자세한 방법이 있습니다.
	사용 방법을 잘 모를 때는 문의 부탁드립니다.
	
	\subsubsection{표준 상태 기호}
	수식 모드에서 \textbackslash circ를 올려서 쓰면 됩니다.
	
	예 : $\Delta G^{\circ}$를 표현하려면 \$\textbackslash Delta
	G\textasciicircum\{\textbackslash circ\}\$
	
	화실 보고서를 보면 o나 0을 올려 표기하는 경우가 많이 있는데, 대부분의 
	화학 서적은 circ를 올려 표기하고 있습니다. 이것이 맞는 표기로 보입니다.
	
	\subsection{표시형 수식}
	원래 표시형 수식은 displaymath를 사용하는 것이 일반적이나,
	일화실 보고서를 작성하다 보면 수식이 여러 줄 나올 때가 많은데
	displaymath는 여러 줄의 수식을 표현하기에 어려움이 있으므로
	gather*를 사용하시는 게 좋습니다.
	개행은 \textbackslash\textbackslash 로 합니다.
	
	(*은 수식 번호를 표시하지 않기 위한 용도입니다.
	번호를 사용하고 싶다면 * 없이 사용하시면 됩니다.)
	
	\section{조사 관련}
	한국어는 조사 앞에 오는 숫자의 끝 발음이 무엇이냐에 따라
	뒤에 붙어야 하는 조사가 달라져야 하기 때문에,
	참고 번호에 조사가 붙으면 무얼 써야 할 지 모르게 됩니다.
	
	예 : \textbackslash ref\{ \# \}에서 그림 \#이 / 그림 \#가
	
	이런 경우, \textbackslash ref\{ \# \}\textbackslash 이 또는
	\textbackslash ref\{ \# \}\textbackslash 가 중 아무거나 쓰시면
	번호에 따라 알아서 조사를 지정해 줍니다. 
%	이 내용은 기본 도움말로 옮기는 게 나을 것 같습니다.
%	일화실 보고서 말고도 전반적인 LaTeX 작성에 도움이 됩니다.
	
	\section{단위 관련}
	단위는 그냥 사용해도 좋으나 권장하는 방법은 siunitx 패키지를 사용하여
	표시하는 방법입니다.
	
	단위를 나타낼 때 주의해야 할 점 두 가지는 첫째로 숫자와 단위 사이는 
	띄어 쓴다는 점, 둘째는 숫자와 단위 사이에는 개행이 일어나면 안 된다는 
	점입니다. 또한 단위 띄어쓰기는 보통 더 짧은 띄어쓰기를 쓰기 때문에,
	\$숫자\textbackslash,\textbackslash mbox\{단위\}\$를 사용합니다.
	
	본 일화실 cls 파일에는 unit과 unitrange를
	정의하여 간단하게 사용할 수 있도록 하였습니다. 기본적으로는
	\textbackslash unit\{숫자\}\{단위\}를 사용하면 
	`$\mbox{숫자}\,\mbox{단위}$'의 형태로 나타나고, 범위인 경우
	\textbackslash unitrange\{숫자1\}\{숫자2\}\{단위\}를 사용하면
	`$\mbox{숫자1 - 숫자2}\,\mbox{단위}$'의 형태로 나타나게 됩니다.
	이는 각각 SI, SIrange에 대응되며, 사용 방법은 거의 같습니다.
	
	단위가 SI 단위라면 siunitx를 사용하는 것이 좋습니다
	siunitx 패키지를 사용하면 단위를 자신이 원하는 형식대로 지정, 변경하여 
	나타내기 쉽습니다.
	
	사용법은 단위 단독으로 쓸 때는
	\textbackslash si\{단위\}를 사용하면 되고,
	숫자와 단위를 같이 사용하실 때는
	\textbackslash SI\{숫자\}\{단위\}를 쓰면 됩니다.
	예시로 중력가속도 \SI{9.8}{\metre\per\square\second}를 쓸 때는
	\textbackslash SI\{9.8\}\{\textbackslash metre\textbackslash
	per\textbackslash square\textbackslash second\}로 쓰시면 됩니다.
	제곱은 square로, 세제곱은 cubic으로 사용하며,
	per 앞에 곱 단위를, 뒤에 나누는 단위를 쓰시면 됩니다.
	
	음수 제곱으로 나타내지 않고 나누기 표시 /로 나타내는 방법은
	\textbackslash sisetup\{per-mode = symbol\}을 document 앞쪽에
	써 주시면 됩니다.
	
	단위 표시의 공통된 사항으로는, 기본적으로 수식 처리이기 때문에
	숫자와 단위가 길어질 경우 overfull이 발생할 수 있다는 것입니다.
	log를 주의 깊게 살펴 overfull이 나지 않도록 유의하시기 바랍니다.
%	overfull이나 underfull의 경우는 기본 도움말에 따로 수록하는 것이
%	더 낫다고 생각합니다.
	
	특별히 본 일화실 cls 파일에는 SI 단위는 아니지만
	화학 실험에서 자주 나오는 atm 단위를 \textbackslash atmosphere로
	추가하였고, 몰 농도 단위를
	\textbackslash molar와 \textbackslash Molar로 추가하였습니다.
	힘의 단위로 사용하는 gf를 \textbackslash gramforce로 추가하였습니다.
	예시로 \textbackslash si\{\textbackslash molar\} 와
	\textbackslash si\{\textbackslash Molar\}는
	각각 \si{\molar} 와 \si{\Molar}로 작동합니다.
	몰 농도에 해당하는 M은 약간 작게 씁니다.
%	문서가 길어지는데, 이것도 윗 부분은 기본 도움말에 있어야 하고
%	본 양식에서 추가된 점만을 설명하는 게 좋을 것 같습니다.
	
	\section{목록 관련}
	enumerate를 사용한 목록의 번호는 기본적으로 선생님께서 주시는
	실험 방법 pdf와 동일한 원 문자를 사용하였습니다.
	그러나 \LaTeX 에서 원 문자를 사용하는 데에는 어려움이 있고, 때문에
	15~개 이상의 item이 있으면 error가 나게 됩니다. 이 경우,
	GSHS-chemexp.cls 파일에서
	\textbackslash usepackage\{dhucs-enumitem\} 이하 3~줄을
	주석처리하고 그 위쪽 2~줄을 활성화하면 괄호 문자로 바뀌어 출력됩니다.
%	추후 on-off 형식으로 전환하는 기능을 만드는 것이 필요해 보입니다.
%	아니면 cls에서 tex 파일로 옮겨 tex 파일에서 전환 가능하도록
%	해야 할 것 같습니다.
	
	또한, 지금 cls 파일에는 enumerate style을 두 단계까지만 설정하였으므로,
	세 단계 이상 들어가게 될 경우 어울리지 않는 기호가 나올 수 있습니다.
	가급적 enumerate는 한 단계만 사용하는 것을 추천합니다.	
	
	\section{표 관련}
	
	표 디자인에 관련된 사항은 기본 도움말을 참고하시기 바랍니다.
	아래는 표를 어느 정도 다룰 수 있는 사람을 대상으로
	설명한 내용이니, 모르시는 분은 건너뛰시기 바랍니다.
%	기본 도움말이 있어야 하는데. 언제쯤 만들어질까요?
	
	\subsection{소숫점 정렬}
	실험 자료를 표로 나타낼 때에는 소숫점 정렬을 하는 것이 좋습니다.
	표의 정렬을 소숫점 정렬로 표시하기 위해서, 정렬 옵션을
	D..\{\#.\#\}과 같이 지정합니다.
	앞쪽 \#에는 실험 자료들 중 정수 부분 최대 자릿수를,
	뒷쪽 \#에는 소수 부분 최대 자릿수를 적으면 됩니다.
	
	주의할 점은, 소숫점 정렬 옵션을 사용하면 표 안의 내용을
	수식으로 인식하도록 되어 있습니다. 따라서 \$과 같은 기호를 사용하여
	따로 수학 기호를 입력할 필요가 없고, 한글을 입력하고자 할 때에는 
	text를 써 주어야 합니다.
	
	맨 윗 줄에는 내용을 입력해야 하므로, multicolumn 옵션을 사용하여
	일반적인 가운데 정렬(c)로 바꾸어 주어야 합니다.
	양식의 multicolumn\{1\}\{c\}\{ \}\& 부분이 한 칸입니다.
	필요한 칸 만큼 쓰면 됩니다.(양식은 3열이므로 세 번)
	%이 도움말은 표에 대해 어느 정도 안다는 가정 하에 작성되었으므로,
	%표를 아예 다룰 줄 모른다면 소숫점 정렬을 포기하고
	%인터넷의 table generater를 사용하시는 게 좋습니다.
	
	\subsection{표 디자인}
	표를 어느 정도 다룰 줄 안다면 디자인은 자유롭게 해도 좋습니다.
	cls 파일에 \textbackslash thickhline이라는
	새로운 command를 설정하였습니다.
	이 command는 굵은 가로선을 긋는 command로,
	예시 파일에서는 표의 맨 위와 아래를 굵은 선으로
	처리하기 위해 사용하였습니다.
	
	\section{그래프 관련}
	그래프를 Microsoft Excel 따위의 프로그램으로 그린 뒤
	그냥 그림으로 저장하여 넣는다면, 그림이 vector 이미지가 아니라서
	확대하면 깨지게 됩니다.
	따라서 그래프를 pdf로 저장한 뒤 포함시키는 것이 좋습니다.
	Excel에서 그래프를 그린 뒤 원본 크기로 pdf로 저장하는 방법은, 
	Microsoft Office 2007 기준으로, 다음과 같습니다.
	
	먼저 Excel에서 그래프의 크기를 `차트 도구'-`서식'-`크기'
	\footnote{Office 2016의 경우에는 `차트 영역 서식'}
	를 통해 확인 또는 지정한 뒤, 그래프를 복사합니다.
	다음으로 Powerpoint에서 `디자인'-`페이지 설정'에 들어가 슬라이드 크기를
	그래프와 같은 크기로 지정하고, 붙여넣기합니다.
	이렇게 하면 그래프가 슬라이드에 정확히 맞게 차게 됩니다.
	맞지 않는다면 크기가 잘못된 것입니다.
	그 다음 `파일'-`다른 이름으로 저장'-`PDF 또는 XPS'
	\footnote{Office 2016의 경우에는 `파일'-`내보내기'-`PDF/XPS 문서 만들기'}
	를 선택한 다음 저장하시면 됩니다. 
	
	
	게시 옵션으로 ISO 19005-1 호환을 권장하나 하지 않으셔도 문제 없습니다.
%	Office 2007 기준으로 설명해서, 확인하는 경로가 다를 수 있기는 합니다.
	includegraphics 옵션에서 scale=1로 지정하면 원본 크기가 됩니다.
	
	\section{기타 참고사항}
	참고 문헌에 웹 사이트를 적는 경우, 언더바(\_) 앞에 \textbackslash 를
	붙이는 것을 잊지 말아야 합니다.
	
	웹 사이트 주소를 그냥 적어둘 경우 한 단어로 취급되기 때문에 underfull 
	등의 오류가 발생할 가능성이 매우 높습니다.
	이를 방지하기 위해 \textbackslash allowbreak를 주소 중간에 넣어 주소가
	중간에 개행될 수 있도록 하시기 바랍니다.
	(모든 빗금(/) 뒤에 붙여 두면 좋습니다.)	
	
	원래 chemstyle 패키지를 포함하여 구 표준 상태 기호나 chemstyle의 
	다양한 기능을 사용하고자 했으나 chemstyle 패키지가 style을 강제로
	지정하는 부분이 너무 많아 어쩔 수 없이 사용을 포기하였습니다.
	style에 구애받지 않는 사람들은 사용하셔도 무방합니다.
	(일화실 보고서를 쓰면서 chemstyle 패키지가 필요할 일은 없을 것입니다)
	
\end{document}
