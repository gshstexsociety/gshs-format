\documentclass{gshs_imaginarylab}
\worktitle{작품명} % 작품명
\workauthor{작품저자1, 작품저자2, 작품저자3} % 작품 제작에 참여한 사람 명단
\workdate{2016}{8}{13} % 날짜
% 이미지는 images 폴더 내에 넣어 주십시오.
% 기타 패키지는 여기에 넣으십시오.
\usepackage{lipsum} % dummy text


\begin{document}
\makeworkcover

\section{How does this work?}
	\lipsum[1]
	\insertfig{example-image-a}
	~
	\insertfig{example-image-a}{
		할 말이 없네요.
	}
	
	\lipsum[2]
	
\section{Laser Cutting}
	\lipsum[3]
	\insertfig{example-image-b}{
		\begin{enumerate}
			\item 레이저 커팅은
			\item 이렇게 하면 된다.
			\item 안전이 최우선
		\end{enumerate}
	}
	
\section{3D 프린팅하기}
	\lipsum[4]
	\subsection{3D 모델링}
	\lipsum[5]



% 참고자료 기재
\titleformat{\section}{\Large\bfseries\sffamily}{}{0pt}{}
\addcontentsline{toc}{section}{References}

\begin{thebibliography}{00}
	\bibitem{gshs}{\url{http://gs.hs.kr} (경기과학고등학교 홈페이지)}
	\bibitem{gshstexsociety}{\url{http://gshslatexintro.github.io} (경기과학고 텍 사용자협회)}
\end{thebibliography}

\end{document}