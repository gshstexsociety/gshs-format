\documentclass[11pt]{report}
\usepackage[left=25mm,right=25mm,top=30mm,bottom=30mm]{geometry}
\usepackage{amsmath} % math
\usepackage{amssymb} % math
\usepackage{graphicx} % to use \includegraphics{}
\usepackage{diagbox} % to make tables
\usepackage{multirow}
\usepackage{caption}
\usepackage{subcaption}
\usepackage{kotex}
\usepackage{color}
\usepackage[hidelinks]{hyperref}
\usepackage[per-mode=symbol]{siunitx}
\sisetup{inter-unit-product =$\cdot$}
\usepackage{titlesec}
\usepackage{amsthm}
\usepackage{datetime}
\usepackage{lipsum}
\usepackage{thmtools}
\title{Linear Algebra Thm Archive - for Mid-Term Exam}
\author{by Gyeonggi Science High School for the Gifted `Linear Algebra' Participants \\
    Main Author : 14129 황동욱 \\
    Revised by : 14121 하석민 \\
    \LaTeX ~ Technician : 14041 박승원
} % 저자 추가하세요.
\date{Last Compilation Time : \today ~ \currenttime}

%%%%%%%%%%%%%%%%%%%%%%%%%%%%
%%%%% 수정하지 마시오. %%%%%
% Theorem formatting
\makeatletter
\newtheoremstyle{GSHScustom} % name
{\topsep}% Space above
{\topsep}% Space below
{}% Body font
{}% Indent amount
{\bfseries}% Theorem head font
{}% Punctuation after theorem head
{\newline} % Space after theorem head
{\thmname{#1}\thmnumber{\@ifnotempty{#1}{ }\@upn{#2}}%
	\thmnote{ {\bfseries {: #3}}}}% Theorem head spec
\makeatother

\makeatletter
\newtheoremstyle{GSHSnonumber} % name
{\topsep}% Space above
{\topsep}% Space below
{}% Body font
{}% Indent amount
{\bfseries}% Theorem head font
{}% Punctuation after theorem head
{\newline} % Space after theorem head
{\thmname{#1}\thmnote{ {\bfseries {: #3}}}}% Theorem head spec
\makeatother


\theoremstyle{GSHScustom}
\newtheorem{theorem}{Theorem}[chapter]

\theoremstyle{GSHSnonumber}
\newtheorem{plaintheorem}{Extra Theorem}[chapter]
\newtheorem{lemma}{Lemma}[chapter]

%% for reducing space above title of chapter
\titleformat{\chapter}[display]
{\normalfont\huge\bfseries}{\chaptertitlename\ \thechapter}{20pt}{\Huge}
\titlespacing*{\chapter}
{0pt}{-20pt}{40pt}
%%%%%%%%% 여기까지 %%%%%%%%


\renewcommand{\labelenumi}{\alph{enumi}.}

\newcommand{\Rn}{$\mathbb{R}^{n}$} % 매번 치기 귀찮아서 아예 명령어를 만들어버림
\newcommand{\Rm}{$\mathbb{R}^{m}$}
\newcommand{\vn}{$ \textbf{v}_1, \textbf{v}_2, \cdots, \textbf{v}_n  $}
\newcommand{\vm}{$ \textbf{v}_1, \textbf{v}_2, \cdots, \textbf{v}_m  $}
\newcommand{\vk}{$ \textbf{v}_1, \textbf{v}_2, \cdots, \textbf{v}_k  $}
\newcommand\inv[1]{#1\raisebox{1.15ex}{$\scriptscriptstyle-\!1$}}

% augmented matrix 만드는 명령어 %
\makeatletter
\renewcommand*\env@matrix[1][*\c@MaxMatrixCols c]{%
	\hskip -\arraycolsep
	\let\@ifnextchar\new@ifnextchar
	\array{#1}}
\makeatother

%\renewcommand{\listtheoremname}{List of Theorems}

\begin{document}
\maketitle
%\tableofcontents
%목차는 딱히 필요없어 보임

% 해당 단원의 tex 파일로 이동하여 편집하세요.
% 참고 : chapter 뿐 아니라 section 마다 이렇게 문서를 나누어 편집하고 싶다는 욕심이 들지도 모르겠지만, 문서가 나뉠 때마다 clearpage 가 되기 때문에 비추천...

\chapter*{Preface and License}
\addcontentsline{toc}{chapter}{Preface and License}
%\lipsum[1-2]
This work is a theorem archive for `Linear Algebra' subject in Gyeonggi Science High School for the Gifted. It is initiated and mainly written by 황동욱, and being revised by 하석민, 박승원.

Each `Theorem' is identically numbered as textbook. (Except Chapter 3.5) On the other hand, `Extra Theorem' is things that aren't discussed or proved in textbook.

Anyway, good luck on your mid-term exam on Friday!

Theorems in this archive can have some errors. Please come to us if you find some of them, then we will revise them.

This work is licensed under a Creative Commons Attribution-NonCommercial-ShareAlike 4.0 International License.
\begin{figure}[h]
	\centering
	\includegraphics[width=0.4\textwidth]{by-nc-sa.pdf}
\end{figure} \footnote{\url{http://creativecommons.org/licenses/by-nc-sa/4.0/}}
\addcontentsline{toc}{chapter}{Table of Contents}
\tableofcontents
\addcontentsline{toc}{chapter}{List of Theorems}

%\begingroup
%\let\clearpage\relax
\listoftheorems[ignoreall,show={theorem}]
\vspace{2cm}
%\endgroup

\addcontentsline{toc}{chapter}{List of Extra Theorems}
\renewcommand{\listtheoremname}{List of Extra Theorems}

\begingroup
\let\clearpage\relax
\listoftheorems[ignoreall,show={plaintheorem,lemma}]
\endgroup


\chapter{Vectors}

% \iffalse 부터 \fi 까지는 주석처리됨
\iffalse
\begin{theorem}[Algebraic Properties of Vectors in \Rn] \label{properties_vectors}
	Let $\vec{u}$, $\vec{v}$, $\vec{w}$ be vectors in \Rn and let $c$ and $d$ be scalars. Then
	\begin{enumerate}
		\item $\vec{u}+\vec{v}=\vec{v}+\vec{u}$ : Commutativity
		\item $(\vec{u}+\vec{v})+\vec{w}=\vec{u}+(\vec{v}+\vec{w})$ : Associativity
		\item $\vec{u}+\vec{0}=\vec{u}$
		\item $\vec{u}+(-\vec{u})=\vec{0}$
		\item $c(\vec{u}+\vec{v})=c\vec{u}+c\vec{v}$ : Distributivity
		\item $(c+d)\vec{u}=c\vec{u}+d\vec{u}$ : Distributivity
		\item $c(d\vec{u})=(cd)\vec{u}$
		\item $1\vec{u}=\vec{u}$
	\end{enumerate}
\end{theorem}

A theorem shown above is thm.\ref{properties_vectors}.

\begin{theorem}
	Let $\vec{u}$, $\vec{v}$, and $\vec{w}$ be vectors in $\mathbb{R}^{n}$ and let $c$ be a scalar. Then
	\begin{enumerate}
		\item $\vec{u}\cdot\vec{v}=\vec{v}\cdot\vec{u}$
		\item $\vec{u}\cdot(\vec{v}+\vec{w}) = \vec{u}\cdot\vec{v}+\vec{u}\cdot\vec{w}$
		\item $(c\vec{u})\cdot\vec{v}=c(\vec{u}\cdot\vec{v})$
		\item $\vec{u}\cdot\vec{u}\geq 0$ and $\vec{u}\cdot\vec{u}=0$ iff $\vec{u}=\vec{0}$
	\end{enumerate}
\end{theorem}
\begin{proof}
	\begin{equation} \label{pi}
	\pi = 3.141592...
	\end{equation}
	By \eqref{pi}, proved! \qedhere
\end{proof}
\fi

\section{Terminology relating to vectors}

A vector can be represented either in geometric way or algebraic way. In geometric definition of vectors, a vector is a \textbf{directed line segment}.
A vector from point $A$ (\textbf{initial point}, or \textbf{tail}) to point $B$ (\textbf{terminal point}, or \textbf{head}) is denoted as $\overrightarrow{AB}$.
Vectors with their tails in the origin is called \textbf{position vectors}, and they are at \textbf{standard position}.
\\
In algebraic view of vectors, a vector is an \textbf{ordered pair} of \textbf{components}. We denote the set of all vectors containing $n$ components in $\mathbb{R}$ as \Rn. Similarly, set of all vectors containing $n$ integer components is $\mathbb{Z}^{n}$.
\\
A vector is written in form of \textbf{column vectors} and \textbf{row vectors}. We use square brackets for denoting vectors, such as 
\begin{displaymath}
\textbf{v} = \begin{bmatrix}4 \\ 1 \\ 6\end{bmatrix},
\begin{bmatrix} 4, & 1, & 6 \end{bmatrix}
\end{displaymath}
\\
A \textbf{zero vector} is a vector which components are all zero. A zero vector is denoted as $\textbf{0}$.
\\
Two vectors are equal if and only if all the components of two vectors are equal. (Of course, the number of components should be same.)
\\
\textbf{Standard unit Vectors} have components which one of them is 1 and rest of them are all 0. Unit vector which has 1 in $i$th component is denoted as $\textbf{e}_i$, and
\begin{displaymath}
\textbf{e}_i = \begin{bmatrix} 0, & 0, & \cdots & 1, & \cdots & 0 \end{bmatrix}
\end{displaymath}
* \textit{Note} We should always denote vector either with arrows ($\vec{v}$), or with boldface letters ($\textbf{v}$). Scalar denotations ($v$) are not allowed.
\\
A set of vectors with $n$ components taken from finite set of integers $\mathbb{Z}_m = \{0, 1, 2,\cdots,m-1\}$ is denoted as $\mathbb{Z}^{n}_{m}$.
$\mathbb{Z}^{n}_{m}$ is closed with respect to operations of vector addition and scalar multiplication (which is defined later). We can perform those operations in same way, but with modulo operations.
Vectors in $\mathbb{Z}^{n}_2$ (all components are 0 or 1) are called \textbf{binary vectors}.

\section{Basic Operations of Vectors}

\textit{Definition.}
Let $\textbf{u}, \textbf{v} \in$ \Rn be $\textbf{u}=\begin{bmatrix} u_1, & u_2, & \cdots & u_n \end{bmatrix},
\textbf{v}=\begin{bmatrix} v_1, & v_2, & \cdots & v_n \end{bmatrix}.$
Then their \textbf{sum} $\textbf{u}+\textbf{v}$ is defined as
\begin{displaymath}
\textbf{u}+\textbf{v}=\begin{bmatrix}u_1+v_1,&u_2+v_2,&\cdots&u_n+v_n\end{bmatrix}
\end{displaymath}
\\
\textit{Definition.}
Let $\textbf{v} \in$ \Rn be $\textbf{v}=\begin{bmatrix} v_1&v_2&\cdots&v_n \end{bmatrix}$, and let $c$ be a real number. Then their \textbf{scalar multiple} $c\textbf{v}$ is denifed as
\begin{displaymath}
c\textbf{v}=\begin{bmatrix} cv_1, & cv_2, & \cdots & cv_n \end{bmatrix}
\end{displaymath}
\\
\textit{Definition.}
A \textbf{negative} of $\textbf{v}$ is defined as $-\textbf{v}=(-1)\textbf{v}$.
\\\\
\textit{Definition.}
The \textbf{difference} of vectors is defined as $\textbf{u}-\textbf{v}=\textbf{u}+(-\textbf{v})$.
\\\\
\textit{Definition.}
Two vectors are \textbf{parallel} if and only if one vector is a scalar multiple of another. (Thus, zero vector is parallel with all vectors.)

\begin{theorem}[Algebraic Properties of Vectors : Basic Vector Operations]
Let \textbf{u},\textbf{v}, and \textbf{w} be vectors in \Rn and let $c$ and $d$ be scalars.
    \begin{enumerate}
        \item $\textbf{u} + \textbf{v} = \textbf{v} + \textbf{u}$ (Commutative Property of Vector Addition)
        \item $(\textbf{u} + \textbf{v}) + \textbf{u} = \textbf{u} + (\textbf{v} + \textbf{w})$ (Associative Property of Vector Addition)
        \item $\textbf{u} + \textbf{0} = \textbf{u}$
        \item $\textbf{u} + (-\textbf{u}) = \textbf{0}$
        \item $c(\textbf{u} + \textbf{v}) = c\textbf{u} + c\textbf{v}$ (Left-Distributive Property of Scalar Multiplication over Vector Addition)
        \item $(c+d)\textbf{u} = c\textbf{u} + d\textbf{u}$ (Right-Distributive Property of Scalar Multiplication over Vector Addition)
        \item $c(d\textbf{u}) = (cd)\textbf{u}$
        \item $1\textbf{u} = \textbf{u}$
    \end{enumerate}
\end{theorem}

\begin{proof}
Let $\textbf{u}, \textbf{v}, \textbf{w} \in$ \Rn be
$\textbf{u} = \begin{bmatrix} u_1, & u_2, & \cdots & u_n \end{bmatrix},
 \textbf{v} = \begin{bmatrix} v_1, & v_2, & \cdots & v_n \end{bmatrix},
 \textbf{w} = \begin{bmatrix} w_1, & w_2, & \cdots & w_n \end{bmatrix}$,
 and let $c$, $d$ be scalars.
 
    \begin{enumerate}
        \item
            \begin{align*}
                \textbf{u}+\textbf{v}
                &= \begin{bmatrix} u_1+v_1, & u_2+v_2, & \cdots & u_n+v_n \end{bmatrix} \\
                &= \begin{bmatrix} v_1+u_1, & v_2+u_2, & \cdots & v_n+u_n \end{bmatrix} = \textbf{v} + \textbf{u}
            \end{align*}
        
        \item
            \begin{align*}
                (\textbf{u}+\textbf{v})+\textbf{w}
                &= \begin{bmatrix} u_1+v_1, & u_2+v_2, & \cdots & u_n+v_n \end{bmatrix} +
                \begin{bmatrix} w_1, & w_2, & \cdots & w_n \end{bmatrix} \\
                &= \begin{bmatrix} (u_1+v_1)+w_1, & (u_2+v_2)+w_2, & \cdots & (u_n+v_n)+w_n \end{bmatrix} \\
                &= \begin{bmatrix} u_1+(v_1+w_1), & u_2+(v_2+w_2), & \cdots & u_n+(v_n+w_n) \end{bmatrix} \\
                &= \textbf{u}+(\textbf{v}+\textbf{w})
            \end{align*}
        
        \item
            \begin{align*}
                \textbf{u}+\textbf{0}
                &= \begin{bmatrix} u_1+0, & u_2+0, & \cdots & u_n+0 \end{bmatrix} \\
                &= \begin{bmatrix} u_1, & u_2, & \cdots & u_n \end{bmatrix} = \textbf{u}
            \end{align*}
        
        \item (Excercise 1.1 24)
            \begin{align*}
                \textbf{u}+(-\textbf{u})
                &= \begin{bmatrix} u_1+(-u_1), & u_2+(-u_2), & \cdots & u_n+(-u_n) \end{bmatrix} \\
                &= \begin{bmatrix} 0, & 0, & \cdots & 0 \end{bmatrix} = \textbf{0}
            \end{align*}
        
        \item (Excercise 1.1 24)
            \begin{align*}
                c(\textbf{u}+\textbf{v})
                &= c \begin{bmatrix} u_1+v_1, & u_2+v_2, & \cdots & u_n+v_n \end{bmatrix} \\
                &= \begin{bmatrix} c(u_1+v_1), & c(u_2+v_2), & \cdots & c(u_n+v_n) \end{bmatrix} \\
                &= \begin{bmatrix} cu_1+cv_1, & cu_2+cv_2, & \cdots & cu_n+cv_n \end{bmatrix} \\
                &= \begin{bmatrix} cu_1, & cu_2, & \cdots & cu_n \end{bmatrix} +
                   \begin{bmatrix} cv_1, & cv_2, & \cdots & cv_n \end{bmatrix} \\
                &= c\textbf{u}+c\textbf{v}
            \end{align*}
        
        \item (Excercise 1.1 24)
            \begin{align*}
                (c+d)\textbf{u}
                &= \begin{bmatrix} (c+d)u_1, & (c+d)u_2, & \cdots & (c+d)u_n \end{bmatrix} \\
                &= \begin{bmatrix} cu_1+du_1, & cu_2+du_2, & \cdots & cu_n+du_n \end{bmatrix} \\
                &= \begin{bmatrix} cu_1, & cu_2, & \cdots & cu_n \end{bmatrix} +
                   \begin{bmatrix} du_1, & du_2, & \cdots & du_n \end{bmatrix} \\
                &= c\textbf{u}+d\textbf{u}
            \end{align*}
        
       \item (Excercise 1.1 24)
            \begin{align*}
                c(d\textbf{u})
                &= c \begin{bmatrix} du_1, & du_2, & \cdots & du_n \end{bmatrix} \\
                &= \begin{bmatrix} cdu_1, & cdu_2, & \cdots & cdu_n \end{bmatrix} \\
                &= cd\begin{bmatrix} u_1, & u_2, & \cdots & u_n \end{bmatrix} \\
                &= (cd)\textbf{u}
            \end{align*}
    \end{enumerate}
\end{proof}

\section{Dot Product}

\textit{Definition.}
Let $\textbf{u}$,$\textbf{v} \in$ \Rn be $\textbf{u} = \begin{bmatrix} u_1, & u_2, & \cdots & u_n \end{bmatrix},
\textbf{v} = \begin{bmatrix} v_1 & v_2 & \cdots & v_n \end{bmatrix}$.
Then the \textbf{dot product} $\textbf{u}\cdot\textbf{v}$ of $\textbf{u}$ and $\textbf{v}$ is defined as
\begin{displaymath}
    \textbf{u}\cdot\textbf{v}=u_1v_1+u_2v_2+\cdots+u_nv_n
\end{displaymath}

\begin{theorem}[Properties of Dot Product]
    Let \textbf{u},\textbf{v}, and \textbf{w} be vectors in \Rn and let $c$ be a scalar.
    \begin{enumerate}
        \item
        $\textbf{u}\cdot\textbf{v}=\textbf{v}\cdot\textbf{u}$ (Commutative Property of Dot Product Operation)
        \item
        $\textbf{u}\cdot(\textbf{v}+\textbf{w})=\textbf{u}\cdot\textbf{v}+\textbf{u}\cdot\textbf{w}$ (Distributive Property of Dot Product Operation over Vector Addition)
        \item
        $(c\textbf{u})\cdot\textbf{v}=c(\textbf{u}\cdot\textbf{v})$
        \item
        $\textbf{u}\cdot\textbf{u}\ge0$ and $\textbf{u}\cdot\textbf{u}=0$ if and only if $\textbf{u}=\textbf{0}$
    \end{enumerate}
\end{theorem}

\begin{proof}
    Let $\textbf{u}, \textbf{v}, \textbf{w} \in$ \Rn be
    $\textbf{u} = \begin{bmatrix} u_1, & u_2, & \cdots & u_n \end{bmatrix},
    \textbf{v} = \begin{bmatrix} v_1, & v_2, & \cdots & v_n \end{bmatrix},
    \textbf{w} = \begin{bmatrix} w_1, & w_2, & \cdots & w_n \end{bmatrix}$,
    and let $c$ be a scalar.
    
    \begin{enumerate}
        \item
            \begin{align*}
                \textbf{u}\cdot\textbf{v}
                &= u_1v_1+u_2v_2+\cdots+u_nv_n \\
                &= v_1u_1+v_2u_2+\cdots+v_nu_n = \textbf{v}\cdot\textbf{u}
            \end{align*}
        \item
            \begin{align*}
                \textbf{u}\cdot(\textbf{v}+\textbf{w})
                &= \begin{bmatrix} u_1, & u_2, & \cdots & u_n \end{bmatrix} \cdot
                   \begin{bmatrix} v_1+w_1, & v_2+w_2, & \cdots & v_n+w_n \end{bmatrix} \\
                &= u_1(v_1+w_1) + u_2(v_2+w_2) + \cdots + u_n(v_n+w_n) \\
                &= (u_1v_1+u_2v_2+\cdots+u_nv_n)+(u_1w_1+u_2w_2+\cdots+u_nw_n) \\
                &= \textbf{u}\cdot\textbf{v}+\textbf{u}\cdot\textbf{w}
            \end{align*}
        \item
            \begin{align*}
                (c\textbf{u})\cdot\textbf{v}
                &= \begin{bmatrix} cu_1, & cu_2, & \cdots & cu_n \end{bmatrix} \cdot
                   \begin{bmatrix} v_1, & v_2, & \cdots & v_n \end{bmatrix} \\
                &= cu_1v_1 + cu_2v_2 + \cdots + cu_nv_n \\
                &= c(u_1v_1 + u_2v_2 + \cdots + u_nv_n) = c(\textbf{u}\cdot\textbf{v})
            \end{align*}
        \item
            \begin{displaymath}
                \textbf{u}\cdot\textbf{u} = u_1^2 + u_2^2 + \cdots + u_n^2 \ge 0
            \end{displaymath}
            and since $u_1^2+u_2^2+\cdots+u_n^2=0$ if and only if $u_1=u_2=\cdots=u_n=0$, $\textbf{u}\cdot\textbf{u}=0$ if and only if $\textbf{u}=\textbf{0}$.
    \end{enumerate}
\end{proof}

\noindent
\textit{Definition.}
The \textbf{length} or \textbf{norm} of $\textbf{v} \in$ \Rn is defined as
\begin{displaymath}
    \Vert\textbf{v}\Vert=\sqrt{\textbf{v}\cdot\textbf{v}}
\end{displaymath}

\begin{theorem}[Properties of Norm]
    For all vectors $\textbf{u}$, $\textbf{v}$ in \Rn and scalar $c$,
    \begin{enumerate}
        \item $\Vert\textbf{v}\Vert=0$ if and only if $\textbf{v}=\textbf{0}$
        \item $\Vert c\textbf{v}\Vert=|c|\Vert\textbf{v}\Vert$
    \end{enumerate}
\end{theorem}

\begin{proof}
    Let $\textbf{u}, \textbf{v} \in$ \Rn be
    $\textbf{u} = \begin{bmatrix} u_1, & u_2, & \cdots & u_n \end{bmatrix},
    \textbf{v} = \begin{bmatrix} v_1, & v_2, & \cdots & v_n \end{bmatrix}$,
    and let $c$ be a scalar.
    \begin{enumerate}
        \item
        Since $\Vert\textbf{v}\Vert=\sqrt{v_1^2+v_2^2+\cdots+v_n^2}=0$ if and only if $v_1=v_2=\cdots=v_n=0$, $\Vert\textbf{v}\Vert=0$ if and only if $\textbf{v}=\textbf{0}$.
        \item
        \begin{align*}
            \Vert c\textbf{v}\Vert
            &= \sqrt{(cv_1)^2 + (cv_2)^2 + \cdots + (cv_n)^2} \\
            &= |c| \sqrt{v_1^2 + v_2^2 + \cdots + v_n^2} = |c|\Vert\textbf{v}\Vert
        \end{align*}
    \end{enumerate}
\end{proof}

\begin{theorem}[The Cauchy-Schwarz Inequality]
    For all vectors $\textbf{u}$, $\textbf{v}$ in \Rn,
    $|\textbf{u}\cdot\textbf{v}|\le\Vert\textbf{u}\Vert\Vert\textbf{v}\Vert$.
\end{theorem}

\begin{proof}
    (Exercise 1.2 71) For any $t \in \mathbb{R}$, $(u_1t-v_1)^2+(u_2t-v_2)^2+\cdots+(u_nt-v_n)^2 \ge 0$.
    Therefore, the determinant of the quatratic equation $(u_1^2 + u_2^2 + \cdots + u_n^2)t^2 - 2(u_1v_1 + u_2v_2 + \cdots + u_nv_n)t + (v_1^2 + v_2^2 + \cdots + v_n^2) = 0$ is negative or $0$. Then
    \begin{align*}
        D/4=(u_1v_1 + u_2v_2 + \cdots + u_nv_n)^2 &- (u_1^2 + u_2^2 + \cdots + u_n^2)(v_1^2 + v_2^2 + \cdots + v_n^2) \le 0 \\
        |u_1v_1 + u_2v_2 + \cdots + u_nv_n| &\le \sqrt{u_1^2 + u_2^2 + \cdots + u_n^2}\sqrt{v_1^2 + v_2^2 + \cdots + v_n^2} \\
        \therefore |\textbf{u}\cdot\textbf{v}| &\le \Vert\textbf{u}\Vert\Vert\textbf{v}\Vert
    \end{align*}
\end{proof}

\begin{theorem}[The Triangle Inequality]
    For all vectors $\textbf{u}$, $\textbf{v}$ in \Rn,
    $\Vert\textbf{u}+\textbf{v}\Vert \le \Vert\textbf{u}\Vert + \Vert\textbf{v}\Vert$.
\end{theorem}

\begin{proof}
    \begin{align*}
        \Vert\textbf{u}+\textbf{v}\Vert^2
        &= (\textbf{u}+\textbf{v})\cdot(\textbf{u}+\textbf{v}) \\
        &= \textbf{u}\cdot\textbf{u}+2(\textbf{u}\cdot\textbf{v})+\textbf{v}\cdot\textbf{v} \\
        &\le \Vert\textbf{u}\Vert^2 + 2|\textbf{u}\cdot\textbf{v}| + \Vert\textbf{v}\Vert^2 \\
        &\le \Vert\textbf{u}\Vert^2 + 2\Vert\textbf{u}\Vert\Vert\textbf{v}\Vert + \Vert\textbf{v}\Vert^2 \\
        &= (\Vert\textbf{u}\Vert + \Vert\textbf{v}\Vert)^2
    \end{align*}
\end{proof}

\noindent
\textit{Definition.}
The \textbf{distance} between $\textbf{u}, \textbf{v} \in$ \Rn is defined as
\begin{displaymath}
    \textnormal{d}(\textbf{u},\textbf{v})=\Vert\textbf{u}-\textbf{v}\Vert
\end{displaymath}
\textit{Definition.}
The \textbf{angle} between nonzero vectors $\textbf{u}, \textbf{v} \in$ \Rn is defined as
\begin{displaymath}
    \cos\theta=\frac{\textbf{u}\cdot\textbf{v}}{\Vert\textbf{u}\Vert\Vert\textbf{v}\Vert}
\end{displaymath}
* \textit{Note.} Since $|\textbf{u}\cdot\textbf{v}| \le \Vert\textbf{u}\Vert\Vert\textbf{v}\Vert$, $-1\le\cos\theta\le1$, which corresponds with properties of cosine function.
\\
\textit{Definition.} $\textbf{u}$ and $\textbf{v}$ are \textbf{orthogonal} to each other if and only if $\textbf{u}\cdot\textbf{v}=0$.
Since $\textbf{0}\cdot\textbf{v}=0$ for every vector $\textbf{v}$ in \Rn, $\textbf{0}$ is orthonogal with every vector.

\begin{theorem}[Pythagoras' Theorem]
    For vectors $\textbf{u}$ and $\textbf{v}$ in \Rn, $\Vert\textbf{u}+\textbf{v}\Vert^2 = \Vert\textbf{u}\Vert^2 + \Vert\textbf{v}\Vert^2$ if and only if $\textbf{u}$ and $\textbf{v}$ are orthogonal.
\end{theorem}

\begin{proof}
    Since $\Vert\textbf{u}+\textbf{v}\Vert^2 = \Vert\textbf{u}\Vert^2 + 2(\textbf{u}\cdot\textbf{v}) + \Vert\textbf{v}\Vert^2$, $\Vert\textbf{u}+\textbf{v}\Vert^2 = \Vert\textbf{u}\Vert^2 + \Vert\textbf{v}\Vert^2$ if and only if $\textbf{u}\cdot\textbf{v}=0$.
\end{proof}

\noindent
\textit{Definition.} For vectors $\textbf{u}, \textbf{v} \in$ \Rn and $\textbf{u} \neq \textbf{0}$, the \textbf{projection of $\textbf{v}$ onto $\textbf{u}$} is denoted as $\textnormal{proj}_\textbf{u}(\textbf{v})$ and defined as
\begin{displaymath}
    \textnormal{proj}_\textbf{u}(\textbf{v})=\left(\frac{\textbf{u}\cdot\textbf{v}}{\textbf{u}\cdot\textbf{u}}\right)\textbf{u}
\end{displaymath}
\\
\textit{Derivation.} The projection of $\textbf{v}$ onto $\textbf{u}$ is scalar multiple of $\textbf{u}$, and its scale is $\Vert\textbf{v}\Vert\cos\theta$. Therefore,
\begin{align*}
    \textnormal{proj}_\textbf{u}(\textbf{v})
    &= \Vert\textbf{v}\Vert\cos\theta\left(\frac{1}{\Vert\textbf{u}\Vert}\right)\textbf{u} \\
    &= \Vert\textbf{v}\Vert\left(\frac{\textbf{u}\cdot\textbf{v}}{\Vert\textbf{u}\Vert\Vert\textbf{v}\Vert}\right)\left(\frac{1}{\Vert\textbf{u}\Vert}\right)\textbf{u} \\
    &= \left(\frac{\textbf{u}\cdot\textbf{v}}{\Vert\textbf{u}\Vert^2}\right)\textbf{u} \\
    &= \left(\frac{\textbf{u}\cdot\textbf{v}}{\textbf{u}\cdot\textbf{u}}\right)\textbf{u}
\end{align*}

\section{Geometry in Vectors}

\begin{table} [h]
	\begin{center}
		\begin{tabular}{cccccc}
			\hline
			& \textbf{Normal Form} & \textbf{General Form} & \textbf{Vector Form} & \textbf{Parametric Form} \\
			\hline
			\textbf{Lines} &
			$\textbf{n}\cdot\textbf{x} = \textbf{n}\cdot\textbf{p}$ &
			$ax +  by = c$ &
			$\textbf{x} = \textbf{p} + t\textbf{d}$ &
			$\begin{cases}
				x = p_x + td_x \\
				y = p_y + td_y
			\end{cases}$ \\
			\hline
		\end{tabular}
		\\ ($t \in \mathbb{R}$)
		\caption{
			Equations of Lines in $\mathbb{R}^2$.
		}
	\end{center}
\end{table}

\begin{table}[h]
	\begin{center}
		\begin{tabular}{cccccc}
			\hline
			& \textbf{Normal Form} & \textbf{General Form} & \textbf{Vector Form} & \textbf{Parametric Form} \\
			\hline
			\textbf{Lines} &
			$\begin{cases}
				 \textbf{n}_1\cdot\textbf{x} = \textbf{n}_1\cdot\textbf{p}_1 \\
				 \textbf{n}_2\cdot\textbf{x} = \textbf{n}_2\cdot\textbf{p}_2
			\end{cases}$ &
			$\begin{cases}
				a_1x + b_1y + c_1z = d_1 \\
				a_2x + b_2y + c_2z = d_2
			\end{cases}$ &
			$\textbf{x} = \textbf{p} + t\textbf{d}$ & 
			$\begin{cases}
				x = p_x + td_x \\
				y = p_y + td_y \\
				z = p_z + td_z
			\end{cases}$ \\
			\hline
			\textbf{Planes} &
			$\textbf{n}\cdot\textbf{x} = \textbf{n}\cdot\textbf{p}$ &
			$ax + by + cz = d$ &
			$\textbf{x} = \textbf{p} + s\textbf{u} + t\textbf{v}$ &
			$\begin{cases}
				x = p_x + su_x + tv_x \\
				y = p_y + su_y + tv_y \\
				z = p_z + su_z + tv_z
			\end{cases}$ \\
			\hline
		\end{tabular}
		\\ ($t, s \in \mathbb{R}$)
		\caption{
			Equations of Lines and Planes in $\mathbb{R}^3$.
		}
	\end{center}
\end{table}

\noindent
The distance from the point $A$ to a line $l$: $\textbf{x}=\textbf{p}+t\textbf{d}$ is represented as
\begin{align*}
	\textnormal{d}(A,l)
	&=\left\| (\textbf{a}-\textbf{p})-\textnormal{proj}_\textbf{d}(\textbf{a}-\textbf{p})\right\| \\
	&= \left\| (\textbf{a}-\textbf{p}) - \left(\frac{\textbf{d}\cdot(\textbf{a}-\textbf{p})}{\textbf{d}\cdot\textbf{d}}\right)\textbf{d} \right\|
\end{align*}
In $\mathbb{R}^2$, if $A(x_0,y_0)$ and $l$ : $ax + by + c = 0$,
\begin{align*}
	\textnormal{d}(A, l) = \frac{|ax_0+by_0+c|}{\sqrt{a^2+b^2}}
\end{align*}
\textit{Derivation.} (Excercise 1.3 39)
Let $\textbf{a}=\begin{bmatrix} x_0 \\ y_0 \end{bmatrix}$,
$\textbf{p}=\begin{bmatrix} x_1 \\ y_1 \end{bmatrix}$,
and $\textbf{d}=\begin{bmatrix} b \\ -a \end{bmatrix}$.
\begin{align*}
	\textnormal{d}(A,l)
	&= \left\|(\textbf{a}-\textbf{p})-\textnormal{proj}_\textbf{d}(\textbf{a}-\textbf{p})\right\| \\
	&= \left\|\begin{bmatrix}
		(x_0-x_1)-\frac{b(x_0-x_1)-a(y_0-y_1)}{a^2+b^2}b \\
		(y_0-y_1)+\frac{b(x_0-x_1)-a(y_0-y_1)}{a^2+b^2}a
	\end{bmatrix}\right\| \\
	&= \left\|\begin{bmatrix}
		\frac{a}{a^2+b^2}\{(ax_0+by_0) - (ax_1+by_1)\} \\
		\frac{b}{a^2+b^2}\{(ax_0+by_0) - (ax_1+by_1)\}
	\end{bmatrix}\right\| \\
	&= \left\|\begin{bmatrix}
	\frac{a}{a^2+b^2}(ax_0+by_0+c) \\
	\frac{b}{a^2+b^2}(ax_0+by_0+c)
	\end{bmatrix}\right\| \\
	&= \frac{|ax_0+by_0+c|}{\sqrt{a^2+b^2}}
\end{align*}

\noindent
The distance from the point $A$ to a plane $\mathcal{P}$ (or a hyperplane) with a normal vector $\textbf{n}$, and point $P$ is on the plane is represented as
\begin{align*}
	\textnormal{d}(A,\mathcal{P})
	&= 
	\left\| \textnormal{proj}_\textbf{n}(\textbf{a}-\textbf{p}) \right\| \\
	&= \left\| \frac{\textbf{n}\cdot(\textbf{a}-\textbf{p})}{\textbf{n}\cdot\textbf{n}}\right\|  \left\| \textbf{n} \right\|
\end{align*}
In $\mathbb{R}^3$, if $A(x_0,y_0,z_0)$ and $\mathcal{P}$: $ax+by+cz+d=0$,
\begin{align*}
	\textnormal{d}(A,\mathcal{P})=\frac{|ax_0+by_0+cz_0+d|}{\sqrt{a^2+b^2+c^2}}
\end{align*}
\textit{Derivation.} (Excercise 1.3 40)
Let $\textbf{a}=\begin{bmatrix} x_0 \\ y_0 \\ z_0 \end{bmatrix}$,
$\textbf{p}=\begin{bmatrix} x_1 \\ y_1 \\ z_1 \end{bmatrix}$,
and $\textbf{n}=\begin{bmatrix} a \\ b \\ c \end{bmatrix}$.
\begin{align*}
	\textnormal{d}(A,\mathcal{P})
	&= |\frac{\textbf{n}\cdot(\textbf{a}-\textbf{p})}{\textbf{n}\cdot\textbf{n}}| \Vert \textbf{n} \Vert \\
	&= |\frac{a(x_0-x_1)+b(y_0-y_1)+c(z_0-z_1)}{a^2+b^2+c^2}|\sqrt{a^2+b^2+c^2} \\
	&= \frac{|(ax_0+by_0+cz_0)-(ax_1+by_1+cz_1)|}{\sqrt{a^2+b^2+c^2}} \\
	&= \frac{|ax_0+by_0+cz_0+d|}{\sqrt{a^2+b^2+c^2}}
\end{align*}

\section{Solutions for Excercises Worthy to Solve}
\begin{enumerate}
	\item \textbf{Excercise 1.2 55}
		\begin{proof}
			\begin{align*}
				\textnormal{d}(\textbf{u},\textbf{v})
				&= \Vert\textbf{u}-\textbf{v}\Vert \\
				&= \Vert (-1)(\textbf{v}-\textbf{u})\Vert
				= |-1|\Vert\textbf{v}-\textbf{u}\Vert \\
				&= \Vert\textbf{v}-\textbf{u}\Vert = \textnormal{d}(\textbf{v},\textbf{u})
			\end{align*}
		\end{proof}
	\item \textbf{Excercise 1.2 56}
		\begin{proof}
			\begin{align*}
			\textnormal{d}(\textbf{u},\textbf{w}) 
			= \Vert\textbf{u} - \textbf{w}\Vert
			&= \Vert(\textbf{u}-\textbf{v})+(\textbf{v}-\textbf{w})\Vert \\
			&\le \Vert\textbf{u}-\textbf{v}\Vert + \Vert\textbf{v}-\textbf{w}\Vert
			= \textnormal{d}(\textbf{u},\textbf{v}) + \textnormal{d}(\textbf{v},\textbf{w})
			\end{align*}
		\end{proof}
	\item \textbf{Excercise 1.2 57}
		\begin{proof}
			By Theorem 1.3 (c), $\textnormal{d}(\textbf{u},\textbf{v}) = \Vert\textbf{u}-\textbf{v}\Vert = 0$ if and only if $\textbf{u}-\textbf{v}=\textbf{0}$. Thus, $\textnormal{d}(\textbf{u},\textbf{v})=0$ if and only if $\textbf{u}=\textbf{v}$.
		\end{proof}
	\item \textbf{Excercise 1.2 59}
		\begin{proof}
			\begin{align*}
				\Vert\textbf{u}\Vert
				&= \Vert(\textbf{u}-\textbf{v})+\textbf{v}\Vert \\
				&\le \Vert\textbf{u}-\textbf{v}\Vert + \Vert\textbf{v}\Vert \\
				\therefore \Vert\textbf{u}&-\textbf{v}\Vert \ge \Vert\textbf{u}\Vert - \Vert\textbf{v}\Vert
			\end{align*}
		\end{proof}
	\item \textbf{Excercise 1.2 60} \\\\
		\textit{Solution.}
		A well-known counterexample for $\textbf{u}\cdot\textbf{v}=\textbf{u}\cdot\textbf{w} \rightarrow \textbf{v}=\textbf{w}$ is the case when $\textbf{u}\cdot\textbf{v}=\textbf{u}\cdot\textbf{w}=0$. Vectors orthogonal with $\textbf{u}$ satisfy the condition, though they are not equal.
	\item \textbf{Excercise 1.2 69}
		\begin{proof}
			\begin{align*}
				\textbf{u}\cdot\textnormal{proj}_\textbf{u}(\textbf{v})
				&= \textbf{u}\cdot\left(\frac{\textbf{u}\cdot\textbf{v}}{\textbf{u}\cdot\textbf{u}}\textbf{u}\right) \\
				&= \left(\frac{\textbf{u}\cdot\textbf{v}}{\textbf{u}\cdot\textbf{u}}\right)\left(\textbf{u}\cdot\textbf{u}\right) = \textbf{u}\cdot\textbf{v} \\
				\therefore \textbf{u}\cdot(\textbf{v}-\textnormal{proj}_\textbf{u}(\textbf{v})) = \textbf{u}\cdot\textbf{v} &- \textbf{u}\cdot\textnormal{proj}_\textbf{u}(\textbf{v}) = \textbf{u}\cdot\textbf{v} - \textbf{u}\cdot\textbf{v} = 0
			\end{align*}
			Therefore, $\textbf{u}$ is orthogonal to $\textbf{v}-\textnormal{proj}_\textbf{u}(\textbf{v})$.
		\end{proof}
	\item \textbf{Excercise 1.2 70}
		\begin{proof}
			\begin{enumerate}
			\item
				\begin{align*}
					\textnormal{proj}_\textbf{u}(\textnormal{proj}_\textbf{u}(\textbf{v}))
					&= \left(\frac{\textbf{u}\cdot\left(\frac{\textbf{u}\cdot\textbf{v}}{\textbf{u}\cdot\textbf{u}}\textbf{u}\right)}{\textbf{u}\cdot\textbf{u}}\right)\textbf{u} \\
					&= \left(\frac{\left(\frac{\textbf{u}\cdot\textbf{v}}{\textbf{u}\cdot\textbf{u}}\right)\left(\textbf{u}\cdot\textbf{u}\right)}{\textbf{u}\cdot\textbf{u}}\right)\textbf{u} \\
					&= \left(\frac{\textbf{u}\cdot\textbf{v}}{\textbf{u}\cdot\textbf{u}}\right)\textbf{u} = \textnormal{proj}_\textbf{u}(\textbf{v})
				\end{align*}
			\item By Excercise 1.2 69, $\textbf{u}$ is orthogonal to $\textbf{v}-\textnormal{proj}_\textbf{u}(\textbf{v})$, thus $\textbf{u}\cdot(\textbf{v}-\textnormal{proj}_\textbf{u}(\textbf{v}))=0$. Therefore, $\textnormal{proj}_\textbf{u}(\textbf{v}-\textnormal{proj}_\textbf{u}(\textbf{v})) = \textbf{0}$.
			\end{enumerate}
		\end{proof}
	\item \textbf{Excercise 1.3 17}
		\begin{proof}
			Consider two lines with slopes $m_1$ and $m_2$. Then the direction vectors of two lines are $\begin{bmatrix} 1 \\ m_1 \end{bmatrix}$, and $\begin{bmatrix} 1 \\ m_2 \end{bmatrix}$, respectively. Since two vectors are orthogonal (or, perpendicular) if and only if $\begin{bmatrix} 1 \\ m_1 \end{bmatrix} \cdot \begin{bmatrix} 1 \\ m_2 \end{bmatrix} = 1 + m_1m_2 = 0$, two lines are perpendicular if and only if $m_1m_2 = -1$.
		\end{proof}
	\item \textbf{Excercise 1.3 41}
		\begin{proof}
			Suppose $\textbf{n}_1\cdot\textbf{x}_1 = c_1$ and $\textbf{n}_2\cdot\textbf{x}_2 = c_2$. Then the distance between two lines is
			\begin{align*}
				\Vert\textnormal{proj}_\textbf{n}(\textbf{x}_2-\textbf{x}_1)\Vert
				&= |\frac{\textbf{n}\cdot(\textbf{x}_2-\textbf{x}_1)}{\textbf{n}\cdot\textbf{n}}|\Vert\textbf{n}\Vert \\
				&= \frac{|\textbf{n}\cdot\textbf{x}_2 - \textbf{n}\cdot\textbf{x}_1|}{\Vert\textbf{n}\Vert} \\
				&= \frac{|c_2 - c_1|}{\Vert\textbf{n}\Vert}
			\end{align*}
		\end{proof}
	\item \textbf{Excercise 1.3 42}
		\begin{proof}
			Suppose $\textbf{n}_1\cdot\textbf{x}_1 = d_1$ and $\textbf{n}_2\cdot\textbf{x}_2 = d_2$. Then the distance between two planes is
			\begin{align*}
			\Vert\textnormal{proj}_\textbf{n}(\textbf{x}_2-\textbf{x}_1)\Vert
			&= |\frac{\textbf{n}\cdot(\textbf{x}_2-\textbf{x}_1)}{\textbf{n}\cdot\textbf{n}}|\Vert\textbf{n}\Vert \\
			&= \frac{|\textbf{n}\cdot\textbf{x}_2 - \textbf{n}\cdot\textbf{x}_1|}{\Vert\textbf{n}\Vert} \\
			&= \frac{|d_2 - d_1|}{\Vert\textbf{n}\Vert}
			\end{align*}
		\end{proof}
	\item \textbf{Excercise 1.3 47} \\
		\textit{Solution.}
			\begin{align*}
				\textbf{p}\cdot\textbf{n}
				&= (\textbf{v}-c\textbf{n})\cdot\textbf{n} \\
				&= \textbf{v}\cdot\textbf{n} - c\Vert\textbf{n}\Vert^2 = 0 \\
				\therefore c &= \frac{\textbf{v}\cdot\textbf{n}}{\Vert\textbf{n}\Vert^2} \\
				\textbf{p} &= \textbf{v} - \frac{\textbf{v}\cdot\textbf{n}}{\Vert\textbf{n}\Vert^2}\textbf{n}
			\end{align*}
\end{enumerate}
\chapter{Systems of Linear Equations}

\section{Terminology}
\textit{Definition.} A vector $\textbf{v}$ is a \textbf{linear combination} of vectors $\textbf{v}_1, \textbf{v}_2, \cdots, \textbf{v}_n$ if there exist scalars $c_1, c_2, \cdots, c_n$ such that
\begin{displaymath}
	c_1\textbf{v}_1 + c_2\textbf{v}_2 + \cdots + c_n\textbf{v}_n = \textbf{0}
\end{displaymath}
The scalars $c_1, c_2, \cdots, c_n$ are called \textbf{coefficients} of linear combination.
\\\\
\textit{Definition.} A \textbf{linear equation} in the $n$ variables $x_1, x_2, \cdots, x_n$ is an equation that can be written in the form of
\begin{displaymath}
	a_1x_1 + a_2x_2 + \cdots + a_nx_n = b
\end{displaymath}
where the \textbf{coefficients} $a_1, a_2, \cdots, a_n$ and the \textbf{constant term} $b$ are constants.
\\\\
A \textbf{set of linear equations} is a finite set of linear equations with same variables. A \textbf{solution} of a linear equation $a_1x_1 + a_2x_2 + \cdots + a_nx_n = b$ is a vector $\begin{bmatrix}
	x_1 & x_2 & \cdots & x_n
\end{bmatrix}$ which satisfies the equation. A solution of a set of linear equations is a vector which is simultaneously a solution of all linear equations in the system. A \textbf{solution set} of a system of linear equations is the set of all solutions of the system.
\\\\
A system of linear equations is \textbf{consistent} if there exists a solution. \textbf{Inconsistent} set of linear equations has an empty solution set.
\noindent Two linear systems are \textbf{equivalent} if they have same solution set.
\\\\
\noindent The \textbf{coefficient matrix} contains the coefficients of variables in the set of linear equations. The \textbf{augmented matrix} is the coefficient matrix augmented by a vector containing constant terms. For the system
\begin{align*}
	\begin{cases}
	a_{11}x_1 + a_{12}x_2 + \cdots + a_{1n}x_n = b_1 \\
	a_{21}x_1 + a_{22}x_2 + \cdots + a_{2n}x_n = b_2 \\
	\cdots \\
	a_{m1}x_1 + a_{m2}x_2 + \cdots + a_{mn}x_n = b_m
	\end{cases}
\end{align*} the coefficient matrix is
\begin{align*}
A = \begin{bmatrix}
	a_{11} & a_{12} & \cdots & a_{1n} \\
	a_{21} & a_{22} & \cdots & a_{2n} \\
	\vdots & \vdots &        & \vdots \\
	a_{m1} & a_{m2} & \cdots & a_{mn}
\end{bmatrix}
\end{align*} and the augmented matrix is
\begin{align*}
	\begin{bmatrix}
		A | \textbf{b}
	\end{bmatrix} = \begin{bmatrix}[cccc|c]
		a_{11} & a_{12} & \cdots & a_{1n} & b_1 \\
		a_{21} & a_{22} & \cdots & a_{2n} & b_2 \\
		\vdots & \vdots &        & \vdots & \vdots \\
		a_{m1} & a_{m2} & \cdots & a_{mn} & b_m \\
	\end{bmatrix}
\end{align*}
\textit{Definition.} A matrix is in \textbf{row echelon form} (REF) if:
\begin{enumerate}
	\item All rows consisting entirely of zeros are at the bottom.
	\item The \textbf{leading entry} (the first nonzero entry) of each rows is located to the left of any leading entries below it.
\end{enumerate}
\textit{Definition.} A matrix is in \textbf{reduced row echelon form} (RREF) if:
\begin{enumerate}
	\item It is in REF.
	\item All leading entries are 1. (\textbf{leading 1})
	\item Each columns containing a leading 1 has 0 everywhere else.
\end{enumerate}

\noindent REF of a matrix is not unique, but all matrices have unique RREF.

\noindent \\ \textit{Definition.} A system of linear equations is \textbf{homogeneous} if the constant term in each equation is zero.

\section{Solving Linear Systems}

A system of linear equations with \textit{real coefficients} has either a unique solution, or infinitely many solutions, or no solutions.
\begin{proof} (Another proof is shown at Theorem 3.22)
	Consider a consistent set of $m$ linear equations
	\begin{align*}
		\begin{cases}
			a_{11}x_1 + a_{12}x_2 + \cdots + a_{1n}x_n = b_1 \\
			a_{21}x_1 + a_{22}x_2 + \cdots + a_{2n}x_n = b_2 \\
			\cdots \\
			a_{m1}x_1 + a_{m2}x_2 + \cdots + a_{mn}x_n = b_m
		\end{cases}
	\end{align*}
	Suppose this set has more than one solutions. Then there exists two solutions $\textbf{x} = \begin{bmatrix}
	x_1 & x_2 & \cdots & x_n
	\end{bmatrix}, \textbf{y} = \begin{bmatrix}
	y_1 &y _2 & \cdots & y_n
	\end{bmatrix}$, where $\textbf{x} \neq \textbf{y}$. For any real number $k$,
	\begin{align*}
		&a_{i1}(kx_1 + (1-k)y_1) + a_{i2}(kx_2 + (1-k)y_2) + \cdots + a_{in}(kx_n + (1-k)y_n) \\
		&= k(a_{i1}x_1 + a_{i2}x_2 + \cdots + a_{in}x_n) + (1-k)(a_{i1}y_1 + a_{i2}y_2 + \cdots + a_{in}y_n) \\
		&= kb_i + (1-k)b_i = b_i
	\end{align*}
	where $1 \le i \le m$.
	Thus, a vector
	\begin{align*}
		\textbf{v}_k = \begin{bmatrix}
		kx_1 + (1-k)y_1 & kx_2 + (1-k)y_2 & \cdots & kx_n + (1-k)y_n
		\end{bmatrix}
	\end{align*} is a also solution of the set of linear equations.
	Therefore, if the set of linear equations has more than one solutions, then there are infinitely many solutions.
\end{proof}

\noindent \textit{Definition.} The \textbf{elementary row operations}
\begin{enumerate}
	\item Interchanging two rows ($R_i \leftrightarrow R_j$)
	\item Multiplying nonzero constant to a row ($kR_i$)
	\item Adding a multiple of a row to another row ($R_i + kR_j$)
\end{enumerate}
can be performed to a matrix, and the matrices before and after applying the elementary row operations are equivalent.
Applying certain sequence of elementary row operations to bring a matrix into REF is called \textbf{row reduction}.

\noindent \\ \textit{Definition.} Matrices $A$ and $B$ are \textbf{row equivalent} if there exists a sequence of elementary row operation that converts $A$ to $B$.

\begin{theorem}
	Matrices $A$ and $B$ are row equivalent if and only if they can be reduced to same REF.
\end{theorem}
\begin{proof}
	Suppose that matrices $A$ and $B$ can be reduced to REF $R$. Then there exists a sequence of elementary row operations which converts $B$ to $R$. Reversing the sequence will convert $R$ to $B$. Then combining the sequence of operations $A \rightarrow R$ and $R \rightarrow B$ will convert $A$ to $B$. Therefore, $A$ and $B$ are row equivalent.
\end{proof}

\noindent \textit{Definition.} The \textbf{rank} of a matrix is the number of nonzero rows in REF of a matrix. \\

\noindent The variables in the linear system corresponding to leading entires in REF of coefficient matrix is called \textbf{leading variables}. The rest are called \textbf{free variables}. \\

\noindent \textit{Note.} A system of linear equations with at least one free variable has infinitely many solutions.

\begin{theorem}[The Rank Theorem]
	Let $A$ be the coefficient matrix of a system of linear equations with $n$ variables. If the system is consistent, then \begin{align*}
		\textnormal{number of free variables} = n - \textnormal{rank}(A)
	\end{align*}
\end{theorem}

\begin{proof}
	The number of leading variables in the system is equal to the number of nonzero rows of the coefficient matrix $A$, which is rank$(A)$. Therefore, the number of free variables is $n-$rank$(A)$.
\end{proof}

\begin{theorem}
	A homogeneous system of linear equations $[A|\textbf{0}]$ with $n$ variables and $m$ equations has infinitely many solutions if $m<n$.
\end{theorem}

\begin{proof}
	Since there exists a trivial solution $\textbf{0}$, the system is consistent. By the Rank Theorem (Thm 2.2),
	\begin{align*}
		\textnormal{number of free variables} = n - \textnormal{rank}(A) \ge n - m > 0
	\end{align*} Therefore there exists at least one free variable, hence the system has infinitely many solutions.
\end{proof}

\section{Solving Linear Systems : Example}

\noindent The \textbf{Gaussian Elimination} is an algorithm which solves a linear system. The procedure is
\begin{enumerate}
	\item Reduce the augmented matrix of the system into REF.
	\item Using back substitution, solve the REF of the matrix.
\end{enumerate}

\noindent The \textbf{Gauss-Jordan Elimination} also solves a linear system, but in easier way. The procedure is
\begin{enumerate}
	\item Reduce the augmented matrix of the system into RREF.
	\item Solve the leading variables in terms of free variables.
\end{enumerate}

\noindent Solving a given system of linear equations \begin{align*}
	\begin{cases}
	   	 x_1 -  x_2 - x_3 + 2x_4 = 1 \\
		2x_1 - 2x_2 - x_3 + 3x_4 = 3 \\
		-x_1 +  x_2 - x_3        = -3
	\end{cases}
\end{align*} using Gaussian Elimination starts from representing the system into augmented matrix \begin{align*}
	\begin{bmatrix}[cccc|c]
		 1 & -1 & -1 & 2 &  1 \\
		 2 & -2 & -1 & 3 &  3 \\
		-1 &  1 & -1 & 0 & -3
	\end{bmatrix}
\end{align*}
Then reduce the augmented matrix using elementary row operations.
\begin{align*}
	\begin{bmatrix}[cccc|c]
		1 & -1 & -1 & 2 &  1 \\
		2 & -2 & -1 & 3 &  3 \\
		-1 &  1 & -1 & 0 & -3
	\end{bmatrix} \xrightarrow{R_2 - 2R_1, R_3 + R_1}
	&\begin{bmatrix}[cccc|c]
		1 & -1 & -1 &  2 &  1 \\
		0 &  0 &  1 & -1 &  1 \\
		0 &  0 & -2 &  2 & -2
	\end{bmatrix} \\
	\xrightarrow{R_3 + 2R_2}
	&\begin{bmatrix}[cccc|c]
		1 & -1 & -1 &  2 &  1 \\
		0 &  0 &  1 & -1 &  1 \\
		0 &  0 &  0 &  0 &  0
	\end{bmatrix}
\end{align*}
The system is consistent because constants of zero-rows are all 0. (If not, the system will be inconsistent) The leading entries are in first and third column, so leading variables are $x_1$ and $x_3$, and free variables are $x_2$ and $x_4$. Using back substitution, solve $x_1$ and $x_3$ in terms of $x_2$ and $x_4$.
\begin{align*}
	x_3 - x_4 = 1 &\rightarrow x_3 = x_4 + 1 \\
	x_1 - x_2 - x_3 + 2x_4 = 1 &\rightarrow x_1 = x_2 - x_4 + 2
\end{align*}
You can obtain same result with reducing REF into RREF.
\begin{align*}
	\begin{bmatrix}[cccc|c]
	1 & -1 & -1 &  2 &  1 \\
	0 &  0 &  1 & -1 &  1 \\
	0 &  0 &  0 &  0 &  0
	\end{bmatrix} \xrightarrow{R_1 + R_2}
	\begin{bmatrix}[cccc|c]
	1 & -1 & 0 & 1 & 2 \\
	0 & 0 & 1 & -1 & 1 \\
	0 & 0 & 0 & 0 & 0
	\end{bmatrix}
\end{align*}
Representing the variables into form of vector will give you the solution set. ($x_2$ corresponds to $s$ and $x_4$ to $t$ in the solution.)
\begin{align*}
	\begin{bmatrix}
		x_1 \\ x_2 \\ x_3 \\ x_4
	\end{bmatrix} =
	\begin{bmatrix}
		2 \\ 0 \\ 1 \\ 0
	\end{bmatrix} + s
	\begin{bmatrix}
		1 \\ 1 \\ 0 \\ 0
	\end{bmatrix} + t
	\begin{bmatrix}
		-1 \\ 0 \\ 1 \\ 1
	\end{bmatrix} (s,t \in \mathbb{R})
\end{align*}

\section{Spanning Sets and Linear Independence}

\begin{theorem}
	A system of linear equations with augmented matrix $[A|\textbf{b}]$ is consistent if and only if $\textbf{b}$ is a linear combination of the columns of $A$.
\end{theorem}

\begin{proof}
	Suppose a system of linear equations with augmented matrix
	\begin{align*}
		[A|\textbf{b}]=
		\begin{bmatrix} [cccc|c]
			a_{11} & a_{12} & \cdots & a_{1n} & b_1 \\
			a_{21} & a_{22} & \cdots & a_{2n} & b_2 \\
			\vdots & \vdots & & \vdots & \vdots \\
			a_{m1} & a_{m2} & \cdots & a_{mn} & b_m
		\end{bmatrix}
	\end{align*} is consistent. Then there exist real numbers $c_1, c_2, \cdots, c_n$ which satisfy
	\begin{align*}
		a_{i1}c_1 + a_{i2}c_2 + \cdots + a_{in}c_n = b_i, \textnormal{for all } 1 \le i \le m
	\end{align*}
	Therefore, $b$ can be represented with linear combination of columns of $A$.
	\begin{align*}
		\textbf{b} = \begin{bmatrix}
			b_1 \\ b_2 \\ \vdots \\ b_m
		\end{bmatrix} = c_1 \begin{bmatrix}
			a_{11} \\ a_{21} \\ \vdots \\ a_{m1}
		\end{bmatrix} + c_2 \begin{bmatrix}
			a_{12} \\ a_{22} \\ \vdots \\ a_{m2}
		\end{bmatrix} + \cdots + c_n \begin{bmatrix}
			a_{1n} \\ a_{2n} \\ \vdots \\ a_{mn}
		\end{bmatrix}
	\end{align*}
	Similarly, if $b$ is a linear combination of the columns of $A$, the coefficients of linear combination satisfy the linear equations of system. Therefore, the system is consistent.
\end{proof}

\noindent \textit{Definition.} Let $S = \{\textbf{v}_1, \textbf{v}_2, \cdots, \textbf{v}_n\}$ be a set of vectors in \Rn. Then the set of all linear combinations of $\textbf{v}_1, \textbf{v}_2, \cdots, \textbf{v}_n$ is called the \textbf{span} of $\textbf{v}_1, \textbf{v}_2, \cdots, \textbf{v}_n$ and denoted by span($\textbf{v}_1, \textbf{v}_2, \cdots, \textbf{v}_n$) or span($S$). If span($S$) = \Rn, then $S$ is a \textbf{spanning set} for \Rn.

\noindent \\ \textit{Note.} To prove that the spanning set span($S$) is equal to another set $T$, proving \textbf{both} $S \subset T$ and $T \subset S$ is essencial.

\noindent \\ \textit{Definition.} A set of vectors $\textbf{v}_1, \textbf{v}_2, \cdots \textbf{v}_n$ is \textbf{linearly dependent} if there exists scalars $c_1, c_2, \cdots c_n$, at least one of which is nonzero, such that
\begin{align*}
	c_1\textbf{v}_1 + c_2\textbf{v}_2 + \cdots + c_n\textbf{v}_n = \textbf{0}
\end{align*}
A set of vectors which are not linearly dependent is \textbf{linearly independent}.

\begin{theorem}
	Vectors $\textbf{v}_1, \textbf{v}_2, \cdots, \textbf{v}_n$ are linearly dependent if and only if at least one of the vectors can be represented with a linear combination of other vectors.
\end{theorem}

\begin{proof}
	If vectors $\textbf{v}_1, \textbf{v}_2, \cdots, \textbf{v}_n$ are linearly dependent, there exist scalars $c_1, c_2, \cdots, c_n$ such that $c_1\textbf{v}_1 + c_2\textbf{v}_2 + \cdots + c_n\textbf{v}_n = \textbf{0}$ and at least one of scalars is nonzero. Suppose $c_1 \neq 0$. Then
	\begin{align*}
		\textbf{v}_1 = -\frac{c_2}{c_1}\textbf{v}_2 - \cdots - \frac{c_n}{c_1}\textbf{v}_n
	\end{align*}
	Thus $\textbf{v}_1$ is represented with a linear combination of other vectors.
	\\
	Suppose that $\textbf{v}_1$ can be expressed as linear combination of $\textbf{v}_2, \textbf{v}_3, \cdots, \textbf{v}_n$. Then there exist scalars $c_2, c_3, \cdots, c_n$ which satisfy $\textbf{v}_1 = c_2\textbf{v}_2 + c_3\textbf{v}_3 + \cdots + c_n\textbf{v}_n$. Then 
	\begin{align*}
		\textbf{v}_1 - c_2\textbf{v}_2 - \cdots - c_n\textbf{v}_n = \textbf{0}
	\end{align*} thus vectors $\textbf{v}_1, \textbf{v}_2, \cdots, \textbf{v}_n$ are linearly dependent.
\end{proof}

\begin{theorem}
	Let $\textbf{v}_1, \textbf{v}_2, \cdots, \textbf{v}_n$ be vectors in \Rn and let $A$ be the $n \times m$ matrix $\begin{bmatrix}
		\textbf{v}_1 & \textbf{v}_2 & \cdots & \textbf{v}_n
	\end{bmatrix}$. Then $\textbf{v}_1, \textbf{v}_2, \cdots, \textbf{v}_n$ are linearly dependent if and only if the homogeneous linear system with augmented matrix $[A|\textbf{0}]$ has a nontrivial solution (or, infinitely many solutions.)
\end{theorem}

\begin{proof}
	Vectors $\textbf{v}_1, \textbf{v}_2, \cdots, \textbf{v}_n$ are linearly dependent if and only if there exist scalars $c_1, c_2, \cdots, c_n$ such that $c_1\textbf{v}_1 + c_2\textbf{v}_2 + \cdots + c_n\textbf{v}_n = \textbf{0}$ where at least one of scalars is nonzero. Since the nonzero vector $\begin{bmatrix}
		c_1 \\ c_2 \\ \vdots \\ c_n
	\end{bmatrix}$ is a solution of the linear system with augmented matrix $[A|\textbf{0}]$, the system has nontrivial solution.
	Similarly, if there exists a nontrivial solution $\begin{bmatrix}
		c_1 \\ c_2 \\ \cdots \\ c_n
	\end{bmatrix}$ of the homogeneous linear system, the components satisfy
	\begin{align*}
		c_1\textbf{v}_1 + c_2\textbf{v}_2 + \cdots + c_n\textbf{v}_n = \textbf{0}
	\end{align*} Since at least one of $c_1, c_2, \cdots, c_n$ is nonzero, vectors $\textbf{v}_1, \textbf{v}_2, \cdots, \textbf{v}_n$ are linearly dependent.
\end{proof}

\begin{plaintheorem}[Elementary Row Operations and Linear Combination]
	(Example 2.25) If matrix $B$ is generated by applying elementary row operations at matrix $A$, then rows of $B$ can be represented as nontrivial linear combination of rows of $A$.
\end{plaintheorem}

\begin{proof}
	Let $A = \begin{bmatrix}
		\textbf{R}_1 \\ \textbf{R}_2 \\ \vdots \\ \textbf{R}_n
	\end{bmatrix}$. By Theorem 3.10, sequence of elementary row operations that would convert $A$ into $B$ will also convert $I_n$ to $E$, where $B = EA$. Since there exists the sequence of elementary row operations converting $I_n$ to $E$, $I_n$ and $E$ are row equivalent, so $\textnormal{rank}(I_n) = \textnormal{rank}(E) = n$. Then the rows of $E$ are all nonzero rows. Therefore, rows of $B = EA$ can be expressed as nontrivial linear combination of rows of $A$.
\end{proof}

\begin{theorem}
	Let $\textbf{v}_1, \textbf{v}_2, \cdots, \textbf{v}_n$ be row vectors in \Rn and let $A$ be the $m\times n$ matrix $\begin{bmatrix}
		\textbf{v}_1 \\ \textbf{v}_2 \\ \vdots \\ \textbf{v}_n
	\end{bmatrix}$ Then $\textbf{v}_1, \textbf{v}_2, \cdots, \textbf{v}_n$ are linearly dependent if and only if rank$(A) < n$.
\end{theorem}

\begin{proof}
	Assume that $\textbf{v}_1, \textbf{v}_2, \cdots, \textbf{v}_n$ are linearly dependent. By Theorem 2.5, at least one of the vectors can be written as a linear combination of other vectors. With relabeling such vector to $\textbf{v}_n$, there exist scalars $c_1, c_2, \cdots, c_{n-1}$ such that
	\begin{align*}
		\textbf{v}_n = c_1\textbf{v}_1 + c_2\textbf{v}_2 + \cdots + c_{n-1}\textbf{v}_{n-1}
	\end{align*}
	Then performing the elementary row operations $(R_n - c_1R_1), (R_n - c_2R_2), \cdots, (R_n - c_{n-1}R_{n-1})$ will create zero row in $n$th row. Therefore, rank$(A) < n$.
	
	\noindent Conversely, if rank$(A) < n$, $A$ would have zero row in its REF, thus there exists certain sequence of elementary operation that would create zero row from $A$. There exist scalars $c_1, c_2, \cdots, c_n$ such that at least one of scalars is nonzero, and
	\begin{align*}
		c_1\textbf{v}_1 + c_2\textbf{v}_2 + \cdots + c_n\textbf{v}_n = \textbf{0}
	\end{align*} Therefore, $\textbf{v}_1, \textbf{v}_2, \cdots, \textbf{v}_n$ are linearly dependent.
\end{proof}

\begin{theorem}
	Any set of $m$ vectors in \Rn is linearly dependent if $m > n$.
\end{theorem}

\begin{proof}
	(also solution for Exercise 2.3 45) Let $A$ be the $m \times n$ matrix with the vectors as its rows. Since rank$(A) < n < m$, by Theorem 2.7, the vectors are linearly dependent.
\end{proof}

\section{Solutions of Exercises Worthy to Solve}
\begin{enumerate}
	\item \textbf{Exercise 2.2 39}
	\begin{proof}
		The augmented matrix of the system is $\begin{bmatrix}[cc|c]
			a & b & r \\ c & d & s
		\end{bmatrix}$. Applying elementary row operations $(cR_1), (aR_2), (R_2 - R_1)$, the REF of the augmented matrix is $\begin{bmatrix}[cc|c]
			ac & bc & rc \\ 0 & ad - bc & sa - rc
		\end{bmatrix}$. If $ad - bc \neq 0$, the rank of augmented matrix is 2, therefore the system has unique solution.
	\end{proof}
	\item \textbf{Exercise 2.2 51}
	\begin{proof}
		Let $\textbf{u} = \begin{bmatrix}
			u_1 \\ u_2 \\ u_3
		\end{bmatrix}$, $\textbf{v} = \begin{bmatrix}
			v_1 \\ v_2 \\ v_3
		\end{bmatrix}$, and $\textbf{x} = \begin{bmatrix}
			x_1 \\ x_2 \\ x_3
		\end{bmatrix}$. Since $\textbf{u}\cdot\textbf{x} = \textbf{v}\cdot\textbf{x} = 0$, $\textbf{x}$ is a solution of the linear system with augmented matrix $\begin{bmatrix} [ccc|c]
			u_1 & u_2 & u_3 & 0 \\ v_1 & v_2 & v_3 & 0
		\end{bmatrix}$. The REF of augmented matrix is $\begin{bmatrix} [ccc|c]
			u_1 & u_2 & u_3 & 0 \\ 0 & u_1v_2-u_2v_1 & u_1v_3-u_3v_1 & 0
		\end{bmatrix}$, and the solution is
		\begin{align*}
			\textbf{x} = \begin{bmatrix}
				0 \\ 0 \\ 0
			\end{bmatrix} + t\begin{bmatrix}
				u_2v_3 - u_3v_2 \\ u_3v_1 - u_1v_3 \\ u_1v_2 - u_2v_1
			\end{bmatrix}
		\end{align*} where $t \in \mathbb{R}$.
	\end{proof}
	\item \textbf{Exercise 2.2 59}
	\begin{proof}
		By Theorem 2.2, there are $n-\textnormal{rank}(A)$ free variables and $\textnormal{rank}(A)$ leading variables in the linear system. Since the values of leading variables are fixed for given values of free variables, the number of solution is $p^{n-\textnormal{rank}(A)}$.
	\end{proof}
	\item \textbf{Exercise 2.2 60}
	
	\noindent \textit{Solution.} Complications arise when we try to solve this directly in $\mathbb{Z}_6$. Therefore, we split the system to $\mathbb{Z}_2$ and $\mathbb{Z}_3$, then find the solutions simultaneously satisfying them.
	
	\noindent In $\mathbb{Z}_2$,
	\begin{align*}
		\begin{bmatrix} [cc|c]
			0 & 1 & 0 \\ 0 & 1 & 0
		\end{bmatrix} \xrightarrow{R_2 - R_1} \begin{bmatrix} [cc|c]
			0 & 1 & 0 \\ 0 & 0 & 0
		\end{bmatrix}
	\end{align*} then the solution is
	\begin{align*}
		\textbf{x} = \begin{bmatrix}
			0 \\ 0
		\end{bmatrix} + t\begin{bmatrix}
			1 \\ 0
		\end{bmatrix} (t = 0, 1)
	\end{align*}
	\noindent In $\mathbb{Z}_3$,
	\begin{align*}
		\begin{bmatrix} [cc|c]
		2 & 0 & 1 \\ 1 & 0 & 2
		\end{bmatrix} \xrightarrow{R_2 + R_1} \begin{bmatrix} [cc|c]
		2 & 0 & 1 \\ 0 & 0 & 0
		\end{bmatrix}
	\end{align*} then the solution is
	\begin{align*}
		\textbf{x} = \begin{bmatrix}
			2 \\ 0
		\end{bmatrix} + t\begin{bmatrix}
			0 \\ 1
		\end{bmatrix} (t = 0, 1, 2)
	\end{align*}
	Combining these solutions, we have
	\begin{align*}
		\textbf{x} = \begin{bmatrix}
			2 \\ 0
		\end{bmatrix} + t\begin{bmatrix}
			3 \\ 2
		\end{bmatrix} (t \in \mathbb{Z}_6)
	\end{align*}
	\item \textbf{Exercise 2.3 20}
	\begin{proof}
		\textbf{(a)} Suppose $\textbf{v} \in \textnormal{span}(S)$. Then $\textbf{v}$ can be expressed as linear combination of vectors in $S$, so there exist scalars $c_1, c_2, \cdots, c_k$ such that $\textbf{v} = c_1\textbf{u}_1 + c_2\textbf{u}_2 + \cdots + c_k\textbf{u}_k$. Then $\textbf{v} = c_1\textbf{u}_1 + c_2\textbf{u}_2 + \cdots + c_k\textbf{u}_k + 0\textbf{u}_{k+1} + \cdots + 0\textbf{u}_m$, so $\textbf{v}$ is also a linear combination of vectors in $T$. Therefore, $\textnormal{span}(S) \subset \textnormal{span}(T)$.
		
		\noindent \textbf{(b)} By (a), $\textnormal{span}(S) = \mathbb{R}^n \subset \textnormal{span}(T)$. Also, $\textnormal{span}(T) \subset \mathbb{R}^n$ since span$(T)$ is a set of vectors in \Rn. Therefore, \Rn = span$(T)$.
	\end{proof}
	\item \textbf{Exercise 2.3 21}
	\begin{proof}
		\textbf{(a)} Since each $\textbf{u}_i$ are linear combinations of vectors $\textbf{v}_1, \textbf{v}_2, \cdots, \textbf{v}_m$, there exist scalars $c_{i1}, c_{i2}, \cdots, c_{im}$ for each $i$ such that $\textbf{u}_i = c_{i1}\textbf{v}_1 + c_{i2}\textbf{v}_2 + \cdots + c_{im}\textbf{v}_m$. Suppose vector $\textbf{w}$ is in span$(\textbf{u}_1, \textbf{u}_2, \cdots, \textbf{u}_k)$. Since $\textbf{w}$ is a linear combination of $\textbf{u}_1, \textbf{u}_2, \cdots, \textbf{u}_k$, there exist scalars $d_1, d_2, \cdots, d_k$ such that $\textbf{w} = d_1\textbf{u}_1 + d_2\textbf{u}_2 + \cdots + d_k\textbf{u}_k$. Then 
		\begin{align*}
			\textbf{w} &= d_1\textbf{u}_1 + d_2\textbf{u}_2 + \cdots + d_k\textbf{u}_k \\
			&= d_1(c_{11}\textbf{v}_1 + \cdots + c_{1m}\textbf{v}_m) + \cdots + d_k(c_{k1}\textbf{v}_1 + \cdots + c_{km}\textbf{v}_m) \\
			&= (c_{11}d_1 + c_{21}d_2 + \cdots + c_{k1}d_k)\textbf{v}_1 + \cdots + (c_{1m}d_1 + c_{2m}d_2 + \cdots + c_{km}d_k)\textbf{v}_m
		\end{align*} Therefore, $\textbf{w}$ can be expressed as a linear combination of vectors $\textbf{v}_1, \textbf{v}_2, \cdots, \textbf{v}_m$, so $\textbf{w} \in$ span($\textbf{v}_1, \textbf{v}_2, \cdots, \textbf{v}_m$).
		
		\noindent \textbf{(b)} The same logic can be applied to prove that span($\textbf{v}_1, \textbf{v}_2, \cdots, \textbf{v}_m$) $\subset$ span($\textbf{u}_1, \textbf{u}_2, \cdots, \textbf{u}_k$). Therefore, span($\textbf{u}_1, \textbf{u}_2, \cdots, \textbf{u}_k$) = span($\textbf{v}_1, \textbf{v}_2, \cdots, \textbf{v}_m$).
		
		\noindent \textbf{(c)} Let $\textbf{v}_1 = \begin{bmatrix}
			1 \\ 0 \\ 0
		\end{bmatrix}, \textbf{v}_2 = \begin{bmatrix}
			1 \\ 1 \\ 0
		\end{bmatrix}, \textbf{v}_3 = \begin{bmatrix}
			1 \\ 1 \\ 1
		\end{bmatrix}$, then $\textbf{v}_1 = \textbf{e}_1$, $\textbf{v}_2 = \textbf{e}_1 + \textbf{e}_2$, and $\textbf{v}_3 = \textbf{e}_1 + \textbf{e}_2 + \textbf{e}_3$. Also, $\textbf{e}_1 = \textbf{v}_1$, $\textbf{e}_2 = \textbf{v}_2 - \textbf{v}_1$, and $\textbf{e}_3 = \textbf{v}_3 - \textbf{v}_2$. Since span($\textbf{e}_1, \textbf{e}_2, \textbf{e}_3$) = $\mathbb{R}^3$, by (b),  span($\textbf{v}_1 , \textbf{v}_2, \textbf{v}_3$) = span($\textbf{e}_1, \textbf{e}_2, \textbf{e}_3$) = $\mathbb{R}^3$.
	\end{proof}
	\item \textbf{Exercise 2.3 44}
	\begin{proof}
		Suppose that vectors $\textbf{v}_1, \textbf{v}_2$ are linearly dependent. Then there exist scalars $c_1, c_2$ such that $c_1\textbf{v}_1 + c_2\textbf{v}_2 = \textbf{0}$.
		
		\noindent (i) $c_1c_2 \neq 0$
		\begin{align*}
			\textbf{v}_1 = -\frac{c_2}{c_1}\textbf{v}_2
		\end{align*}
		\noindent (ii) If not, without loss of generality, $c_1 \neq 0$ and $c_2 = 0$.
		\begin{align*}
			c_1\textbf{v}_1 + c_2\textbf{v}_2 &= \textbf{0} \\
			c_1\textbf{v}_1 &= \textbf{0} \\
			\therefore \textbf{v}_1 &= \textbf{0} = 0\textbf{v}_2
		\end{align*}
	\end{proof}
	\item \textbf{Exercise 2.3 46}
	\begin{proof}
		Given a set of vectors \vn, suppose that vectors $\textbf{v}_{a_1}, \textbf{v}_{a_2}, \cdots, \textbf{v}_{a_m}$ are linearly dependent, where $\{a_1, a_2, \cdots, a_m\} \subset \{1, 2, \cdots n\}$. Then there exists scalars $d_1, d_2, \cdots, d_m$ such that at least one of scalars is nonzero and $d_1\textbf{v}_{a_1} + d_2\textbf{v}_{a_2} + \cdots + d_m\textbf{v}_{a_m} = \textbf{0}$.
		
		\noindent Let 
		\begin{align*}
		c_i = \begin{cases}
			d_j &\mbox{if  } i = a_j \\
			0 &\mbox{if  } i \notin \{a_1, a_2, \cdots, a_m \}
		\end{cases}
		\end{align*} then $c_1\textbf{v}_1 + c_2\textbf{v}_2 + \cdots + c_n\textbf{v}_n = d_1\textbf{v}_{a_1} + d_2\textbf{v}_{a_2} + \cdots + d_m\textbf{v}_{a_m} = \textbf{0}$. Therefore, the set $\textbf{v}_1, \textbf{v}_2, \cdots, \textbf{v}_n$ is linearly dependent. The contrapositive proposition that the subset of linearly independent set of vectors is always linearly independent is also true.
	\end{proof}
	\item \textbf{Exercise 2.3 47}
	\begin{proof}
		We prove this proposition by Exercise 2.3 21.
		
		\noindent (i) For every $\textbf{v}_i \in S'$, $\textbf{v}_i = 0\textbf{v}_1 + \cdots + 1\textbf{v}_i + \cdots + 0\textbf{v}_k + 0\textbf{v}$. Thus, every vectors in $S'$ can be expressed as a linear combination of vectors in $S$.
		
		\noindent (ii) For every $\textbf{v}_i \in S$, $\textbf{v}_i = 0\textbf{v}_1 + \cdots + 1\textbf{v}_i + \cdots + 0\textbf{v}_k$, and $\textbf{v}$ is a linear combination of $\textbf{v}_1, \textbf{v}_2, \cdots, \textbf{v}_k$. Thus, every vectors in $S$ can be expressed as a linear combination of vectors in $S'$.
		
		\noindent By Exercise 2.3 21(b), span($\textbf{v}_1, \textbf{v}_2, \cdots, \textbf{v}_k, \textbf{v}$) = span($\textbf{v}_1, \textbf{v}_2, \cdots, \textbf{v}_k$).
	\end{proof}
	\item \textbf{Exercise 2.3 48}
	\begin{proof}
		Suppose that $\textbf{v} = c_1\textbf{v}_1 + c_2\textbf{v}_2 + \cdots + c_k\textbf{v}_k (c_1 \neq 0)$ and vectors $\textbf{v}, \textbf{v}_2, \cdots, \textbf{v}_k$ are linearly dependent. Then 
	\end{proof}
\end{enumerate}
\chapter{Matrices}
\section{Terminology}

\textit{Definition.} A \textbf{matrix} is a rectangular array of numbers, which are called as \textbf{entries} or \textbf{elements}. If the matrix has $n$ rows and $m$ columns, the \textbf{size} of the matrix is $n \times m$.

\noindent \\ A $1 \times n$ matrix is called a \textbf{row matrix}, or \textbf{row vector}. A $n \times 1$ matrix is called a \textbf{column matrix}, or \textbf{column vector}. (\textit{A vector is considered as a matrix}) We can denote matrices using row vectors or column vectors, such as 
\begin{align*}
A = \begin{bmatrix}
\textbf{A}^{C}_1 & \textbf{A}^{C}_2 & \cdots & \textbf{A}^{C}_m
\end{bmatrix} = \begin{bmatrix}
\textbf{A}^{R}_1 \\ \textbf{A}^{R}_2 \\ \vdots \\ \textbf{A}^{R}_n
\end{bmatrix}
\end{align*} where $\textbf{A}^{C}_i$ is the $i$th column of $A$ and $\textbf{A}^{R}_i$ is the $i$th row of $A$.

\noindent \\ The element at $i$th row and $j$th column is denoted by $A_{ij}$. We can also denote matrices using elements, such as $A = [A_{ij}]$.

\noindent \\ \textit{Definition.} The \textbf{diagonal entries}of $A$ are $A_{ii}$.

\noindent \\ \textit{Definition.} The \textbf{square matrix} is a matrix which has same number of rows and columns (so the size is $n \times n$). \textbf{Diagonal matrix} is a square matrix which has its nondiagonal entries as 0. A diagonal matrix with all of its diagonal entries are the same are \textbf{scalar matrix}. If the value of diagonal entries are all 1, it is \textbf{identity matrix}.

\noindent A $n \times n$ identity matrix is denoted as $I_n$, and
\begin{align*}
I_n = \begin{bmatrix}
1 & 0 & \cdots & 0 \\
0 & 1 & \cdots & 0 \\
\vdots & \vdots & & \vdots \\
0 & 0 & \cdots & 1
\end{bmatrix}
\end{align*}

\noindent A \textbf{zero matrix} $O$ is a matrix which all of its entires are zero.

\noindent \\ \textit{Definition.} Two matrices are \textbf{equal} if and only if

(i) The size of two matrices are the same.

(ii) The corresponding entries of the matrices are the same.

\section{Matrix Operations}

\textit{Definition.} If $A$ and $B$ are both $n \times m$ matrices, the \textbf{sum} of $A$ and $B$ is defined as
\begin{align*}
A + B = [A_{ij} + B_{ij}]
\end{align*} where $A+B$ is also an $n \times m$ matrix.

\noindent \\ \textit{Definition.} If $A$ is an $n \times m$ matrix and $c$ is a scalar, the \textbf{scalar multiplication} $cA$ is defined as
\begin{align*}
cA = [cA_{ij}]
\end{align*} where $cA$ is also an $n \times m$ matrix.

\noindent \\ \textit{Definition.} If $A$ is an $m \times n$ matrix and $B$ is an $n \times r$ matrix, then the \textbf{product} $AB$ is defined as
\begin{align*}
AB = [A_{i1}B_{1j} + A_{i2}B_{2j} + \cdots + A_{in}B_{nj}]
\end{align*} where $AB$ is an $m \times r$ matrix.

\noindent \\ \textit{* Note.} $(AB)_{ij} = \textbf{A}^{R}_i \cdot \textbf{B}^{C}_j$

\noindent \\ \textit{Defintion.} A \textbf{transpose} of an $n \times m$ matrix $A$ is denoted as $A^{T}$, which is an $m \times n$ matrix and
\begin{align*}
(A^{T})_{ij} = A_{ji}
\end{align*}

\noindent \\ \textit{Definition.} A square matrix $A$ is \textbf{symmetric} if $A^{T} = A$. \\


\noindent \\ Matrices can be divided into \textbf{submatrices} by partitioning the matrix into ceratin blocks. We introduce partitioned matrix in order to perform matrix multiplication in easier way.

\begin{plaintheorem}[Multiplication of Partitioned Matrices]
	If matrices $A$ and $B$ are partitioned as $A = \begin{bmatrix}
	A_{11} & \cdots & A_{1m} \\
	\vdots &        & \vdots \\
	A_{n1} & \cdots & A_{nm}
	\end{bmatrix}$ and $B = \begin{bmatrix}
	B_{11} & \cdots & B_{1r} \\
	\vdots &        & \vdots \\
	B_{m1} & \cdots & B_{mr}
	\end{bmatrix}$, then $AB$ can be partitioned as
	\begin{align*}
	AB = \begin{bmatrix}
	A_{11}B_{11} + A_{12}B_{21} + \cdots + A_{1m}B_{m1}
	& \cdots & A_{11}B_{1r} + A_{12}B_{2r} + \cdots + A_{1m}B_{mr} \\
	\vdots & & \vdots \\
	A_{n1}B_{11} + A_{n2}B_{21} + \cdots + A_{nm}B_{m1}
	& \cdots & A_{n1}B_{1r} + A_{n2}B_{2r} + \cdots + A_{nm}B_{mr}
	\end{bmatrix}
	\end{align*} assuming that all the products are defined.
\end{plaintheorem}
\begin{proof}
	Let $A$ be an $n \times m$ matrix and let $B$ be an $m \times r$ matrix so that $AB$ is defined. Then
	\begin{align*}
	AB = A\begin{bmatrix}
	\textbf{B}^{C}_1 & \textbf{B}^{C}_2 & \cdots & \textbf{B}^{C}_r
	\end{bmatrix} = \begin{bmatrix}
	A\textbf{B}^{C}_1 & A\textbf{B}^{C}_2 & \cdots & A\textbf{B}^{C}_r
	\end{bmatrix} &\textnormal{ (matrix-column representation)} \\
	= \begin{bmatrix}
	\textbf{A}^{R}_1 \\ \textbf{A}^{R}_2 \\ \vdots \\ \textbf{A}^{R}_n
	\end{bmatrix}B = \begin{bmatrix}
	\textbf{A}^{R}_1B \\ \textbf{A}^{R}_2B \\ \vdots \\ \textbf{A}^{R}_nB
	\end{bmatrix} &\textnormal{ (row-matrix representation)} \\
	= \begin{bmatrix}
	\textbf{A}^{R}_1 \\ \textbf{A}^{R}_2 \\ \vdots \\ \textbf{A}^{R}_n
	\end{bmatrix}\begin{bmatrix}
	\textbf{B}^{C}_1 & \textbf{B}^{C}_2 & \cdots & \textbf{B}^{C}_r
	\end{bmatrix} = \begin{bmatrix}
	\textbf{A}^{R}_1\textbf{B}^{C}_1 & \cdots &
	\textbf{A}^{R}_1\textbf{B}^{C}_r \\
	\vdots & & \vdots \\
	\textbf{A}^{R}_n\textbf{B}^{C}_1 & \cdots &
	\textbf{A}^{R}_n\textbf{B}^{C}_r		
	\end{bmatrix} &\textnormal{ (row-column representation)} \\
	= \begin{bmatrix}
	\textbf{A}^{C}_1 & \textbf{A}^{C}_2 & \cdots & \textbf{A}^{C}_m
	\end{bmatrix} \begin{bmatrix}
	\textbf{B}^{R}_1 \\ \textbf{B}^{R}_2 \\ \vdots \\ \textbf{B}^{R}_m
	\end{bmatrix} &\textnormal{ (column-row representation)}\\
	= \textbf{A}^{C}_1\textbf{B}^{R}_1 + \cdots + \textbf{A}^{C}_m\textbf{B}^{R}_m &\textnormal{ (outer product expansion)}
	\end{align*}
\end{proof} 

\begin{theorem}
	Let $A$ be an $n \times m$ matrix. Then
	\begin{enumerate}
		\item $\textbf{e}_iA = \textbf{A}^{R}_i$
		\item $A\textbf{e}_j = \textbf{A}^{C}_j$
	\end{enumerate}
\end{theorem}

\begin{proof}
	\textbf{(a)} (Exercise 3.1 41) Since the size of $\textbf{e}_i$ is $1 \times n$ and the size of $A$ is $n \times m$, the size of $\textbf{e}_iA$ is $1 \times m$, which is same with the size of $\textbf{A}^{R}_i$. Also,
	\begin{align*}
	(\textbf{e}_iA)_{1k} = 0A_{1k} + \cdots + 1A_{ik} + \cdots + 0A_{nk} = A_{ik}
	\end{align*} Therefore $\textbf{e}_iA = \begin{bmatrix}
	A_{i1} & A_{i2} & \cdots & A_{im}
	\end{bmatrix} = \textbf{A}^{R}_i$.
	
	\noindent \\ \textbf{(b)} Since the size of $A$ is $n \times m$ and the size of $\textbf{e}_j$ is $m \times 1$, the size of $A\textbf{e}_j$ is $n \times 1$, which is same with the size of $\textbf{A}^{C}_i$. Also,
	\begin{align*}
	(A\textbf{e}_j)_{k1} = 0A_{k1} + \cdots + 1A_{ki} + \cdots + 0A_{km} = A_{ki}
	\end{align*} Therefore $A\textbf{e}_j = \begin{bmatrix}
	A_{1i} \\ A_{2i} \\ \vdots \\ A_{ni}
	\end{bmatrix} = \textbf{A}^{C}_i$.
\end{proof}

\begin{theorem}[Algebraic Properties of Matrix Addition and Scalar Multiplication]
	Let $A$, $B$, and $C$ be matrices of the same size and let $c$ and $d$ be scalars.
	\begin{enumerate}
		\item $A+B = B+A$ (Commutativity of Matrix Addition)
		\item $(A+B)+C = A+(B+C)$ (Associativity of Matrix Addition)
		\item $A+O = A$
		\item $A+(-A) = O$
		\item $c(A+B) = cA+cB$ (Left Distributivity of Scalar Multiplication over Matrix Addition)
		\item $(c+d)A = cA+dA$ (Right Distributivity of Scalar Multiplication over Matrix Addition)
		\item $c(dA) = (cd)A$
		\item $1A = A$
	\end{enumerate}	
\end{theorem}

\begin{proof}
	The size of all matrices in both sides of equation will be equal to the size of $A$, $B$, and $C$. The proof in componentwise perspective is on below.
	\begin{enumerate}
		\item (Exercise 3.2 17)
		\begin{align*}
		(A+B)_{ij} = A_{ij} + B_{ij} = B_{ij} + A_{ij} = (B+A)_{ij}
		\end{align*}
		\item (Exercise 3.2 17)
		\begin{align*}
		((A+B)+C)_{ij} = (A+B)_{ij} + C_{ij} &= (A_{ij} + B_{ij}) + C_{ij} \\
		&= A_{ij} + (B_{ij} + C_{ij}) = A_{ij} + (B+C)_{ij} = (A+(B+C))_{ij}
		\end{align*}
		\item (Exercise 3.2 17)
		\begin{align*}
		(A+O)_{ij} = A_{ij} + O_{ij} = A_{ij} + 0 = A_{ij}
		\end{align*}
		\item (Exercise 3.2 17)
		\begin{align*}
		(A+(-A))_{ij} = A_{ij} + (-A)_{ij} = A_{ij} + (-A_{ij}) = 0
		\end{align*}
		\item (Exercise 3.2 18)
		\begin{align*}
		(c(A+B))_{ij} = c(A+B)_{ij} = c(A_{ij} + B_{ij}) = cA_{ij} + cB_{ij} = (cA)_{ij} + (cB)_{ij} = (cA + cB)_{ij}
		\end{align*}
		\item (Exercise 3.2 18)
		\begin{align*}
		((c+d)A)_{ij} = (c+d)A_{ij} = cA_{ij} + dA_{ij} = (cA)_{ij} + (dA)_{ij} = (cA + dA)_{ij}
		\end{align*}
		\item (Exercise 3.2 18)
		\begin{align*}
		(c(dA))_{ij} = c(dA)_{ij} = c(dA_{ij}) = (cd)A_{ij} = ((cd)A)_{ij}
		\end{align*}
		\item (Exercise 3.2 18)
		\begin{align*}
		(1A)_{ij} = 1A_{ij} = A_{ij}
		\end{align*}
	\end{enumerate}
\end{proof}

\begin{theorem} [Properties of Matrix Multiplication]
	Let $A$, $B$, and $C$ be matrices and let $k$ be the scalar. If the operations can be defined,
	\begin{enumerate}
		\item $A(BC) = (AB)C$
		\item $A(B+C) = AB + AC$
		\item $(A+B)C = AC + BC$
		\item $k(AB) = (kA)B = A(kB)$
		\item $I_nA = A = AI_m$, where size of $A$ is $n \times m$
	\end{enumerate}
\end{theorem}

\begin{proof} The proof of (a) is also done in page 229 of the textbook. (Not Included in Exam)
	\begin{enumerate}
		\item Let $A$ be a $n \times m$ matrix, $B$ be a $m \times p$ matrix, and $C$ be a $p \times q$ matrix. Then the size of $BC$ is $m \times q$, so the size of $A(BC)$ is $n \times q$. Also, the size of $AB$ is $n \times p$, so the size of $(AB)C$ is $n \times q$. Therefore, the size of $A(BC)$ and $(AB)C$ are the same.
		\begin{align*}
		((AB)C)_{ij} &= \sum_{k=1}^{p}((AB)_{ik}C_{kj}) \\
		&= \sum_{k=1}^{p}(\sum_{l=1}^{m}A_{il}B_{lk})C_{kj} \\
		&= \sum_{k=1}^{p}\sum_{l=1}^{m}(A_{il}B_{lk}C_{kj}) \\
		&= \sum_{l=1}^{m}A_{il}(\sum_{k=1}^{p}B_{lk}C_{kj}) \\
		&= \sum_{l=1}^{m}A_{il}(BC)_{lj} \\
		&= (A(BC))_{ij}
		\end{align*}
		
		\item Let $A$ be a $n \times m$ matrix, and let $B$ and $C$ be $m \times r$ matrices. Then the size of $B+C$ is $m \times r$, so the size of $A(B+C)$ is $n \times r$. The size of $AB$ and $AC$ are also $n \times r$, so the size of matrices at both sides of equations are the same.
		\begin{align*}
		(A(B+C))_{ij} &= \textbf{A}^{R}_i \cdot ((B+C)^{C}_j) \\
		&= \textbf{A}^{R}_i \cdot (\textbf{B}^{C}_j + \textbf{C}^{C}_j) \\
		&= \textbf{A}^{R}_i \cdot \textbf{B}^{C}_j + \textbf{A}^{R}_i \cdot \textbf{C}^{C}_j \\
		&= (AB)_{ij} + (AC)_{ij} = (AB+AC)_{ij}
		\end{align*}
		
		\item (Exercise 3.2 19) Let $A$ and $B$ be $n \times m$ matrices, and let $C$ be a $m \times r$ matrix. Then the size of $A+B$ is $n \times m$, so the size of $(A+B)C$ is $n \times r$. The size of $AC$ and $BC$ are also $n \times r$, so the size of matrices at both sides of equations are the same.
		\begin{align*}
		((A+B)C)_{ij} &= ((A+B)^{R}_i) \cdot \textbf{C}^{C}_j \\
		&= (\textbf{A}^{R}_i + \textbf{B}^{R}_i) \cdot \textbf{C}^{C})_j \\
		&= \textbf{A}^{R}_i \cdot \textbf{C}^{C}_j + \textbf{B}^{R}_i \cdot \textbf{C}^{C}_j \\
		&= (AC)_{ij} + (BC)_{ij} = (AC+BC)_{ij} 
		\end{align*}
		
		\item (Exercise 3.2 20) Let $A$ be a $n \times m$ matrix and $B$ be a $m \times r$ matrix. Then the size of $k(AB)$, $(kA)B$, and $A(kB)$ are all $n \times r$, so all the size of the matrices are the same.
		\begin{align*}
		(k(AB))_{ij} &= k(\textbf{A}^{R}_i \cdot \textbf{B}^{C}_j) \\
		&= (k\textbf{A}^{R}_i) \cdot \textbf{B}^{C}_j = ((kA)B)_{ij} \\
		&= \textbf{A}^{R}_i \cdot (k\textbf{B}^{C}_j) = (A(kB))_{ij}
		\end{align*}
		
		\item (Exercise 3.2 21) Let $A$ be a $n \times m$ matrix. Since $I_n = \begin{bmatrix}
		\textbf{e}_1 \\ \vdots \\ \textbf{e}_n
		\end{bmatrix}$ and $I_m = \begin{bmatrix}
		\textbf{e}_1 & \cdots & \textbf{e}_m
		\end{bmatrix}$,
		\begin{align*}
		I_nA = \begin{bmatrix}
		\textbf{e}_1 \\ \vdots \\ \textbf{e}_n
		\end{bmatrix}A = \begin{bmatrix}
		\textbf{e}_1A \\ \vdots \\ \textbf{e}_nA
		\end{bmatrix}& = \begin{bmatrix}
		\textbf{A}^{R}_1 \\ \vdots \\ \textbf{A}^{R}_n
		\end{bmatrix} = A \\
		AI_m = A\begin{bmatrix}
		\textbf{e}_1 & \cdots & \textbf{e}_m
		\end{bmatrix} = \begin{bmatrix}
		A\textbf{e}_1 & \cdots & A\textbf{e}_m
		\end{bmatrix}& = \begin{bmatrix}
		\textbf{A}^{C}_1 & \cdots & \textbf{A}^{C}_m
		\end{bmatrix} = A
		\end{align*}
	\end{enumerate}	
\end{proof}

\noindent \textit{Definition.} For a square matrix $A$ and nonnegative integer $n$, define $A^n$ as
\begin{align*}
A^n = \begin{cases}
I &\mbox{ if } n = 0 \\
A^{n-1}A = AA^{n-1} &\mbox{ otherwise }
\end{cases}
\end{align*} 

\begin{plaintheorem}[Matrix Powers]
	If $A$ is a square matrix and $r, s$ are nonnegative integers,
	\begin{enumerate}
		\item $A^rA^s = A^{r+s}$
		\item $(A^r)^s = A^{rs}$
	\end{enumerate}
\end{plaintheorem}


\begin{proof}
	Let $A$ be an $n \times n$ matrix.
	
	\begin{enumerate}
		\item Claim 1 : For a nonnegative integer $r$, $A^rA^s = A^{r+s}$ for all nonnegative integer $s$.
		
		\noindent (i) $A^rA^0 = A^rI_n = A^r$
		
		\noindent (ii) Suppose that $A^rA^k = A^{r+k}$. Then $A^rA^{k+1} = A^rA^kA = A^{r+k}A = A^{r+k+1}$.
		
		\noindent By (i), (ii), for all nonnegative integer $s$, $A^rA^s = A^{r+s}$.
		
		\noindent \\ Claim 2 : For a nonnegative integer $s$, $A^rA^s = A^{r+s}$ for all nonnegative integer $r$.
		
		\noindent (i) $A^0A^s = I_nA^s = A^s$
		
		\noindent (ii) Suppose that $A^kA^s = A^{k+s}$. Then $A^{k+1}A^s = AA^kA^s = AA^{k+s} = A^{k+s+1}$.
		
		\noindent By (i), (ii), for all nonnegative integer $r$, $A^rA^s = A^{r+s}$.
		
		\noindent \\ By Claim 1 and 2, for all nonnegative integers $r, s$, $A^rA^s = A^{r+s}$.
		
		\item For any nonnegative integer $r$,
		
		\noindent (i) $(A^r)^0 = I_n = A^{0} = A^{r\times0}$
		
		\noindent (ii) Suppose that $(A^r)^k = A^{rk}$. Then $(A^r)^{k+1} = (A^r)^kA^r = A^{rk}A^r = A^{rk+r} = A^{r(k+1)}$.
		
		\noindent By (i), (ii), for all nonnegative integers $r, s$, $(A^r)^s = A^{rs}$.
	\end{enumerate}
\end{proof}

\begin{theorem} [Properties of Transpose]
	Let $A$ and $B$ be matrices and let $k$ be a scalar. If the operations are defined,
	\begin{enumerate}
		\item $(A^{T})^{T} = A$
		\item $(A+B)^{T} = A^{T} + B^{T}$
		\item $(kA)^{T} = k(A^{T})$
		\item $(AB)^{T} = B^{T}A^{T}$
		\item $(A^{r})^{T} = (A^{T})^{r}$ for all r $\in \mathbb{Z}^{+}$
	\end{enumerate}
\end{theorem}

\begin{proof}
	Let $A$ and $B$ be matrices that all the operations are defined.
	\begin{enumerate}
		\item (Exercise 3.2 30) If $A$ is an $n \times m$ matrix, $A^{T}$ is an $m \times n$ matrix, thus $(A^{T})^{T}$ is an $n \times m$ matrix. The size of $(A^{T})^{T}$ and $A$ is the same.
		\begin{align*}
		((A^{T})^{T})_{ij} = (A^{T})_{ji} = A_{ij}			
		\end{align*}
		\item (Exercise 3.2 30) If $A$ and $B$ are both $n \times m$ matrices, the size of $(A+B)$ is also $n \times m$, so $(A+B)^{T}$ and $A^{T}$ and $B^{T}$ are all $m \times n$ matrices.
		\begin{align*}
		((A+B)^{T})_{ij} = (A+B)_{ji} = A_{ji} + B_{ji} = (A^{T})_{ij} + (B^{T})_{ij} = (A^{T} + B^{T})_{ij}
		\end{align*}
		\item (Exercise 3.2 30) If $A$ is an $n \times m$ matrix, both $(kA)^{T}$ and $k(A^{T})$ are $m \times n$ matrices.
		\begin{align*}
		((kA)^{T})_{ij} = (kA)_{ji} = k(A_{ji}) = k(A^{T})_{ij} = (k(A^{T}))_{ij}
		\end{align*}
		\item If $A$ is an $n \times m$ matrix and $B$ is an $m \times r$ matrices, then $(AB)^{T}$ is a $r \times n$ matrix. The size of $B^{T}$ and $A^{T}$ is $r \times m$ and $m \times n$, so the size of $B^{T}A^{T}$ is $r \times n$.
		\begin{align*}
		((AB)^{T})_{ij} = (AB)_{ji} = \textbf{A}^{R}_j \cdot \textbf{B}^{C}_i = (\textbf{A}^{T})^{C}_j \cdot (\textbf{B}^{T})^{R}_i = (B^{T}A^{T})_{ij}
		\end{align*}
		\item (Exercise 3.2 31) Let $A$ be a $n \times n$ matrix.
		
		\noindent (i) $(A^0)^{T} = (I_n)^{T} = I_n = (A^{T})^{0}$
		
		\noindent (ii) Suppose that for an arbitrary nonnegative integer $r$, $(A^r)^T = (A^T)^r$. Then
		\begin{align*}
		(A^{r+1})^T = (A^rA)^T = A^T(A^r)^T = A^T(A^T)^r = (A^T)^{r+1}
		\end{align*}
		By (i) and (ii), $(A^r)^T = (A^T)^r$ for all nonnegative integer $r$.
	\end{enumerate}
\end{proof}

\begin{theorem}
	For any matrix $A$,
	\begin{enumerate}
		\item If $A$ is a square matrix, then $A + A^{T}$ is symmetrical.
		\item $AA^{T}$ and $A^{T}A$ are symmetrical.
	\end{enumerate}
\end{theorem}

\begin{proof}
	\noindent
	\begin{enumerate}
		\item Let $A$ be a square matrix. Then
		\begin{align*}
		(A+A^T)^T = A^T + (A^T)^T = A^T + A = A+A^T
		\end{align*}
		Therefore $A+A^T$ is symmetrical.
		\item (Exercise 3.2 34) Let $A$ be a $n \times m$ matrix. Then the size of $A^T$ is $m \times n$ matrix, so the products $AA^T$ and $A^TA$ are defined.
		\begin{align*}
		& (AA^T)^T = (A^T)^TA^T = AA^T \\
		& (A^TA)^T = A^T(A^T)^T = A^TA
		\end{align*}
		Therefore $AA^T$ and $A^TA$ are symmetrical.
	\end{enumerate}
\end{proof}

\section{Other Types of Matrices : Exercise 3.2}
\textit{Definition.} An \textbf{upper triangular} matrix is a square matrix with all of its entries below the main diagonal are zero. In other words, $A$ is an upper triangular matrix if
\begin{align*}
A_{ij} = 0 \textnormal{ if } i > j
\end{align*}

\begin{plaintheorem}[Properties of Upper Triangular Matrices]
	If $A$ and $B$ are both upper triangular matrices in the same size, the product $AB$ is also an upper triangular matrix.
\end{plaintheorem}

\begin{proof}
	(Exercise 3.2 29)
	
	\noindent (i) If $A$ and $B$ are both $1 \times 1$ upper triangular matrices, let $A = [A_{11}]$ and $B = [B_{11}]$. Then $AB = [A_{11}B_{11}]$, which is also an upper triangular matrix.
	
	\noindent (ii) Suppose that the product of $n \times n$ upper triangular matrices is also upper triangular matrix. Let $A$ and $B$ be both $n+1 \times n+1$ upper triangular matrices. Then we can partition $A$ and $B$ as
	\begin{align*}
	A = \begin{bmatrix}
	A_{11} & A_{12} \\ O & A_{22}
	\end{bmatrix}, B = \begin{bmatrix}
	B_{11} & B_{12} \\ O & B_{22}
	\end{bmatrix}
	\end{align*} where $A_{11}$ and $B_{11}$ are $n \times n$ matrices, $A_{12}$ and $B_{12}$ are $1 \times n$ matrices, and $A_{22}$ and $B_{22}$ are scalars. Since $A$ and $B$ are upper triangular, $A_{11}$ and $B_{11}$ are also upper triangular matrices.
	\begin{align*}
	AB = \begin{bmatrix}
	A_{11}B_{11} + A_{12}O & A_{11}B_{12} + A_{12}B_{22} \\
	OB_{11} + A_{22}O & OB_{12} + A_{22}B_{22}
	\end{bmatrix} = \begin{bmatrix}
	A_{11}B_{11} & A_{11}B_{12} + A_{12}B_{22} \\
	O & A_{22}B_{22}
	\end{bmatrix}
	\end{align*}
	Since $A_{11}B_{11}$ is an upper triangular matrix, $AB$ is also upper triangular.
	
	\noindent By (i) and (ii), for all upper triangular matrices $A$ and $B$, the product $AB$ is also upper triangular. 
\end{proof}

\begin{plaintheorem}[Properties of Symmetric Matrices]
	\begin{enumerate}
		\item If $A$ and $B$ are symmetric, then $A+B$ is also symmetric.
		\item If $A$ is symmetric, then $kA$ is also symmetric for any scalar $k$.
		\item If $A$ and $B$ are symmetric, then $AB$ is symmetric if and only if $AB = BA$.
	\end{enumerate}
\end{plaintheorem}


\begin{proof}
	Let $A$ and $B$ be symmetric matrices that the operations are defined, and let $k$ be a scalar.
	\begin{enumerate}
		\item (Exercise 3.2 35)
		\begin{align*}
		(A+B)^{T} = A^{T} + B^{T} = A+B
		\end{align*} Therefore $A+B$ is symmetric.
		\item (Exercise 3.2 35)
		\begin{align*}
		(kA)^{T} = k(A^{T}) = kA
		\end{align*} Therefore $kA$ is symmetric.
		\item (Exercise 3.2 36)
		\begin{align*}
		(AB)^{T} = B^{T}A^{T} = BA
		\end{align*} Therefore $AB$ is symmetric if and only if $AB = BA$.
	\end{enumerate}
\end{proof}
\noindent \textit{Definition.} A square matrix is \textbf{skew-symmetric} if $A^T = -A$. In other words, $A_{ij} = -A_{ji}$.
\begin{plaintheorem}[Properties of Skew-Symmetric Matrices]
	\begin{enumerate}
		\item If $A$ and $B$ are skew-symmetric, $A+B$ is also skew-symmetric.
		\item If $A$ is a square matrix, then $A-A^T$ is skew-symmetric.
		\item Any square matrix $A$ can be represented as the sum of a symmetric matrix and a skew-symmetric matrix.
	\end{enumerate}
\end{plaintheorem}


\begin{proof}
	Let $A$ and $B$ be square matrices that the operations are defined.
	\begin{enumerate}
		\item (Exercise 3.2 40) Suppose that $A$ and $B$ are skew-symmetric, so that $A^T = -A$ and $B^T = -B$. Then
		\begin{align*}
		(A+B)^T = A^T + B^T = (-A) + (-B) = -(A+B)
		\end{align*} Therefore $A+B$ is also skew-symmetric.
		\item (Exercise 3.2 42)
		\begin{align*}
		(A - A^T)^T = A^T - (A^T)^T = A^T - A = -(A - A^T)
		\end{align*} Therefore $A - A^T$ is also skew-symmetric.
		\item (Exercise 3.2 43) By Theorem : Properties of Symmetric Matrices and Theorem : Properties of Skew-Symmetric Matrices, $A + A^T$ is symmetric and $A - A^T$ is skew-symmetric. Then representing $A$ as
		\begin{align*}
		A = \frac{1}{2}(A + A^T) + \frac{1}{2}(A - A^T)
		\end{align*} which is a sum of symmetric matrix and skew-symmetric matrix.
	\end{enumerate}
\end{proof}

\noindent \textit{Definition.} A \textbf{trace} of a $n \times n$ matrix $A$ is denoted as tr($A$) and defined as
\begin{align*}
\textnormal{tr}(A) = A_{11} + A_{22} + \cdots + A_{nn}
\end{align*}

\begin{plaintheorem}[Properties of the Trace of Matrices]
	Let $A$ and $B$ be $n \times n$ matrices, and let $k$ be a scalar.
	\begin{enumerate}
		\item tr($A+B$) = tr($A$) + tr($B$)
		\item tr($kA$) = $k$tr($A$)
		\item tr($AB$) = tr($BA$)
		\item tr($AA^T$) = $\sum_{1 \le i, j \le n}^{} A_{ij}^{2}$
	\end{enumerate}
\end{plaintheorem}


\begin{proof}
	Let $A$ and $B$ be $n \times n$ matrices, and let $k$ be a scalar.
	\begin{enumerate}
		\item (Exercise 3.2 44)
		\begin{align*}
		\textnormal{tr}(A+B) &= (A+B)_{11} + (A+B)_{22} + \cdots + (A+B)_{nn} \\
		&= (A_{11} + B_{11}) + (A_{22} + B_{22}) + \cdots + (A_{nn} + B_{nn}) \\
		&= (A_{11} + A_{22} + \cdots + A_{nn}) + (B_{11} + B_{22} + \cdots + B_{nn}) = \textnormal{tr}(A) + \textnormal{tr}(B)
		\end{align*}
		\item (Exercise 3.2 44)
		\begin{align*}
		\textnormal{tr}(kA) &= (kA)_{11} + (kA)_{22} + \cdots + (kA)_{nn} \\
		&= k(A_{11} + A_{22} + \cdots + A_{nn}) = k\textnormal{tr}(A)
		\end{align*}
		\item (Exercise 3.2 45)
		\begin{align*}
		\textnormal{tr}(AB) &= (AB)_{11} + (AB)_{22} + \cdots + (AB)_{nn} \\
		&= \sum_{1 \le i, j \le n}^{} A_{ij}B_{ji} \\
		&= \sum_{1 \le i, j \le n}^{} B_{ij}A_{ji} \\
		&= (BA)_{11} + (BA)_{22} + \cdots + (BA)_{nn} = \textnormal{tr}(BA)
		\end{align*}
		\item (Exercise 3.2 46)
		\begin{align*}
		\textnormal{tr}(AA^T) &= (AA^T)_{11} + (AA^T)_{22} + \cdots + (AA^T)_{nn} \\
		&= \sum_{1 \le i, j \le n}^{} A_{ij}(A^T)_{ji} \\
		&= \sum_{1 \le i, j \le n}^{} A_{ij}^{2}
		\end{align*}
	\end{enumerate}
\end{proof}

\section{The Inverse of a Matrix}
\textit{Definition.} If $A$ is an $n \times n$ matrix, an \textbf{inverse} of $A$ is an $n \times n$ matrix, denoted as $\inv{A}$, such that
\begin{align*}
A\inv{A} = I_n \textnormal{ and } \inv{A}A = I_n
\end{align*} If such $\inv{A}$ exists, $A$ is \textbf{invertible}.

\begin{theorem}
	If $A$ is invertible, then its inverse is unique.
\end{theorem}

\begin{proof}
	Suppose that $A'$ and $A''$ are both inveses of $A$. Then
	\begin{align*}
	AA' = I = A'A \textnormal{ and } AA'' = I = A''A
	\end{align*}
	Thus, \begin{align*}
	A' = A'I = A'(AA'') = (A'A)A'' = IA'' = A''
	\end{align*} Therefore, the inverse of $A$ is unique.
\end{proof}

\begin{theorem}
	If $A$ is an invertible $n \times n$ matrix, then the system of linear equations given by $A\textbf{x} =\textbf{b}$ has the unique solution $\textbf{x} = \inv{A}\textbf{b}$.
\end{theorem}

\begin{proof}
	(i) \begin{align*}
	A(\inv{A}\textbf{b}) = (A\inv{A})\textbf{b} = I_n\textbf{b} = \textbf{b}
	\end{align*} Therefore the system is consistent.
	
	\noindent (ii) Suppose that $\textbf{x}'$ is another solution of $A\textbf{x} = \textbf{b}$. Then
	\begin{align*}
	A\textbf{x}' = \textbf{b} & \Rightarrow \inv{A}(A\textbf{x}') = \inv{A}\textbf{b}
	& \Rightarrow (\inv{A}A)\textbf{x}' = \textbf{x}' = \inv{A}\textbf{b}
	\end{align*} Thus, $\textbf{x}'$ is the same solution as before. Therefore, the solution is unique.
\end{proof}

\begin{theorem}
	If $A = \begin{bmatrix}
	a & b \\ c & d
	\end{bmatrix}$, then $A$ is invertible if and only if $ad - bc \neq 0$ (a \textbf{determinant} of $A$, denoted as det $A$), and in that case \begin{align*}
	\inv{A} = \frac{1}{ad-bc}\begin{bmatrix}
	d & -b \\ -c & a
	\end{bmatrix}
	\end{align*}
\end{theorem}

\begin{proof}
	($\Leftarrow$) If $ad - bc \neq 0$,
	\begin{align*}
	\begin{bmatrix}
	a & b \\ c & d
	\end{bmatrix} (\frac{1}{ad-bc}\begin{bmatrix}
	d & -b \\ -c & a
	\end{bmatrix}) &= \frac{1}{ad-bc}\begin{bmatrix}
	ad-bc & 0 \\ 0 & ad-bc
	\end{bmatrix} = \begin{bmatrix}
	1 & 0 \\ 0 & 1
	\end{bmatrix} \\
	(\frac{1}{ad-bc}\begin{bmatrix}
	d & -b \\ -c & a
	\end{bmatrix}) \begin{bmatrix}
	a & b \\ c & d
	\end{bmatrix} &= \frac{1}{ad-bc}\begin{bmatrix}
	ad-bc & 0 \\ 0 & ad-bc
	\end{bmatrix} = \begin{bmatrix}
	1 & 0 \\ 0 & 1
	\end{bmatrix}
	\end{align*}
	Therefore, $A$ is invertible and $\inv{A} = \frac{1}{ad-bc}\begin{bmatrix}
	d & -b \\ -c & a
	\end{bmatrix}$
	
	\noindent ($\Rightarrow$) Suppose that $ad - bc = 0$ and scalars $x, y, z, w$ exist such that \begin{align*}
	\begin{bmatrix}
	a & b \\ c & d
	\end{bmatrix} \begin{bmatrix}
	x & y \\ z & w
	\end{bmatrix} = \begin{bmatrix}
	x & y \\ z & w
	\end{bmatrix} \begin{bmatrix}
	a & b \\ c & d
	\end{bmatrix} = \begin{bmatrix}
	1 & 0 \\ 0 & 1
	\end{bmatrix}
	\end{align*}
	Then \begin{align*}
	\begin{bmatrix}
	a & b \\ c & d
	\end{bmatrix} \begin{bmatrix}
	x & y \\ z & w
	\end{bmatrix} = \begin{bmatrix}
	ax+bz & ay+bw \\ cx+dz & cy+dw
	\end{bmatrix} = \begin{bmatrix}
	1 & 0 \\ 0 & 1
	\end{bmatrix}
	\end{align*}
	Since $ad = bc$, \begin{align*}
	1 = (ax+bz)(cy+dw) &= acxy + adxw + bcyz + bdwz \\
	&= acxy + bcxw + adyz + bdwz = (ay+bw)(cx+dz) = 0
	\end{align*} Thus, such $x, y, z, w$ does not exist, and therefore $A$ is not invertible.
\end{proof}

\begin{theorem} [Properties of Invertible Matrices]
	If $A$, $B$ are invertible matrices of the same size and $c$ is a nonzero scalar,
	\begin{enumerate}
		\item $\inv{A}$ is invertible and \begin{align*}
		\inv{(\inv{A})} = A
		\end{align*}
		\item $cA$ is invertible and \begin{align*}
		\inv{(cA)} = \frac{1}{c}\inv{A}
		\end{align*}
		\item $AB$ is invertible and \begin{align*}
		\inv{(AB)} = \inv{B}\inv{A}
		\end{align*}
		\item $A^T$ is invertible and \begin{align*}
		\inv{(A^T)} = (\inv{A})^{T}
		\end{align*}
		\item $A^n$ is invertible for all nonnegative integers $n$ and \begin{align*}
		\inv{(A^n)} = (\inv{A})^n
		\end{align*}
	\end{enumerate}
\end{theorem}

\begin{proof}
	Let $A$ and $B$ be invertible matrices of the same size and let $c$ be a nonzero scalar.
	\begin{enumerate}
		\item
		\begin{align*}
		\inv{A}A = A\inv{A} = I
		\end{align*} thus $A$ is an inverse of $\inv{A}$.
		\item (Exercise 3.3 14)
		\begin{align*}
		(cA)(\frac{1}{c}\inv{A}) &= (c\frac{1}{c})(A\inv{A}) = I \\
		(\frac{1}{c}\inv{A})(cA) &= (\frac{1}{c}c)(\inv{A}A) = I
		\end{align*} thus $\frac{1}{c}\inv{A}$ is an inverse of $cA$.
		\item
		\begin{align*}
		(AB)(\inv{B}\inv{A}) &= A(B\inv{B})\inv{A} = A\inv{A} = I \\
		(\inv{B}\inv{A})(AB) &= \inv{B}(\inv{A}A)B = \inv{B}B = I
		\end{align*} thus $\inv{B}\inv{A}$ is an inverse of $AB$.
		\item (Exercise 3.3 15)
		\begin{align*}
		A^T(\inv{A})^T &= (\inv{A}A)^T = I^T = I \\
		(\inv{A})^TA^T &= (A\inv{A})^T = I^T = I
		\end{align*} thus $(\inv{A})^T$ is an inverse of $A^T$.
		\item (i) $\inv{(A^0)} = \inv{I} = I = (\inv{A})^0$
		
		\noindent (ii) Suppose that for arbitrary nonnegative integer $n$, $\inv{(A^n)} = (\inv{A})^n$. Then \begin{align*}
		\inv{(A^{n+1})} = \inv{(A^nA)} = \inv{A}\inv{(A^n)} = \inv{A}(\inv{A})^n = (\inv{A})^{n+1}
		\end{align*}
		By (i) and (ii), $\inv{(A^n)} = (\inv{A})^n$ for any nonnegative integer $n$.
	\end{enumerate}
\end{proof}

\noindent \textit{Definition.} An \textbf{elementary matrix} is any matrix that can be obatined by performing \textit{single} elementary row operation on an identity matrix.

\noindent \\ \textbf{Types of Elementary Matrices}

\noindent Let $E$ be an $n \times n$ matrix. Then $E = \begin{bmatrix}
\textbf{E}^R_1 \\ \textbf{E}^R_2 \\ \vdots \\ \textbf{E}^R_n
\end{bmatrix}$ is an elementary matrix if and only if $E$ satisfies at least one of the followings:
\begin{enumerate}
	\item There exist integers $p, q$ such that $1 \le p, q \le n$ and \begin{align*}
	\textbf{E}^R_i = \begin{cases}
	\textbf{e}_q &\mbox{ if } i=p \\
	\textbf{e}_p &\mbox{ if } i=q \\
	\textbf{e}_i &\mbox{ otherwise } 
	\end{cases}
	\end{align*}
	\item There exists an integer $p$ and a nonzero scalar $k$ such that $1 \le p \le n$ and
	\begin{align*}
	\textbf{E}^R_i = \begin{cases}
	k\textbf{e}_p &\mbox{ if } i=p \\
	\textbf{e}_i &\mbox{ otherwise }
	\end{cases}
	\end{align*}
	\item There exists integers $p, q$ and a scalar $k$ such that $1 \le p, q \le n$ and \begin{align*}
	\textbf{E}^R_i = \begin{cases}
	\textbf{e}_p + k\textbf{e}_q &\mbox{ if } i=p \\
	\textbf{e}_i &\mbox{ otherwise }
	\end{cases}
	\end{align*}
\end{enumerate}

\begin{theorem}
	If a elementary row operation converts $I_n$ to $E$, then the same operation converts $n \times r$ matrix $A$ to $EA$.
\end{theorem}

\begin{proof}
	Let $A$ be a $n \times r$ matrix, $k$ be a nonzero scalar, $p, q$ be integers between 1 and $n$, and $E$ be an elementary matrix which corresponds to each type of elementary row operation.
	\begin{align*}
	A = \begin{bmatrix}
	\textbf{A}^R_1 \\ \vdots \\ \textbf{A}^R_p \\ \vdots \\ \textbf{A}^R_q \\ \vdots \\ \textbf{A}^R_n
	\end{bmatrix} &\xrightarrow{R_p \leftrightarrow R_q} \begin{bmatrix}
	\textbf{A}^R_1 \\ \vdots \\ \textbf{A}^R_q \\ \vdots \\ \textbf{A}^R_p \\ \vdots \\ \textbf{A}^R_n  
	\end{bmatrix} = \begin{bmatrix}
	\textbf{e}_1A \\ \vdots \\ \textbf{e}_qA \\ \vdots \\ \textbf{e}_pA \\ \vdots \\ \textbf{e}_nA
	\end{bmatrix} = \begin{bmatrix}
	\textbf{e}_1 \\ \vdots \\ \textbf{e}_q \\ \vdots \\ \textbf{e}_p \\ \vdots \\ \textbf{e}_n
	\end{bmatrix}A = EA \\
	A = \begin{bmatrix}
	\textbf{A}^R_1 \\ \vdots \\ \textbf{A}^R_p \\ \vdots \\ \textbf{A}^R_n
	\end{bmatrix} &\xrightarrow{kR_p} \begin{bmatrix}
	\textbf{A}^R_1 \\ \vdots \\ k\textbf{A}^R_p \\ \vdots \\ \textbf{A}^R_n
	\end{bmatrix} = \begin{bmatrix}
	\textbf{e}_1A \\ \vdots \\ k(\textbf{e}_pA) \\ \vdots \\ \textbf{e}_nA
	\end{bmatrix} = \begin{bmatrix}
	\textbf{e}_1 \\ \vdots \\ k\textbf{e}_p \\ \vdots \\ \textbf{e}_n
	\end{bmatrix}A = EA \\
	A = \begin{bmatrix}
	\textbf{A}^R_1 \\ \vdots \\ \textbf{A}^R_p \\ \vdots \\ \textbf{A}^R_n
	\end{bmatrix} &\xrightarrow{R_p + kR_q} \begin{bmatrix}
	\textbf{A}^R_1 \\ \vdots \\ \textbf{A}^R_p + k\textbf{A}^R_q \\ \vdots \\ \textbf{A}^R_n
	\end{bmatrix} = \begin{bmatrix}
	\textbf{e}_1A \\ \vdots \\ \textbf{e}_pA + k(\textbf{e}_qA) \\ \vdots \\ \textbf{e}_nA
	\end{bmatrix} = \begin{bmatrix}
	\textbf{e}_1 \\ \vdots \\ \textbf{e}_p + k\textbf{e}_q \\ \vdots \\ \textbf{e}_n
	\end{bmatrix}A = EA
	\end{align*}
\end{proof}

\begin{theorem}
	All elementary matrices are invertible, and the inverse of an elementary matrix is also an elementary matrix.
\end{theorem}

\begin{proof}
	Let $E$ be an elementary matrix. Since there exists a reverse operation for the elementary row operations corresponds to $E$, the elementary matrix $E'$ of the reverse operation will satisfy $EE' = E'E = I$ by Theorem 3.10. Therefore, the inverse of $E$ exists and the inverse is also an elementary matrix.
\end{proof}

\begin{theorem}[The Fundamental Theorem of Invertible Matrices - Version 1]
	Let $A$ be an $n \times n$ matrix. Then the following propositions are equivalent;
	\begin{enumerate}
		\item $A$ is invertible.
		\item $A\textbf{x} = \textbf{b}$ has a unique solution for every $\textbf{b} \in$ \Rn.
		\item $A\textbf{x} = \textbf{0}$ has only the trivial solution.
		\item The RREF of $A$ is $I_n$.
		\item $A$ is a product of elementary matrices.
	\end{enumerate}
\end{theorem}

\begin{proof}
	We give the proof for this theorem at Version 2.
\end{proof}

\begin{theorem}
	Let $A$ be a square matrix. If $B$ is a square matrix such that either $AB = I$ or $BA = I$, then $A$ is invertible and $\inv{A} = B$.
\end{theorem}

\begin{proof}
	(i) Suppose that $BA = I$. Then the linear system $A\textbf{x} = \textbf{0}$ has only the trivial solution, since \begin{align*}
	\textbf{x} = I\textbf{x} = (BA)\textbf{x} = B(A\textbf{x}) = \textbf{0}
	\end{align*}
	Thus $A$ is invertible by F.T.I.M. Then \begin{align*}
	BA = I \Rightarrow (BA)\inv{A} = \inv{A} \Rightarrow B(A\inv{A}) = B = \inv{A}
	\end{align*} Therefore $B$ is an inverse of $A$.
	
	\noindent (ii) (Exercise 3.3 41) Suppose that $AB = I$. By (i), $B$ is invertible and $\inv{B} = A$. Therefore, $A$ is also invertible and $\inv{A} = \inv{(\inv{B})} = B$.
\end{proof}

\begin{theorem}
	Let $A$ be a square matrix. Then a sequence of elementary row operations which converts $A$ to $I$ also converts $I$ to $\inv{A}$.
\end{theorem}

\begin{proof}
	Let $E_1, E_2, \cdots, E_k$ be the elementary matrices which correspond to each step of elementary row operation converting $A$ to $I$. Then by Theorem 3.10, \begin{align*}
	I = (E_k\cdots E_2E_1)A
	\end{align*} Then by Theorem 3.13, \begin{align*}
	\inv{A} = E_k\cdots E_2E_1 = (E_k\cdots E_2E_1)I
	\end{align*} Therefore, the same steps of elementary row operation also converts $I$ to $\inv{A}$.
\end{proof}

\section{Subspaces, Basis, Dimension, and Rank}

\textit{Definition.} Let $S$ be a set of vectors in \Rn. Then $S$ is a \textbf{subspace} of \Rn if \begin{enumerate}
	\item $\textbf{0} \in S$
	\item $\textbf{u}, \textbf{v} \in S \Rightarrow \textbf{u}+\textbf{v} \in S$
	\item $\textbf{u} \in S \Rightarrow c\textbf{u} \in S$ for any scalar $c$
\end{enumerate}

\noindent \textit{Definition.} Let $A$ be an $n \times m$ matrix.

\noindent The \textbf{row space} of $A$, denoted by row($A$), is a subspace of \Rm spanned by the rows of $A$.

\noindent The \textbf{column space} of $A$, denoted by col($A$), is a subspace of \Rn spanned by the columns of $A$.

\noindent The \textbf{null space} of $A$, denoted by null($A$), is a subspace of \Rm consisting of solutions of the homogeneous linear system $A\textbf{x} = \textbf{0}$.  

\begin{theorem}
	For vectors \vk in \Rn, span( \vk) is a subspace of \Rn.	
\end{theorem}

\begin{proof}
	Let \vk be vectors in \Rn, and let $S$ = span( \vk).
	
	\noindent (i) Since $0\textbf{v}_1 + 0\textbf{v}_2 + \cdots + 0\textbf{v}_k = \textbf{0}$, $\textbf{0} \in S$.
	
	\noindent (ii) Let $\textbf{u}$ and $\textbf{v}$ be vectors in $S$. Then there exist scalars $c_1, c_2 \cdots, c_k$ and $d_1, d_2, \cdots, d_k$ such that \begin{align*}
	\textbf{u} &= c_1\textbf{v}_1 + c_2\textbf{v}_2 + \cdots + c_k\textbf{v}_k \\
	\textbf{v} &= d_1\textbf{v}_1 + d_2\textbf{v}_2 + \cdots + d_k\textbf{v}_k
	\end{align*} Then \begin{align*}
	\textbf{u} + \textbf{v} &= (c_1\textbf{v}_1 + c_2\textbf{v}_2 + \cdots + c_k\textbf{v}_k) + (d_1\textbf{v}_1 + d_2\textbf{v}_2 + \cdots + d_k\textbf{v}_k) \\ &= (c_1+d_1)\textbf{v}_1 + \cdots + (c_k+d_k)\textbf{v}_k
	\end{align*} Therefore $\textbf{u}+\textbf{v} \in S$.
	
	\noindent (iii) Let $\textbf{u}$ be a vector in $S$. Then there exist scalars $c_1, c_2, \cdots, c_k$ such that \begin{align*}
	\textbf{u} &= c_1\textbf{v}_1 + c_2\textbf{v}_2 + \cdots + c_k\textbf{v}_k
	\end{align*} Then for all scalar $c$, \begin{align*}
	c\textbf{u} &= c(c_1\textbf{v}_1 + c_2\textbf{v}_2 + \cdots + c_k\textbf{v}_k) \\
	&= cc_1\textbf{v}_1 + cc_2\textbf{v}_2 + \cdots + cc_k\textbf{v}_k
	\end{align*} Therefore $c\textbf{u} \in S$.
	
	\noindent By (i), (ii), and (iii), $S$ is a subspace of \Rn.
\end{proof}

\begin{theorem}
	If matrices $A$ and $B$ are row equivalent, then row($A$) = row($B$).
\end{theorem}

\begin{proof}
	Let $A$ and $B$ be $n \times m$ matrices. Since $A$ and $B$ are row equivalent, there exist $n \times n$ elementary matrices $E_1, E_2, \cdots ,E_k$ which satisfies \begin{align*}
	B = (E_k\cdots E_2E_1)A
	\end{align*} by Theorem 3.10. Let $C = E_k\cdots E_2E_1$.
	
	\noindent Since $B = CA$, \begin{align*}
	B = \begin{bmatrix}
	\textbf{C}^R_1 \\ \vdots \\ \textbf{C}^R_n
	\end{bmatrix}A = \begin{bmatrix}
	\textbf{C}^R_1A \\ \vdots \\ \textbf{C}^R_nA
	\end{bmatrix} = \begin{bmatrix}
	C_{11}\textbf{A}^R_1 + \cdots + C_{1n}\textbf{A}^R_n \\ \vdots \\
	C_{n1}\textbf{A}^R_1 + \cdots + C_{nn}\textbf{A}^R_n
	\end{bmatrix}
	\end{align*} Thus, rows of $B$ are linear combinations of rows of $A$. Also, since $C$ is invertible by Theorem 3.11, $A = \inv{C}B$, so rows of $A$ are linear combinations of rows of $B$. Therefore, by the proposition at Exercise 2.3 21(b), span($\textbf{A}^R_1, \cdots, \textbf{A}^R_n$) = span($\textbf{B}^R_1, \cdots, \textbf{B}^R_n$), that is row($A$) = row($B$).
\end{proof}

\begin{theorem}
	Let $A$ be an $n \times m$ matrix and let $N$ be the set of solutions of the homogeneous linear system $A\textbf{x} = \textbf{0}$. Then $N$ is a subspace of \Rm.
\end{theorem}

\begin{proof}
	Let $A$ be an $n \times m$ matrix and let $N$ be the set of all solutions of $A\textbf{x} = \textbf{0}$.
	
	\noindent (i) Since $A\textbf{0} = \textbf{0}$, $\textbf{0} \in N$.
	
	\noindent (ii) Let $\textbf{u}, \textbf{v}$ be vectors in $N$. Since $A\textbf{u} = A\textbf{v} = \textbf{0}$, $A(\textbf{u}+\textbf{v}) = A\textbf{u} + A\textbf{v} = \textbf{0}$. Therefore, $\textbf{u} + \textbf{v} \in N$.
	
	\noindent (iii) Let $\textbf{u}$ be a vector in $N$. Since $A\textbf{u} = \textbf{0}$, for all scalar $c$, $A(c\textbf{u}) = c(A\textbf{u}) = c\textbf{0} = \textbf{0}$. Therefore, $c\textbf{u} \in N$.
	
	\noindent By (i), (ii), and (iii), $N$ is a subspace of \Rn.
\end{proof}

\begin{theorem}
	For any system of linear equations $A\textbf{x} = \textbf{b}$ with real coefficients, exactly one of the following is true; \begin{enumerate}
		\item The system has no solution.
		\item The system has a unique solution.
		\item The system has infinitely many solutions.
	\end{enumerate}
\end{theorem}

\begin{proof}
	Consider the case when the system of linear equations has more than two solutions. Suppose that $\textbf{x}_1, \textbf{x}_2 \in$ \Rn are solutions of the linear system $A\textbf{x} = \textbf{b}$. Then $A(\textbf{x}_2 - \textbf{x}_1) = \textbf{0}$, so $\textbf{x}_2 - \textbf{x}_1 \in$ null($A$). By the definition of subspace, for any scalar $c$, $c(\textbf{x}_2 - \textbf{x}_1) \in$ null($A$), so $A(c(\textbf{x}_2 - \textbf{x}_1)) = \textbf{0}$. Since $A(\textbf{x}_1 + c(\textbf{x}_2 - \textbf{x}_1)) = A\textbf{x}_1 + A(c(\textbf{x}_2 - \textbf{x}_1)) = \textbf{b} + \textbf{0} = \textbf{b}$, there exist infinitely many solutions.
\end{proof}

\noindent \textit{Definition.} A \textbf{basis} for a subspace $S$ in \Rn is a set of vectors in $S$ that \begin{enumerate}
	\item spans $S$
	\item is linearly independent.
\end{enumerate}

\noindent \textbf{How to Find Bases of Subspaces Related to Matrix}
\begin{enumerate}
	\item The basis of row($A$) consists of the nonzero rows of RREF of the matrix.
	\item The basis of col($A$) consists of the columns of $A$ where the leading 1s of $A$ are located.
	\item The basis of null($A$) consists of the vectors which spans the solution of $A\textbf{x} = \textbf{0}$.
\end{enumerate} 

\noindent \textbf{How to Find Subspaces Related to Matrix : Example}

\noindent Find bases of row($A$), col($A$), and null($A$) where \begin{align*}
A = \begin{bmatrix}
1 & 1 & 3 & 1 & 6 \\ 2 & -1 & 0 & 1 & -1 \\
-3 & 2 & 1 & -2 & 1 \\ 4 & 1 & 6 & 1 & 3
\end{bmatrix}
\end{align*}

\noindent \textit{Solution.} The RREF of $A$ is \begin{align*}
R = \begin{bmatrix}
1 & 0 & 1 & 0 & -1 \\ 0 & 1 & 2 & 0 & 3 \\
0 & 0 & 0 & 1 & 4 \\ 0 & 0 & 0 & 0 & 0
\end{bmatrix}
\end{align*}
The basis of row($A$) is \begin{align*}
\left\{ \begin{bmatrix}
1 & 0 & 1 & 0 & -1
\end{bmatrix}, \begin{bmatrix}
0 & 1 & 2 & 0 & 3
\end{bmatrix}, \begin{bmatrix}
0 & 0 & 0 & 1 & 4
\end{bmatrix} \right\}
\end{align*}
The basis of col($A$) consists of the first, second, and fourth columns of $A$, so the basis of col($A$) is \begin{align*}
\left\{ \begin{bmatrix}
1 \\ 2 \\ -3 \\ 4
\end{bmatrix}, \begin{bmatrix}
1 \\ -1 \\ 2 \\ 1
\end{bmatrix}, \begin{bmatrix}
1 \\ 1 \\ -2 \\ 1
\end{bmatrix} \right\}
\end{align*}
The solution of $A\textbf{x} = \textbf{0}$ is \begin{align*}
\textbf{x} = s\begin{bmatrix}
-1 \\ -2 \\ 1 \\ 0 \\ 0
\end{bmatrix} + t\begin{bmatrix}
1 \\ -3 \\ 0 \\ -4 \\ 1
\end{bmatrix} (s,t \in \mathbb{R})
\end{align*}
thus the basis of null($A$) is \begin{align*}
\left\{
\begin{bmatrix}
-1 \\ -2 \\ 1 \\ 0 \\ 0
\end{bmatrix}, \begin{bmatrix}
1 \\ -3 \\ 0 \\ -4 \\ 1
\end{bmatrix}
\right\}
\end{align*}

\begin{theorem}
	Let $S$ be a subspace of \Rn. Then any two bases for $S$ have the same number of vectors.
\end{theorem}

\begin{proof}
	Let $\mathcal{B}_1 = \{\textbf{u}_1, \cdots, \textbf{u}_k\}$ and $\mathcal{B}_2 = \{\textbf{v}_1, \cdots, \textbf{v}_m \}$ be bases for subspace $S$ of \Rn.
	
	\noindent Suppose that $k > m$. Since $\mathcal{B}_1 \subset S$ and $\mathcal{B}_2$ is a basis for $S$, there exist scalars $a_{ij}$ such that \begin{align*}
	\textbf{u}_1 &= a_{11}\textbf{v}_1 + \cdots + a_{1m}\textbf{v}_m \\
	&\vdots \\
	\textbf{u}_k &= a_{k1}\textbf{v}_1 + \cdots + a_{km}\textbf{v}_m
	\end{align*} Since $\mathcal{B}_2$ is a basis for $S$, $\textbf{v}_1, \cdots, \textbf{v}_m$ are linearly independent. So the linear system \begin{align*}
	&c_1\textbf{u}_1 + \cdots + c_k\textbf{u}_k \\ &= (c_1a_{11} + \cdots + c_ka_{k1})\textbf{v}_1 + (c_1a_{12} + \cdots + c_ka_{k2})\textbf{v}_2 + \cdots + (c_1a_{1m} + \cdots + c_ka_{km})\textbf{v}_m = \textbf{0}
	\end{align*} has only the trivial solution. Thus, the homogeneous linear system with augmented matrix \begin{align*}
	\begin{bmatrix}[ccc|c]
	a_{11} & \cdots & a_{k1} & 0 \\
	\vdots & & \vdots & \vdots \\
	a_{1m} & \cdots & a_{km} & 0
	\end{bmatrix}
	\end{align*} is consistent, and the system has infinitely many solutions because the size of the matrix is $m \times k$. There exists a nontrivial solution of this system, so there exist scalars $c_1, \cdots, c_k$ such that at least one of $c_1, \cdots, c_k$ are nonzero, and $c_1\textbf{u}_1 + \cdots + c_k\textbf{u}_k = \textbf{0}$. Therefore, $\textbf{u}_1, \cdots, \textbf{u}_k$ are linearly dependent, which contradicts with the assumption.
	
	\noindent Similarly, in case of $k < m$ also comes up to contradiction. Therefore, $k = m$.
\end{proof}

\noindent \textit{Definition.} If $S$ is a subspace of \Rn, then the number of vectors in a basis for $S$ is the \textbf{dimension} of $S$, denoted by dim $S$.

\begin{theorem}
	For any matrix $A$,
	\begin{align*}
	\textnormal{dim row}(A) = \textnormal{dim col}(A)
	\end{align*}
\end{theorem}

\begin{proof}
	The number of nonzero rows and the number of leading 1s are the same. Since the nonzero rows form the basis for row($A$), and the columns of $A$ with the leading 1s form the basis for col($A$), dim row($A$) = dim col($A$).
\end{proof}

\noindent \textit{Definition.} The \textbf{rank} of a matrix $A$, denoted by rank($A$), is the dimension of its row and column spaces.

\begin{theorem}
	For any matrix $A$,
	\begin{align*}
	\textnormal{rank}(A) = \textnormal{rank}(A^T)
	\end{align*}
\end{theorem}

\begin{proof}
	\begin{align*}
	\mbox{rank}(A) = \mbox{dim row}(A) = \mbox{dim col}(A) = \mbox{dim row}(A^T) = \mbox{rank}(A^T)
	\end{align*}
\end{proof}

\noindent \textit{Definition.} The \textbf{nullity} of a matrix $A$, denoted by nullity($A$), is the dimension of its null space.

\begin{theorem}[The Rank Theorem]
	If $A$ is an $m \times n$ matrix, then \begin{align*}
	\textnormal{rank}(A) + \textnormal{nullity}(A) = n
	\end{align*}
\end{theorem}

\begin{proof}
	By Theorem 2.2.
\end{proof}

\begin{theorem} [The Fundamental Theorem of Invertible Matrices : Version 2]
	Let $A$ be an $n \times n$ matrix. The following propositions are equivalent;
	\begin{enumerate}
		\item $A$ is invertible.
		\item $A\textbf{x} = \textbf{b}$ has a unique solution for every $\textbf{b} \in$ \Rn.
		\item $A\textbf{x} = \textbf{0}$ has only the trivial solution.
		\item The RREF of $A$ is $I_n$.
		\item $A$ is a product of elementary matrices.
		\item rank($A$) = $n$
		\item nullity($A$) = $0$
		\item The columns of $A$ are linearly independent.
		\item The columns of $A$ span \Rn.
		\item The columns of $A$ form a basis for \Rn.
		\item The rows of $A$ are linearly independent.
		\item The rows of $A$ span \Rn.
		\item The rows of $A$ form a basis for \Rn.
	\end{enumerate}
\end{theorem}

\begin{proof}
	Let $A$ be an $n \times n$ matrix.
	
	\noindent (a $\Rightarrow$ b) Theorem 3.7
	
	\noindent \\ (b $\Rightarrow$ c) Since $\textbf{x} = \textbf{0}$ is a solution of $A\textbf{x} = \textbf{0}$, $\textbf{0}$ is the unique solution of the system. Therefore, the system $A\textbf{x} = \textbf{0}$ has only the trivial solution.
	
	\noindent \\ (c $\Rightarrow$ d) The corresponding augmented martrix of the homogeneous system with unique solution $\textbf{0}$ is $[I_n | \textbf{0}]$. Since $A\textbf{x}=\textbf{0}$ has only the unique solution $\textbf{0}$, $A$ is row equivalent with $I_n$, therefore RREF of $A$ is $I_n$.
	
	\noindent \\ (d $\Rightarrow$ e)
	Since $A$ and $I_n$ are row equivalent, there exists a sequence of elementary row operations which converts $I_n$ to $A$. Let $E_1, E_2, \cdots, E_k$ be the corresponding elementary matrices to each steps fo elementary row operations. Then by Theorem 3.10, \begin{align*}
	A = E_{k}\cdots E_2E_1I_n = E_{k}\cdots E_2E_1
	\end{align*} Therefore $A$ is a product of elementary matrices.
	
	\noindent \\ (e $\Rightarrow$ a)
	Since all elementary matrices are invertible, $A$, which is the product of elementary matrices, is also invertible.
	
	\noindent \\ (d $\Leftrightarrow$ f)
	If the RREF of $A$ is $I_n$, since rank($I_n$) = $n$, by Theorem 3.16 rank($A$) = rank($I_n$) = $n$. Conversely, if rank($A$) = $n$, the RREF of $A$ has $n$ leading 1s so the RREF should be $I_n$.
	
	\noindent \\ (f $\Leftrightarrow$ g)
	Rank Theorem.
	
	\noindent \\ (c $\Leftrightarrow$ h)
	Let $A$ be an $n \times m$ matrix. Since $A\textbf{x} = \textbf{0}$ has only the trivial solution,  \begin{align*}
	c_1\textbf{A}^C_1 + \cdots + c_m\textbf{A}^C_m = \textbf{0}
	\end{align*} is true only if $c_1 = \cdots = c_m = 0$. Therefore, columns of $A$ are linearly independent.
	
	\noindent Conversely, if columns of $A$ are linearly independent, the system $c_1\textbf{A}^C_1 + \cdots + c_m\textbf{A}^C_m = \textbf{0}$ does not have a nontrivial solution, so $A\textbf{x} = \textbf{0}$ has only the trivial solution. 
	
	\noindent \\ (b $\Rightarrow$ i)
	Since for any $\textbf{b} \in$ \Rn, $A\textbf{x} = \textbf{b}$ is consistent, $\textbf{b}$ is a linear combination of colmuns of $A$. Therefore, the columns of $A$ span \Rn.
	
	\noindent \\ (i $\Rightarrow$ f)
	col($A$) = \Rn, so rank($A$) = dim col($A$) = dim \Rn = $n$.
	
	\noindent \\ (h and i $\Leftrightarrow$ j)
	Definition of basis for a subspace of \Rn.
	
	\noindent \\ By Theorem 3.21, rank($A^T$) = rank($A$). Therefore, the proposition related to columns of $A$ can be applied to rows of $A$, so (k), (l), and (m) are also equivalent.
	
\end{proof}

\begin{theorem}
	Let $A$ be an $n \times m$ matrix. Then \begin{enumerate}
		\item rank($A^TA$) = rank($A$)
		\item The $n \times n$ matrix $A^TA$ is invertible if and only if rank($A$) = $n$.
	\end{enumerate}
\end{theorem}

\begin{proof}
	\begin{enumerate}
		\item Let $A$ be an $n \times m$ matrix.
		
		\noindent (i) Suppose $\textbf{x}$ is a solution for $A\textbf{x} = \textbf{0}$. Then $A^TA\textbf{x} = \textbf{0}$. Therefore null($A$) $\subset$ null($A^TA$).
		
		\noindent (ii) Suppose $\textbf{x}$ is a solution for $A^TA\textbf{x} = \textbf{0}$. Then \begin{align*}
			(A\textbf{x}) \cdot (A\textbf{x}) = (A\textbf{x})^T(A\textbf{x}) = \textbf{x}^TA^TA\textbf{x} = \textbf{x}^T\textbf{0} = 0
		\end{align*} Therefore, $A\textbf{x} = \textbf{0}$, so null($A^TA$) $\subset$ null($A$).
		
		\noindent By (i), (ii), null($A$) = null($A^TA$), so nullity($A$) = nullity($A^TA$), thus rank($A$) = rank($A^TA$) by Rank Theorem.
		
		\item Let $A$ be an $n \times n$ matrix. Since rank($A$) = rank($A^TA$), by F.T.I.M, $A^TA$ is invertible if and only if rank($A^TA$) = rank($A$) = $n$.
		
	\end{enumerate}
\end{proof}

\begin{theorem}
	Let $S$ be a subspace of \Rn and let $\mathcal{B} = \{$ \vk $\}$ be a basis for $S$. For every vector $\textbf{v}$ in $S$, there is exactly one way to represent $\textbf{v}$ as a linear combination of the basis vectors in $\mathcal{B}$.
\end{theorem}

\begin{proof}
	(Existence) Since $\mathcal{B}$ is a basis of $S$, any vector $\textbf{v} \in S$ is a linear combination of vectors in $\mathcal{B}$.
	
	\noindent  \\ (Uniqueness) Suppose that there are more than two ways to represent $\textbf{v} \in S$ as a linear combination of $\mathcal{B} = \{$\vk $\}$. That is, the linear system with augmented matrix [$A \vert \textbf{v}$] = $\begin{bmatrix}[ccc|c]
		\textbf{v}_1 & \cdots & \textbf{v}_k & \textbf{v}
	\end{bmatrix}$ has more than two solutions. Let those two solutions be $\textbf{x}_1$ and $\textbf{x}_2$, then $A\textbf{x}_1 = \textbf{v}$ and $A\textbf{x}_2 = \textbf{v}_2$. Since $A(\textbf{x}_1 - \textbf{x}_2) = \textbf{0}$, the homogeneous system $A\textbf{x} = \textbf{0}$ has a nontrivial solution, which contradicts with F.T.I.M. Therefore, representation of $\textbf{v}$ as a linear combination of $\mathcal{B}$ is unique.
\end{proof}

\noindent \textit{Definition.} Let $S$ be a subspace of \Rn and let $\mathcal{B}$  = \{\vk\} be a basis for $S$. Let $\textbf{v}$ be a vector in $S$, and $\textbf{v} = c_1\textbf{v}_1 + c_2\textbf{v}_2 + \cdots + c_k\textbf{v}_k$. Then $c_1, c_2, \cdots, c_k$ are the \textbf{coordinates of $\textbf{v}$ with respect to $\mathcal{B}$}, and the column vector \begin{align*}
[\textbf{v}]_\mathcal{B} = \begin{bmatrix}
c_1 \\ c_2 \\ \vdots \\ c_k
\end{bmatrix}
\end{align*} is called the \textbf{coordinate vector of $\textbf{v}$ with respect ot $\mathcal{B}$}.

\begin{plaintheorem}[Inequality on subspaces' dimension]
	\textit{Note.} This lemma will be used in this subsection as `Lemma'. \\
	Let $ S_{1} $, $ S_{2} $ be subspaces of $ \mathbb{R}^{n} $. If $ S_{1} \subset S_{2} $, then $ \text{dim}(S_{1}) \le \text{dim}(S_{2})  $
\end{plaintheorem}
\begin{proof}
	Let $ \mathcal{B}_{1} = \{\textbf{u}_1, \textbf{u}_2, \cdots \textbf{u}_k\} $ be a basis for $ S_{1} $, $ \mathcal{B}_{2} = \{\textbf{v}_1, \textbf{v}_2, \cdots \textbf{v}_m\} $ be a basis for $ S_{2} $ . 
	Suppose $ k>m $. Since $ \mathcal{B} \subset S_{1} \subset S_{2} $, there exist $ a_{ij} $ ($ 1\leq i \leq k $, $ 1 \leq j \leq m $) such that 
	\begin{gather*}
	\textbf{u}_1 = a_{11}\textbf{v}_1+ \cdots + a_{1m}\textbf{v}_m \\
	\vdots \\
	\textbf{u}_k = a_{k1}\textbf{v}_1+ \cdots + a_{km}\textbf{v}_m \\
	\end{gather*}
	Then $ c_1\textbf{u}_1+c_2\textbf{u}_2+\cdots c_k\textbf{u}_k = \left(c_{1}a_{11}+\cdots + c_{k}a_{k1}\right)\textbf{v}_1 + \cdots \left(c_{1}a_{1m}+\cdots + c_{k}a_{km}\right)\textbf{v}_m = \textbf{0}$ has only the trivial solution $ c_{1}a_{11}+\cdots + c_{k}a_{k1} \cdots c_{1}a_{1m}+\cdots + c_{k}a_{km} = 0$ since $ \mathcal{B}_{2} $ is linearly independent.
	
	$\begin{bmatrix}[ccc|c]
	a_{11} & \cdots & a_{k1} & 0 \\
	\vdots & & \vdots & \vdots \\
	a_{1m} & \cdots & a_{km} & 0 \\
	\end{bmatrix}$ has a nontrivial solution since 
	$ \text{rank}\left(\begin{bmatrix}[ccc|c]
	a_{11} & \cdots & a_{k1} & 0 \\
	\vdots & & \vdots & \vdots \\
	a_{1m} & \cdots & a_{km} & 0 \\
	\end{bmatrix}\right) \leq m < k$. 
	The number of free variables is $ k-\text{rank}\left(\begin{bmatrix}[ccc|c]
	a_{11} & \cdots & a_{k1} & 0 \\
	\vdots & & \vdots & \vdots \\
	a_{1m} & \cdots & a_{km} & 0 \\
	\end{bmatrix}\right) > 0 $. \\
	Therefore, there exist scalars $ c_{1}, \cdots, c_{k} $ such that at least one of $ c_{1}, \cdots, c_{k} $ are nonzero and $ c_{1}\textbf{u}_{1} + \cdots + c_{k}\textbf{u}_{k} = \textbf{0}$, so $ \textbf{u}_{1}, \cdots , \textbf{u}_{k} $ are linearly dependent. \\
	However, since $ \mathcal{B}_{1} $ is a basis, $ \textbf{u}_{1}, \cdots , \textbf{u}_{k} $ should be linearly independent : contradiction.
	$ \therefore k\leq m $, that is, $ \text{dim} S_{1} \leq \text{dim} S_{2} $.
\end{proof}

\section{Solutions of Exercises Worthy to Solve}

% Typed by 14041 박승원

\begin{enumerate}
	\item \textbf{Exercise 3.5 57}
	\begin{proof}
		Let $ A $ be an $ m\times n $ matrix and let $ R $ be the RREF of A. Then for any $ \textbf{v} \in $ null($A$) , 
		$ R\textbf{v} = \begin{bmatrix}
		\textbf{R}^R_1 \cdot\textbf{v} \\
		\vdots \\
		\textbf{R}^R_n \cdot\textbf{v} \\
		\end{bmatrix} = \textbf{0}$. $ \therefore \textbf{R}^R_i\cdot\textbf{v}=\textbf{0} $ for $ 1\leq i \leq n $.
		For every $ \textbf{n} \in $ row($A$) , $ \exists c_1,c_2, \cdots c_n $ such that $ \textbf{u}=c_1\textbf{R}^R_1 + c_2\textbf{R}^R_2 + \cdots c_n\textbf{R}^R_n $ 
		since $ \{\textbf{R}^R_1, \cdots \textbf{R}^R_n \} $ form a basis for row($A$), 
		then 
		\begin{align*}
		\textbf{u}\cdot\textbf{v} &= \left(c_1\textbf{R}^R_1 + c_2\textbf{R}^R_2 + \cdots c_n\textbf{R}^R_n\right)\cdot \textbf{v}  \\
		&= c_1\textbf{R}^R_1\cdot\textbf{v} + c_2\textbf{R}^R_2\cdot\textbf{v} + \cdots c_n\textbf{R}^R_n\cdot\textbf{v} = 0\\
		\end{align*}
		Therefore, for any $ \textbf{u} \in $ row($A$) and $ \textbf{v} \in $ null($A$), $ \textbf{u}\cdot \textbf{v}=0 $ so $ \textbf{u} $ and $ \textbf{v} $ are orthogonal.
	\end{proof}
	\item \textbf{Exercise 3.5 58}
	\begin{proof}
		Let $A$, $ B $ be $ n\times n $ matrices with rank $ n $.
		Then by F.T.I.M, $ A\textbf{x}=\textbf{0} $ and $ B\textbf{x}=\textbf{0} $ has only the trivial solution, and $ \exists \inv{A} $, $ \exists \inv{B} $. \\
		Suppose that $ AB\textbf{x}=\textbf{0} $ has a nontrivial solution $ \textbf{x}_1\neq \textbf{0} $. Then $ AB\textbf{x}_1=\textbf{0} $, so $ \inv{A}AB\textbf{x}_1=B\textbf{x}_1=\textbf{0} $, so the system $ B\textbf{x}=\textbf{0} $ has a nontrivial solution. Since this contradict with our assumption, $ (AB)\textbf{x}=\textbf{0} $ has only the trivial solution. \\
		By F.T.I.M, rank of $ AB $ is $ n $.
	\end{proof}
	Another solution : 
	\begin{proof}
		By F.T.I.M, $ \exists \inv{A}$ and $ \exists \inv{B}$. Therefore, $ \inv{AB}=\inv{B}\inv{A} $. By F.T.I.M, rank($ AB $) $ =n $.
	\end{proof}
	\item \textbf{Exercise 3.5 59}
	\begin{proof}
		\noindent \textbf{(a)} For $ n\times m $ matrix $ A $ and $ m\times r $ matrix $ B $,\\ $ AB = \begin{bmatrix}
		\textbf{A}^R_1 \\
		\vdots \\
		\textbf{A}^R_n \\
		\end{bmatrix}B = \begin{bmatrix}
		\textbf{A}^R_1 B \\
		\vdots \\
		\textbf{A}^R_n B \\
		\end{bmatrix} = \begin{bmatrix}
		A_{11}\textbf{B}^R_1 + \cdots + A_{1m}\textbf{B}^R_m
		\vdots
		A_{n1}\textbf{B}^R_1 + \cdots + A_{nm}\textbf{B}^R_m
		\end{bmatrix}$. \\
		Since rows of $ AB $ are linear combinations of $ \textbf{B}^R_1, \cdots , \textbf{B}^R_m $, $ \text{row}(AB) \subset \text{row}(B) $. \\
		$ \therefore $ By Lemma,$ \text{rank}(AB) \leq \text{rank}(B) $.
		\noindent \textbf{(b)} In case when $ A=O $.
	\end{proof}
	Another solution of \textbf{(a)} : 
	\begin{proof}
		For $ n\times m $ matrix $ A $ and $ m\times r $ matrix $ B $,
		\begin{align*}
		\text{rank}(B) &= r - \text{nullity}(B) \\
		\text{rank}(AB) &= r - \text{nullity}(AB) \\
		\end{align*}
		$ \forall \textbf{x}\in \text{null}(B) $, $ (AB)\textbf{x}=A(B\textbf{x})=A\textbf{0}=\textbf{0} $. \textit{i.e.} $ \textbf{x}\in \text{null}(AB) $. $ \therefore \text{null}(B) \subset \text{null}(AB) $. \\
		By Lemma, $ \text{nullity}(B)\leq \text{nullity}(AB)$. $ \therefore \text{rank}(B) \geq \text{rank}(AB) $.
	\end{proof}
	\item \textbf{Exercise 3.5 60}
	\begin{proof}
		\noindent \textbf{(a)}
		For $ n\times m $ matrix $ A $ and $ m\times r $ matrix $ B $, 
		$ AB=A\begin{bmatrix}
		\textbf{B}^C_1 & \textbf{B}^C_2 & \cdots & \textbf{B}^C_r
		\end{bmatrix} = \begin{bmatrix}
		A\textbf{B}^C_1 & A\textbf{B}^C_2 & \cdots & A\textbf{B}^C_r
		\end{bmatrix} \begin{bmatrix}
		\left(B_{11}\textbf{A}^C_1 + \cdots + B_{m1}\textbf{A}^C_m \right) & \cdots & \left(B_{1r}\textbf{A}^C_1 + \cdots + B_{mr}\textbf{A}^C_m \right)
		\end{bmatrix}$
		Since columns of $ AB $ are linear combinations of $ \textbf{A}^C_1 \cdots \textbf{A}^C_m$, $ \text{col}(AB) \subset \text{col}(A) $. \\
		$ \therefore $ By Lemma, $ \text{rank}(AB) \leq \text{rank}(B) $.
		\noindent \textbf{(b)}
		In case when $ B=O $
	\end{proof}
	Another solution of \textbf{(a)} :
	\begin{proof}
		\begin{align*}
		\text{rank}(AB) &= \text{rank}(B^{T}A^{T}) && \text{(Theorem 3.21)} \\
		&\leq \text{rank}(A^{T}) && \text{(Exercise 3.5 59)} \\
		&= \text{rank}(A) && \text{(Theorem 3.21)} \\
		\end{align*}
	\end{proof}
	\item \textbf{Exercise 3.5 61}
	\begin{proof}
		\noindent \textbf{(a)} 
		By Exercise 3.5 59, $ \text{rank}(A) = \text{rank}\left(\inv{U}\left(UA\right)\right) \leq \text{rank}(UA) \leq \text{rank}(A) $. $ \therefore \text{rank}(UA) = \text{rank}(A) $. 
		
		\noindent \textbf{(b)}
		By Exercise 3.5 60, $ \text{rank}(A) = \text{rank}\left(\left(AV\right)\inv{V}\right) \leq \text{rank}(AV) \leq \text{rank}(A) $. $ \therefore \text{rank}(AV) = \text{rank}(A) $.
	\end{proof}
	\item \textbf{Exercise 3.5 62}
	\begin{proof}
		\noindent ($\Rightarrow$)
		Let $ A $ be an $ m\times n $ matrix with rank 1. Then $ \text{dim}\left(\text{row}(A)\right) = \text{dim}\left(\text{col}(A)\right) = 1$. 
		For $ \textbf{v} \in \mathbb{R}^m$, suppose that $ \{\textbf{v}^T\} $ is a basis for $ \text{row}(A) $. Then $ \exists c_{1}, c_{2}, \cdots , c_{m} $ such that $ \textbf{A}^R_i = c_{i}\textbf{u}^T $. 
		Then $ A = \begin{bmatrix}
		\textbf{A}^R_1 \\
		\vdots \\
		\textbf{A}^R_m \\
		\end{bmatrix} = \begin{bmatrix}
		c_{1}\textbf{v}^T \\
		\vdots \\
		c_{m}\textbf{v}^T \\
		\end{bmatrix} = \begin{bmatrix}
		c_{1} \\
		\vdots \\
		c_{m} \\
		\end{bmatrix}\textbf{v}^T$. Let $ \textbf{u}=\begin{bmatrix}
		c_{1} \\
		\vdots \\
		c_{m} \\
		\end{bmatrix} $, then $ \textbf{u}\textbf{v}^T = A $
		
		\noindent ($\Leftarrow$)
		Suppose that for $ \textbf{u}=\begin{bmatrix}
		u_1 \\
		\vdots \\
		u_m \\
		\end{bmatrix} \in \mathbb{R}^m $ and 
		$ \textbf{v}=\begin{bmatrix}
		v_1 \\
		\vdots \\
		v_m \\
		\end{bmatrix} \in \mathbb{R}^m $, $ \textbf{u}\textbf{v}^T = A$.
		\footnote{사실 문제가 틀림. Nonzero vectors $ \textbf{u} $, $ \textbf{v} $ 여야 함.} \\
		$ A=\textbf{u}\textbf{v}^T = \textbf{u}\begin{bmatrix}
		v_1 & \cdots & v_n \\
		\end{bmatrix} = \begin{bmatrix}
		v_{1}\textbf{u} & \cdots & v_{n}\textbf{u} \\
		\end{bmatrix} $ \\
		Then $ \{\textbf{u}\} $ forms a basis for $ \text{col}(A) $, since the columns of $ A $ are linear combination of $ \textbf{u} $. \\
		$ \therefore \text{dim}\left(\text{col}(A)\right) = \text{rank}(A) = 1$
	\end{proof}
	\item \textbf{Exercise 3.5 63}
	\begin{proof}
		Let $ A $ be an $ m\times n $ matrix. Then $ A=AI_{n} = \begin{bmatrix}
		\textbf{A}^C_1 & \cdots & \textbf{A}^C_n
		\end{bmatrix} \begin{bmatrix}
		\textbf{e}_1 \\
		\vdots \\
		\textbf{e}_n \\
		\end{bmatrix} = \textbf{A}^C_1\textbf{e}_1 + \cdots + \textbf{A}^C_n\textbf{e}_n $. Since $ \textbf{A}^C_1 \in \mathbb{R}^m $ and $ \textbf{e}_i\in \mathbb{R}^n $, ranks of $ \textbf{A}^C_i\textbf{e}_i $ are all 1.
	\end{proof}
	\item \textbf{Exercise 3.5 64}
	\begin{proof}
		For $ m\times n $ matrices $ A,B $, let $ \mathcal{B}_1 $, $ \mathcal{B}_2 $ be bases for $ \text{row}(A) $,$ \text{row}(B) $. Then $ \left(\textbf{A+B}\right)^R_i = \textbf{A}^R_i + \textbf{B}^R_i $, so $ \left(\textbf{A+B}\right)^R_i $ are linear combinations of vectors in $ \mathcal{B}_1 \cup  \mathcal{B}_2$. \\
		Thus, there exists a subset $ C \subset B $ which forms a basis for $ \text{row}(A+B) $. \\
		$ \therefore \text{rank}(A+B)=\left|C\right| \leq \left|\mathcal{B}_1 \cup \mathcal{B}_2 \right| \leq \left|\mathcal{B}_1\right| + \left|\mathcal{B}_2\right| = \text{rank}(A)+\text{rank}(B) $
	\end{proof}
	\item \textbf{Exercise 3.5 65}
	\begin{proof}
		$ AA = A\begin{bmatrix}
		\textbf{A}^C_1 \cdots \textbf{A}^C_n
		\end{bmatrix} = \begin{bmatrix}
		A\textbf{A}^C_1 \cdots A\textbf{A}^C_n
		\end{bmatrix} = 0$. $ \therefore \textbf{A}^C_1 \cdots \textbf{A}^C_n $ are solutions of the system $ A\textbf{x}=\textbf{0} $. $ \textbf{A}^C_1 \cdots \textbf{A}^C_n \in \text{null}(A) $. and by definition of subspaces, $ \text{col}(A) \subset \text{null}(A) $. \\
		$ \therefore $ By Lemma, $ \text{dim}\left(col(A)\right) = \text{rank}(A) \leq \text{null}(A) $. $ \therefore \text{null}(A) \geq n/2 $
	\end{proof}
	\item \textbf{Exercise 3.5 66}
	\begin{proof}
		\noindent \textbf{(a)}
		$ \left(\textbf{x}^TA\textbf{x}\right)^T = \textbf{x}^T A\textbf{x} = -\textbf{x}^T A \textbf{x} $. $ \therefore $ $ \textbf{x}^T A \textbf{x} $ is also skew-symmetric. Since $ \textbf{x}^T A \textbf{x} $ is a $ 1\times 1 $ matrix and all main diagonal entries are zero in skew-symmetric matrices, $ \textbf{x}^T A \textbf{x} $ = 0.
		\noindent \textbf{(b)}
		Let $ \textbf{x}\in \mathbb{R}^n $ such that $ \left(I_n+A\right)\textbf{x}=\textbf{0} $. Then $ \textbf{x}^T\left(I_n+A\right)\textbf{x} = \textbf{x}^T I_n \textbf{x} + \textbf{x}^T A \textbf{x} = \textbf{x}^T\textbf{x}+0 = \textbf{x}^T \textbf{x} = 0$. Since $ \textbf{x}^T\textbf{x}=0 $, $ \textbf{x}=\textbf{0} $. \\
		$ \therefore $ $ \text{null}(I_n+A)=\{\textbf{0}\} $. Since $ \text{rank}(I_n+A)=n $, by F.T.I.M, $ A+I_n $ is invertible.
	\end{proof}
\end{enumerate}
%\include{Ch4_eigenvalues_and_eigenvectors}
\appendix
\chapter{Cautions on Exam}
\section{Notations}
\begin{table}[h]
	\centering
	\begin{tabular}{|c|c|c|}
		\hline
		Wrong & Right & Explanation \\
		\hline
		$ 2 \cdot 3 = 6 $ & $ 2 \times 3 = 6 $ & \multirow{2}{*}{$ \cdot $ for dot product is only valid for vectors.} \\
		\cline{1-2}
		$ 2 \cdot \textbf{v}$ & $2\textbf{v} $ & \\
		\hline
		$\begin{bmatrix} 1 & 3 & 4 \end{bmatrix}$ & $\begin{bmatrix} 1, & 3, & 4 \end{bmatrix}$ & When writing row vectors, commas are necessary. \\
		\hline
		
	\end{tabular}
\end{table}
\section{Description}
\begin{itemize}
\item Free variables like $s$, $t$ and $u$ must be indicated that they are arbitrary real number. \textit{e.g.} $ s,t \in \mathbb{R} $
%\item Whenever you use FTIM, always start by writing `according to fundamental theorem of invertible matrices' or `가역행렬의 기본정리에 의해'.
\item When proving $ \text{span}\left(\textbf{v}_1,\textbf{v}_2,\textbf{v}_3\right) = \mathbb{R}^{3} $, you must prove each side, not only one.
\item $ A=B $ means not only same entries, but also same size.
%\item Do not write reduced row echelon form as `ref'. This abbreviation is not defined in our book.
\item Write as `적어도 하나는 0이 아닌' instead of `모두 0은 아닌' or else.
\item Abbreviation : only 4 things are allowed : `REF', `RREF', 'EMO', 'F.T.I.M.'. Each of them stands for (Reduced) Row Echelon Form, Elementary Matrix Operation, Fundamental Theorems of Invertible Matrices.
\end{itemize}

\end{document}
