% !TeX program = xelatex
%% 부득이하게 pdflatex을 사용해야 할 경우 위의 magic comment를 제거하십시오.

%%%%%%%%%%%%%%%%%%%%%%%%%%%%%%%%%%%%%%%%%%%%%%%%%%%%%%%%%%%%%%%%%%%%%%%%%%%%%%%%%
%%%  LaTeX document class of the thesis of the Gyeonggi Science High School   %%%
%%%  Last edition 2021.02.19 by Junu Kwon                                     %%%
%%%  Continously being modified by gshslatexintro after 2016.02.02.           %%%
%%%  Check the latest version at : latex.gs.hs.kr                             %%%
%%%  Also refer to https://www.facebook.com/gshstexsociety                    %%%
%%%%%%%%%%%%%%%%%%%%%%%%%%%%%%%%%%%%%%%%%%%%%%%%%%%%%%%%%%%%%%%%%%%%%%%%%%%%%%%%%

\documentclass{gshs_thesis}
% 이곳에 필요한 별도의 패키지들을 적어넣으시오.
%\usepackage{...}
\usepackage{verbatim} % for commment, verbatim environment
\usepackage{spverbatim} % automatic linebreak verbatim environment
\usepackage{listings}
\lstset{
	basicstyle=\small\ttfamily,
	columns=flexible,
	breaklines=true
}

% -----------------------------------------------------------------------
%                   이 부분은 수정하지 마시오.
% -----------------------------------------------------------------------
\titleheader{졸업논문청구논문}
\school{과학영재학교 경기과학고등학교}
\approval{위 논문은 과학영재학교 경기과학고등학교 졸업논문으로\\
졸업논문심사위원회에서 심사 통과하였음.}
\chairperson{심사위원장}
\examiner{심사위원}
\apprvsign{(인)}
\korabstract{초 록}
\koracknowledgement{감사의 글}
\korresearches{연 구 활 동}

%: ----------------------------------------------------------------------
%:                  논문 제목과 저자 이름을 입력하시오
% ----------------------------------------------------------------------
\title{생체모방 초발수성 표면의 나노구조에 따른 발수성의 정량적 분석} %한글 제목
\engtitle{Quantitative analysis of the relationship between biomimetic Super-hydrophobic Surface's nano-structure and it's water repellency} %영문 제목
\korname{홍 길 동} %저자 이름을 한글로 입력하시오 (글자 사이 띄어쓰기)
\engname{Hong, Gil-Dong} %저자 이름을 영어로 입력하시오 (family name, personal name)
\chnname{洪 吉 東} %저자 이름을 한자로 입력하시오 (글자 사이 띄어쓰기)
\studid{21000} %학번을 입력하시오

%------------------------------------------------------------------------
%                  심사위원과 논문 승인 날짜를 입력하시오
%------------------------------------------------------------------------
\advisor{Mok, Chinook}  %지도교사 영문 이름 (family name, personal name)
\judgeone{박 교 수} %심사위원장
\judgetwo{김 대 감}   %심사위원1
\judgethree{목 진 욱} %심사위원2(지도교사)
\degreeyear{2022}   %졸업 년도
%\degreedate{2015}{11}{13} %논문 승인 날짜 양식
\degreedate{2023. 6. 30} %논문 승인 날짜 양식1
\degreedatekor{2023년 6월 30일} %논문 승인 날짜 양식2

%------------------------------------------------------------------------
%                  논문제출 전 체크리스트를 확인하시오
%------------------------------------------------------------------------
\checklisttitle{[논문제출 전 체크리스트]} %수정하지 마시오
\checklistI{1. 이 논문은 내가 직접 연구하고 작성한 것이다.} %수정하지 마시오
% 이 항목이 사실이라면 다음 줄 앞에 "%"기호 삽입, 다다음 줄 앞의 "%"기호 제거하시오
\checklistmarkI{$\square$}
%\checklistmarkI{$\text{\rlap{$\checkmark$}}\square$}
\checklistII{2. 인용한 모든 자료(책, 논문, 인터넷자료 등)의 인용표시를 바르게 하였다.} %수정하지 마시오
% 이 항목이 사실이라면 다음 줄 앞에 "%"기호 삽입, 다다음 줄 앞의 "%"기호 제거하시오
\checklistmarkII{$\square$}
%\checklistmarkII{$\text{\rlap{$\checkmark$}}\square$}
\checklistIII{3. 인용한 자료의 표현이나 내용을 왜곡하지 않았다.} %수정하지마시오
% 이 항목이 사실이라면 다음 줄 앞에 "%"기호 삽입, 다다음 줄 앞의 "%"기호 제거하시오
\checklistmarkIII{$\square$}
%\checklistmarkIII{$\text{\rlap{$\checkmark$}}\square$}
\checklistIV{4. 정확한 출처제시 없이 다른 사람의 글이나 아이디어를 가져오지 않았다.} %수정하지 마시오
% 이 항목이 사실이라면 다음 줄 앞에 "%"기호 삽입, 다다음 줄 앞의 "%"기호 제거하시오
\checklistmarkIV{$\square$}
%\checklistmarkIV{$\text{\rlap{$\checkmark$}}\square$}
\checklistV{5. 논문 작성 중 도표나 데이터를 조작(위조 혹은 변조)하지 않았다.} %수정하지 마시오
% 이 항목이 사실이라면 다음 줄 앞에 "%"기호 삽입, 다다음 줄 앞의 "%"기호 제거하시오
\checklistmarkV{$\square$}
%\checklistmarkV{$\text{\rlap{$\checkmark$}}\square$}
\checklistVI{6. 다른 친구와 같은 내용의 논문을 제출하지 않았다.} %수정하지 마시오
% 이 항목이 사실이라면 다음 줄 앞에 "%"기호 삽입, 다다음 줄 앞의 "%"기호 제거하시오
\checklistmarkVI{$\square$}
%\checklistmarkVI{$\text{\rlap{$\checkmark$}}\square$}

\usepackage{tocloft}
\setlength{\cftbeforesecskip}{0pt}
\setlength{\cftbeforesubsecskip}{0pt}
\setlength{\cftbeforesubsubsecskip}{0pt}
\begin{document}
%\renewcommand\baselinestretch{1.2} % line spacing in the paragraph
%\baselineskip=22pt plus1pt         % line spacing in the paragraph
\baselineskip=2.2em         % line spacing in the paragraph

\maketitle  % command to print the title page with above variables
\setcounter{page}{1}
%---------------------------------------------------------------------
%                  영문 초록을 입력하시오
%---------------------------------------------------------------------
\begin{abstracts}     %this creates the heading for the abstract page
	\addcontentsline{toc}{section}{Abstract}  %%% TOC에 표시
	\noindent{
		Put your abstract here. It is completely consistent with 한글초록.
	}
\end{abstracts}

%---------------------------------------------------------------------
%                  국문 초록을 입력하시오
%---------------------------------------------------------------------
\begin{abstractskor}        %this creates the heading for the abstract page
	\addcontentsline{toc}{section}{초록}  %%% TOC에 표시
	\noindent{
		초록(요약문)은 가장 마지막에 작성한다. 연구한 내용, 즉 본론부터 요약한다. 서론 요약은 하지 않는다. 대개 첫 문장은 연구 주제 (+방법을 핵심적으로 나타낼 수 있는 문구: 실험적으로, 이론적으로, 시뮬레이션을 통해)를 쓴다. 다음으로 연구 방법을 요약한다. 선행 연구들과 구별되는 특징을 중심으로 쓴다. 뚜렷한 특징이 없다면 연구방법은 안써도 상관없다. 다음으로 연구 결과를 쓴다. 연구 결과는 추론을 담지 않고, 객관적으로 서술한다. 마지막으로 결론을 쓴다. 이 연구를 통해 주장하고자 하는 바를 간략히 쓴다. 요약문 전체에서 연구 결과와 결론이 차지하는 비율이 절반이 넘도록 한다. 읽는 이가 요약문으로부터 얻으려는 정보는 연구 결과와 결론이기 때문이다. 연구 결과만 레포트하는 논문인 경우, 결론을 쓰지 않는 경우도 있다.
	}
\end{abstractskor}


%----------------------------------------------
%   Table of Contents (자동 작성됨)
%----------------------------------------------
\cleardoublepage
\addcontentsline{toc}{section}{Contents}
\setcounter{secnumdepth}{3} % organisational level that receives a numbers
\setcounter{tocdepth}{3}    % print table of contents for level 3
\baselineskip=2.2em
\tableofcontents


%----------------------------------------------
%     List of Figures/Tables (자동 작성됨)
%----------------------------------------------
\cleardoublepage
\clearpage
\listoftables  % 표 목록과 캡션을 출력한다. 만약 논문에 표가 없다면 이 줄의 맨 앞에 %기호를 넣어서 코멘트 처리한다.

\cleardoublepage
\clearpage
\listoffigures	% 그림 목록과 캡션을 출력한다. 만약 논문에 그림이 없다면 이 줄의 맨 앞에 %기호를 넣어서 코멘트 처리한다.


%%%%%%%%%%%%%%%%%%%%%%%%%%%%%%%%%%%%%%%%%%%%%%%%%%%%%%%%%%%
%%%% Main Document %%%%%%%%%%%%%%%%%%%%%%%%%%%%%%%%%%%%%%%%
%%%%%%%%%%%%%%%%%%%%%%%%%%%%%%%%%%%%%%%%%%%%%%%%%%%%%%%%%%%
\cleardoublepage
\clearpage
\renewcommand{\thepage}{\arabic{page}}
\setcounter{page}{1}

%% Images and Plots

\graphicspath{ {images/} }

%-----------------------------------------------------
%  Introduction
%-----------------------------------------------------

\section{Introduction}

서론은 연구를 진행하게 된 배경을 기술하는 곳으로 보통 다음과 같은 순서로 쓰는 편이다.
\begin{itemize}
\item{연구 주제의 전반적 관심을 조명.}
\item{연구 분야의 스페셜 이슈를 조명.}
\item{해당 이슈를 해결하기 위한 다양한 선행 연구들을 서술.}
\item{선행 연구들의 한계점을 기술.}
\item{한계를 극복하기 위한 본 연구의 목적을 밝힘.}
\item{논문의 구성을 서술 (optional).}
\end{itemize}
서론은 과거부터 현재까지 해당 분야의 연구 진행을 기술하기 때문에 선행 연구 논문들을 레퍼런스로 도입하는 경우가 빈번하게 나타난다. \LaTeX에서 참고문헌을 표기하는 방법을 알아보자. 먼저 이 문서의 후반부에 위치한 레퍼런스 부분을 찾아간다. 이 문서를 컴파일했을 때 생성된 PDF 파일에는 {\bf References}라고 나와 있지만 여기서는 {\textbackslash}begin\{thebibliography\}\{99\}로 시작에서 {\textbackslash}end\{thebibliography\}로 종료되는 그 사이에 참고문헌을 작성하면 된다. 여기서 숫자 99는 참고문헌이 100개 넘는 논문을 작성하는 것이 아니라면 그대로 놔둔다. 참고문헌 작성 예시는 다음과 같다.
\begin{lstlisting}
\bibitem{Mok06}C. Mok, C.-M. Ryu, P. H. Yoon, and A. T. Y. Lui, (2006), ``Global twofluid stability of bifurcated current sheets'', J. Goephys. Res., {\bf 111}, A03203.
\end{lstlisting}
{\textbackslash}bibitem 다음의 \{ \} 안 참고문헌의 입력은 표준 APA 양식을 따른다. Author, Co-authors, (Publication Year), Article title, Journal title, Volume(Issue), pp.-pp. 순으로 작성한다. 저자는 family name, personal name 순으로 입력하며 family name 다음에 쉼표를 찍는다. 저널명은 이탤릭체로 표기하며 관용적으로 약어로 표기 가능하다. 단 약어는 저널에서 공식으로 인정된 표준 약어를 사용한다. 약어 뒤에는 마침표를 넣는다. 권 (volume)은 기울임체로 숫자만 입력한다. 호 (issue)와 출판년도는 괄호 안에 입력한다.

이제 서론에서 해당 논문을 인용할 준비 작업은 끝났다. 서론에서 필요한 부분에 이 논문을 인용 표기할 경우 {\textbackslash}cite 라고 입력한 후 \{ \} 안에 해당 논문을 표시하면 된다. 표시하는 방법은 바로 레퍼런스에서 {\textbackslash}bibitem 이후 \{ \} 안에 적었던 것을 넣어주면 된다. 논문 인용 표시가 문장 마지막에 등장할 때는 마침표의 위치는 인용 표시 다음이다. 아래 문장은 논문 인용 표시의 예로 C. Mok의 2010년 논문에서 인용하였다 \cite{Mok10}.
\begin{lstlisting}
Various plasma instabilities have been proposed as playing important roles during the substorm onset process. These include the tearing \cite{Schindler74, Sitnov97, Zelenyi08}, ballooning \cite{Cheng98, Bhattacharjee98, Dobias04, Zhu03, Saito08, Friedrich01}, lower hybrid drift \cite{Shinohara98, Yoon02, Mok06}, Kelvin--Helmholtz \cite{Rostoker84, Dovias06}, and the ion Weibel \cite{Yoon93, Sadovskii01} instabilities.
\end{lstlisting}
위와 같이 입력한 후 컴파일하면 pdf 파일에는 다음과 같이 나타날 것이다.
\begin{quote}
Various plasma instabilities have been proposed as playing important roles during the substorm onset process. These include the tearing \cite{Schindler74, Sitnov97, Zelenyi08}, ballooning \cite{Cheng98, Bhattacharjee98, Dobias04, Zhu03, Saito08, Friedrich01}, lower hybrid drift \cite{Shinohara98, Yoon02, Mok06}, Kelvin--Helmholtz \cite{Rostoker84, Dovias06}, and the ion Weibel \cite{Yoon93, Sadovskii01} instabilities.
\end{quote}
이 때 참고문헌이 번호 순서대로 나오도록 한다. 또한 세 개 이상의 문헌이 연속된 번호로 이어진 경우 자동으로 첫 번호와 마지막 번호가 hyphen으로 연결된 형태로 등장함을 확인할 수 있다.


참고문헌은 다음의 조건들을 만족해야 한다.
\begin{itemize}
\item{저자가 명시되어야 한다.}
\item{검증이 된 내용이어야 한다.}
\item{이미 출판되어 수정이 불가능해야 한다.}
\end{itemize}
전문 논문 저널에 수록된 논문들은 위 조건들을 만족하므로 되도록 논문을 참고문헌으로 삼도록 한다. 웹사이트는 위 조건들을 만족하지 못하므로 참고문헌으로 부적절하다. 또한 누구라도 책을 출판할 수 있으므로 전문 서적을 참고문헌으로 사용하는 경우에는 널리 받아들여지고 인정받는 서적만 사용해야 한다. 사실 전공 서적의 저자는 여러 연구 논문들을 참고로 하여 책을 집필하기 때문에 전공서적에도 참고 문헌(논문)이 명시되어 있다. 이 경우 전공 서적 대신에 책에서 지시하는 논문을 참고문헌으로 삼도록 한다.



%-----------------------------------------------------
% Next Section (e.g. Experiment, Linear theory, etc...) 
%-----------------------------------------------------
\section{Equations, Figures, and Tables}
연구 본문은 하위 절(subsection, subsubsection, ...)등이 등장하는 경우가 있다. 이 때는 {\textbackslash}subsection\{section name\}, {\textbackslash}subsubsection\{subsubsection name\}을 사용하여 하위 절을 구성하면 된다. 여기서는 수식 작성법, 그림 넣는 법, 표 만드는 법에 대해 살펴본다.

\subsection{Equations}

먼저 문장 속에서 수식을 사용하는 경우에는 \$ (수식) \$과 같이 처리하면 된다. 다음과 같이 작성한 후 컴파일을 하면
\begin{quote}
운동 에너지는 \$(1/2)mv$\wedge$2\$으로 표현된다. 여기서 \$m\$은 물체의 질량, \$v\$ 는 물체의 속력이다.
\end{quote}
pdf 파일에는 다음과 같이 나타난다.
\begin{quote}
운동 에너지는 $(1/2)mv^2$으로 표현된다. 여기서 $m$은 물체의 질량, $v$는 물체의 속력이다.
\end{quote}
위의 예시에서 보듯이 문장 속에서 수식을 사용할 때는 한 줄로 입력한다. 즉 분수 형태의 수식은 슬래쉬`/'로 대체하여 표현한다. 줄 간격을 일정하게 유지하기 위함이다.

다음은 수식이 문장 밖으로 나와 한 줄을 통째로 차지하는 경우이다. 수식을 작성하는 명령어는 다양하지만, 여기서는 equation, align에 대해서만 다룬다. 먼저 equation의 사용법을 예를 통해 확인해보자. 만약 다음과 같이 입력을 하고
\begin{lstlisting}
	\begin{equation}
	\int_V \nabla\cdot{\rm\bf F} dV = \oint_S {\rm\bf F}\cdot d{\rm\bf A}. \label{eq001}
	\end{equation}
\end{lstlisting}
컴파일을 하면 다음과 같은 식이 pdf파일에 나타나는 것을 확인할 수 있다.
\begin{equation}
\int_V \nabla\cdot{\rm\bf F} dV = \oint_S {\rm\bf F}\cdot d{\rm\bf A}. \label{eq001}
\end{equation}
문장 속에서 수식의 번호(라벨)을 호출하는 경우가 종종 있다. 이 때는 {\textbackslash}ref\{eq001\}을 사용하면 된다. 여기서 eq001은 수식에서 label 명령어 다음에 작성된 문구이다. 수식마다 각기 다른 라벨을 작성해야 하며 논문 저자가 기억하기 편한 것을 사용하면 된다. 서로 다른 수식에 중복된 라벨을 사용하면, 나중에 작성된 라벨의 수식 번호만 호출됨에 유의하라. 사용 예는 다음과 같다.
\begin{quote}
(작성 예) Divergence 이론은 임의의 벡터 필드 \${\textbackslash}rm{\textbackslash}bf F\$가 임의의 폐곡면 외부를 향하는 플럭스의 총량은 \${\textbackslash}nabla{\textbackslash}cdot\{{\textbackslash}rm{\textbackslash}bf F\}\$ 를 폐곡면 내부 부피에 대하여 적분한 것과 같다는 것으로 이를 식으로 표현하면 식 ({\textbackslash}ref\{eq001\}) 과 같다.
\end{quote}
\begin{quote}
(컴파일 결과) Divergence 이론은 임의의 벡터 필드 $\rm\bf F$가 임의의 폐곡면 외부를 향하는 플럭스의 총량은 $\nabla\cdot{\rm\bf F}$를 폐곡면 내부 부피에 대하여 적분한 것과 같다는 것으로 이를 식으로 표현하면 식 (\ref{eq001})과 같다.
\end{quote}
그런데 수식에 번호를 붙일 필요가 없는 경우도 있다. 이 경우에는 수식이 끝나고 equation 명령어를 닫기 전 {\textbackslash}nonumber라고 입력하면 된다.
\begin{lstlisting}
	\begin{equation}
	\pi=3.14159265358979... \nonumber
	\end{equation}
\end{lstlisting}
\begin{equation}
\pi=3.14159265358979... \nonumber
\end{equation}
지금까지 equation 명령어에 대해 살펴보았는데, equation은 한 줄로 표현 가능한 수식에 사용된다. 그런데 수식 중에는 한 줄로 표현하기 너무 길어서 줄바꿈을 해야 할 필요가 있는 경우, 또는 여러 줄에 걸쳐서 계산 과정을 보여줄 필요가 있는 경우가 있다. 이 때 사용되는 대표적인 명령어가 align이다.
\begin{lstlisting}
	\begin{align}
	\frac{dE}{dt} &=\frac{\partial E}{\partial t} +\sum_i \dot{q}_i \frac{\partial E}{\partial q_i} +\sum_i \ddot{q}_i \frac{\partial E}{\partial\dot{q}_i} \notag\\
	&=\sum_j \dot{q}_j \sum_k \left( A_{jk} q_k +M_{jk} \ddot{q}_k \right). \label{eq002}
	\end{align}
\end{lstlisting}
\TeX\ 파일에 위와 같이 입력하고 컴파일을 해보면 다음과 같은 식을 얻는다.
\begin{align}
\frac{dE}{dt} &=\frac{\partial E}{\partial t} +\sum_i \dot{q}_i \frac{\partial E}{\partial q_i} +\sum_i \ddot{q}_i \frac{\partial E}{\partial\dot{q}_i} \notag\\
&=\sum_j \dot{q}_j \sum_k \left( A_{jk} q_k +M_{jk} \ddot{q}_k \right). \label{eq002}
\end{align}
\TeX\ 파일에 입력된 등호 앞의 \& 기호는 줄 맞춤 표시이다. 즉, 첫 번째 줄의 \&와 두 번째 줄의 \&가 같은 수직선상에 위치한다는 의미이다. 첫 번째 줄에서는 수식 번호를 넣고 싶지 않아서 {\textbackslash}notag라는 명령어를 사용하였다. equation 명령어에서 사용된 {\textbackslash}nonumber와 같은 기능을 한다. 줄을 넘길 때는 {\textbackslash}{\textbackslash}를 사용한다. 이번에는 줄이 너무 길어서 여러 줄에 나누어 표현하는 식의 예를 살펴보자. (P. H. Yoon의 2006 논문에서 참고하였음 \cite{Yoon06})
\begin{lstlisting}
	\begin{align}
	\delta P_{ij}^a =&\frac{i}{B_0} \frac{\Omega_a}{\omega-{\rm\bf k}\cdot{\rm\bf v}_a} \left( m_a n \epsilon_{ikl} v_k^a B_l \delta v_j^a +m_a n \epsilon_{jkl} v_k^a B_l \delta v_i^a \right. \notag\\
	&\left. +\epsilon_{ikl} \delta B_l P_{jk}^a +\epsilon_{jkl} \delta B_l P_{ik}^a +\epsilon_{ikl} B_l \delta P_{jk}^a +\epsilon_{jkl} B_l \delta P_{ik}^a \right) . \label{eq003}
	\end{align}
\end{lstlisting}
위와 같이 작성한 후 컴파일하면 pdf 파일에 다음과 같은 식이 등장한다.
\begin{align}
\delta P_{ij}^a =&\frac{i}{B_0} \frac{\Omega_a}{\omega-{\rm\bf k}\cdot{\rm\bf v}_a} \left( m_a n \epsilon_{ikl} v_k^a B_l \delta v_j^a +m_a n \epsilon_{jkl} v_k^a B_l \delta v_i^a \right. \notag\\
&\left. +\epsilon_{ikl} \delta B_l P_{jk}^a +\epsilon_{jkl} \delta B_l P_{ik}^a +\epsilon_{ikl} B_l \delta P_{jk}^a +\epsilon_{jkl} B_l \delta P_{ik}^a \right) . \label{eq003}
\end{align}
보는 바와 같이 괄호 내부가 너무 길어서 줄바꿈을 했더니 첫 줄에서 괄호가 열리고 다음 줄에서 괄호가 닫힌다. \TeX에서는 괄호를 연 뒤, 닫지 않고 줄바꿈을 하면 에러가 발생한다. 따라서 첫 줄에서 `{\textbackslash}left('로 괄호를 연 뒤, `{\textbackslash}right.'으로 괄호를 닫았다. 이 표시는 실제로 pdf 파일에는 등장하지 않지만 에러 방지를 위해 넣어준 것이다. 두 번째 줄에서도 마찬가지로 `{\textbackslash}left.'으로 (pdf 파일에는 보이지 않는) 괄호를 연 뒤 `{\textbackslash}right)'으로 괄호를 닫아주었다.

align 명령어는 각 줄마다 수식 번호가 작성 가능하다. 그래서 식 (\ref{eq003})과 같이 하나의 식을 두 줄에 나누어 표현할 때 (첫 번째 줄에는 notag 명령어를 넣어 수식 번호를 제거하고) 두 번째 줄에 수식 번호가 등장하도록 하였다. 그런데 이처럼 하나의 식을 여러 줄에 걸쳐 표현할 때는 수식 번호의 상하 위치가 수식 전체의 중앙에 등장하는 편이 더 보기 좋다. 이 경우에는 equation 명령어 내부에 다시 split 명령어를 사용하면 된다. 다음 식 (\ref{eq004})는 이 방법을 사용한 것이다.
\begin{lstlisting}
	\begin{equation}
	\begin{split}
	\mathcal{D}_{m'm}^{(1/2)} (\alpha,\beta,\gamma)=&\left\langle j=\frac{1}{2},m'\right|\exp\left(\frac{-i\,J_z \alpha}{\hbar}\right)\\
	&\times\exp\left(\frac{-i\,J_y \beta}{\hbar}\right)\exp\left(\frac{-i\,J_z \gamma}{\hbar}\right)\left| j=\frac{1}{2},m\right\rangle.
	\end{split}
	\label{eq004}
	\end{equation}
\end{lstlisting}

\begin{equation}
\begin{split}
\mathcal{D}_{m'm}^{(1/2)} (\alpha,\beta,\gamma)=&\left\langle j=\frac{1}{2},m'\right|\exp\left(\frac{-i\,J_z \alpha}{\hbar}\right)\\
&\times\exp\left(\frac{-i\,J_y \beta}{\hbar}\right)\exp\left(\frac{-i\,J_z \gamma}{\hbar}\right)\left| j=\frac{1}{2},m\right\rangle.
\end{split}
\label{eq004}
\end{equation}

\TeX에서 입력되는 수식은 기호와 명령어의 명칭 및 사용법을 알아야 작성할 수 있다. 사용법이 잘 정리된 문서들을 인터넷에서 쉽게 구할 수 있으므로 스스로 해결할 수 있을 것으로 생각된다.


\subsection{Figures}
논문에 들어가는 그림은 크게 사진과 그래프로 나눌 수 있다. 먼저 사진파일(확장자 jpg, png)을 넣는 방법은 다음과 같다.
\begin{lstlisting}
	\begin{figure}[t]
		\begin{center}
			\includegraphics[width=6cm]{Figure File}
			\caption{This is caption}
			\label{figure::label}
		\end{center}
	\end{figure}
\end{lstlisting}
{\textbackslash}begin\{figure\} 옆의 [t]라는 옵션은 그림을 페이지의 맨 위에 넣는다는 것이다. 옵션의 종류는 t, b, h, p가 있다. 각각의 의미는, t (그림을 페이지 맨 위에 위치), b (그림을 페이지 맨 아래에 위치), h (그림을 본문에 작성한 위치에 넣기. 단 충분한 공간이 없는 경우에는 다음 페이지에 들어감), p (한 페이지에 그림만 등장함). 논문에 등장하는 그림들은 되도록 페이지 위에 위치할 수 있도록 하자. 즉 [t] 옵션을 설정한다. 단, section title이 맨 위에 위치하는 경우에는 [b] 옵션을 사용하여 그림을 페이지 바닥에 위치시킨다.

includegraphics다음의 [ ] 안에는 크기를 지정할 수 있는 옵션을 넣어주고 \{ \} 안에는 파일이름(확장자 포함)을 넣어준다. 이때 그림 파일은 \LaTeX\ 파일과 같은 폴더에 들어 있어야 한다. 다음의 caption에는 그림 설명을 넣어준다. 마지막으로 label에는 수식에서와 마찬가지로 이 그림의 라벨을 지정하여 본문 속에서 ref 명령어를 사용하여 그림을 언급하기 편하도록 한다.

이번에는 그래프를 넣는 방법을 알아보자. 그래프는 사진과 약간 차이가 있는데 그것은---엑셀 등 그래프 작성 툴에서 그래프를 그린 후 그래프를 그림파일로 저장할 때, 확장자 pdf, eps 등 postscript 벡터 이미지로 저장을 해야 한다는 것이다. 그 외에는 사진 이미지 삽입과 동일하다. 그림 \ref{Fig01}은 엑셀에서 작성한 그래프를 pdf로 저장한 것과 png로 저장한 것을 비교하기 위한 예이다. 위쪽 그림은 png 확장자, 아래 그림은 pdf 확장자의 파일을 삽입한 것이다. 컴파일 후 생성된 논문 pdf파일을 확대한 상태로 그래프를 살펴보면 그 차이를 알 수 있다.


\begin{figure}[t]
\begin{center}
\includegraphics[width=7.2cm]{Figure01.png}\\
\includegraphics[width=9cm]{Figure01.pdf}
\caption{A linearly damped beat wave.} \label{Fig01}
\end{center}
\end{figure}

논문에 그림을 넣을 때 주의사항은 다음과 같다.
\begin{itemize}
\item{논문에 반드시 필요한 그림인가.}
\item{남들이 이해하는데 있어서 가장 알맞은 형태로 표현되었는가.}
\item{캡션과 본문에서의 설명은 충분한가.}
\end{itemize}
많은 학생들이 연구 과정에서 만들어진 사진들과 그래프들을 선별 과정 없이, 있는대로 다 넣으려는 경향을 보였다. 실험에서 사용한 장치들 사진이나 실험실 사진들을 넣는 경우도 있고, 연구 결과에서 텍스트 없이 그래프들만 나열시킨 경우도 있다. 따라서 십여장 남짓한 본문에서 텍스트가 차지하는 줄 수와 그림이 차지하는 줄 수를 비교하면, 그림의 점유 비율이 월등히 높은 논문들이 많았다. 원칙이 있는 건 아니지만, 글과 그림의 페이지 점유 정도를 비교할 때 글이 차지하는 비중을 높게 하도록 한다. 본인이 작성한 논문에서 그림의 양이 너무 많다면 그림의 양을 줄이고 글의 양을 늘려보자.

먼저 실험 장치 사진을 넣는 경우를 보자. 논문에서 실험 장치 사진을 넣는 것은 i) 장치 개발 자체가 연구 목적인 경우나 ii) 연구실에서 독자적으로 개발한 장치로 연구를 수행하기에 독자들이 연구 장치에도 호기심을 가질 것으로 판단되는 경우에만 필요한 것으로 굳이 필요한 상황이 아니라면 넣지 않도록 한다. 다만 연구 대상이 되는 물체의 상태 변화를 이해하기 쉽도록 그림으로 나타내는 과정에서 장치의 모식도가 함께 그려지는 경우는 종종 있다.

실험 결과를 충분한 텍스트 없이 그림만 나열하는 경우도 있다. 이 때는 그림들이 반드시 모두 들어가야 하는지 생각해본다. 각각의 그림들에 대해서 그림이 차지하는 공간 이상으로 글로 설명해야 할 내용이 있는지 생각해본다. 만약 여러 그림 중 하나만 대표가 선정되어 설명하게 된다면 나머지 그림들은 논문에서 불필요한 그림들이다. 그러나 모든 그림들을 종합하여 볼 필요가 있으며, 종합된 내용을 글로 설명해야 한다면 그림의 표현 방식이 잘못된 것이다. 이 때는 독자들에게 내용 전달에 있어서 효과적인 그림의 표현 방식이 어떤 것일지 고민을 해야 한다.

추가 : subfigure를 쓰는 방법에 대해 알아보자.

% example-image-a, example-image-b는 따로 이미지 파일이 존재하지 않아도 된다.
% http://tex.stackexchange.com/questions/130505/referencing-to-subfigures-in-main-caption 여기에 나와있는 큰 "A", "B" 그림이다. 혹시 에러로 인해 보이지 않는다면 참고. 
\begin{lstlisting}
	\begin{figure}[h]
	\centering
	\begin{subfigure}[t]{0.5\linewidth}
	\centering\includegraphics[width=100pt]{example-image-a}
	\caption{\label{subfig:fig1}}
	\end{subfigure}%
	\begin{subfigure}[t]{0.5\linewidth}
	\centering\includegraphics[width=100pt]{example-image-b}
	\caption{\label{subfig:fig2}}
	\end{subfigure}
	\caption{\subref{subfig:fig1} shows A and~\subref{subfig:fig2} shows B.}
	\label{subfigure_example}
	\end{figure}
\end{lstlisting}
\begin{figure}[h]
	\centering
	\begin{subfigure}[t]{0.5\linewidth}
		\centering\includegraphics[width=100pt]{example-image-a}
		\caption{\label{subfig:fig1}}
	\end{subfigure}%
	\begin{subfigure}[t]{0.5\linewidth}
		\centering\includegraphics[width=100pt]{example-image-b}
		\caption{\label{subfig:fig2}}
	\end{subfigure}
	\caption{\subref{subfig:fig1} shows A and~\subref{subfig:fig2} shows B.}
	\label{subfigure_example}
\end{figure}

동일한 맥락 상에서 그림이 여러 개일 경우 subfigure를 쓰게 된다. 위의 그림의 경우 `A' 라는 label 이 subfig:fig1, `B'라는 그림은 subfig:fig2, 전체 그림은 subfigure\_example 이다. 여기에서 \ref{subfig:fig1}, \subref{subfig:fig2}, \ref{subfigure_example} 등 여러 종류의 referencing을 하고 싶은데, 이는 Table \ref{table_subfigure_ref} 에 나타나 있다. 
\begin{table}[h]
	\caption{Subfigure에서 사용할 수 있는 여러 종류의 referencing.}
	\label{table_subfigure_ref}
	\begin{center}
		\begin{tabular}{|c|c|}
			\hline
			명령어 & 결과 \\
			\hline
			\textbackslash ref\{subfig:fig1\} & \ref{subfig:fig1} \\
			\hline
			\textbackslash subref\{subfig:fig1\} & \subref{subfig:fig1} \\
			\hline
			\textbackslash ref\{subfigure\_example\} & \ref{subfigure_example} \\
			\hline
		\end{tabular}
	\end{center}
\end{table}

한 가지 주의해야 할 점은, \textbackslash subref\{\} 명령어는 여기에서는 (a) 와 같이 나타나지만, 기본 설정으로는 a 와 같이 나타난다는 점이다. gshs\_thesis.cls 를 사용하지 않는 다른 tex 문서에서 (a) 와 같이 나타내고 싶다면 다음과 같은 설정을 추가하면 된다.
\begin{quote}
	\textbackslash captionsetup\{subrefformat=parens\}
\end{quote}

추가 : 그림의 caption 이 길어질 경우, 이것이 List of Figures 에 그대로 실리게 될 경우 길어져서 가독성을 해칠 수 있다. 이럴 때는 다음과 같이 caption 에 List of Figures 에 실리게 될 내용만을 대괄호 안에 같이 적어 주면 된다. 
\begin{quote}
	\verb+\+caption[List of Figures 에 실리는 내용]\{본문에 실리는 내용\}
\end{quote}

\subsection{Tables}

\LaTeX에서 표를 넣는 방법은 i)엑셀 등 외부 프로그램에서 작성된 표를 pdf 그림파일로 변환하여 그림처럼 includegraphics 명령어로 넣는 방법 (물론 이 경우 {\textbackslash}begin\{figure\} 대신 {\textbackslash}begin\{table\}를 사용해야 함), ii)\LaTeX에서 표를 직접 제작하는 방법의 두 가지가 있다. \LaTeX에서는 명령어에 의해 표의 구획과 정렬, 선의 종류를 표현하므로 초보자에게는 다소 어려운 작업이 될 수 있다. 아마도 \LaTeX를 사용함에 있어서 가장 불편한 점은 표 제작 작업일 것이다. 다행히 인터넷에 `latex', `table' 등의 조합으로 검색하면 마우스로 제작한 표를 \LaTeX로 변환해주는 웹사이트들을 찾을 수 있다.

다음은 간단한 표의 예이다. 다음과 같이 입력을 하고 컴파일을 하면,
\begin{lstlisting}
	\begin{table}[h]
	\caption{Physical parameters.} \label{table01}
	\begin{center}
	\begin{tabular}{c|c|c}
	\hline
	& symbol & value \\ \hline
	Earth's mass & $M_E$ & $6.0\times 10^{24} {\rm kg}$ \\
	Earth's radius & $R_E$ & $6.4\times 10^6 {\rm m}$ \\
	Gravitational constant & $G$ & $6.67\times 10^{-11} {\rm N m^2 /kg^2}$ \\ \hline
	\end{tabular}
	\end{center}
	\end{table}
\end{lstlisting}
Table \ref{table01}과 같은 표를 얻게 된다.
\begin{table}[h]
\caption{Physical parameters.} \label{table01}
\begin{center}
\begin{tabular}{c|c|c}
\hline
 & symbol & value \\ \hline
Earth's mass & $M_E$ & $6.0\times 10^{24} {\rm kg}$ \\
Earth's radius & $R_E$ & $6.4\times 10^6 {\rm m}$ \\
Gravitational constant & $G$ & $6.67\times 10^{-11} {\rm N m^2 /kg^2}$ \\ \hline
\end{tabular}
\end{center}
\end{table}

주의사항 : 본 gshs\_thesis.tex 양식을 이용하여 표를 작성할 때는, 반드시 table - center - tabular 환경을 사용해야 한다. 개인적인 성향에 따라 table 안에 \textbackslash centering 을 쓰고 tabular 환경을 사용할 수도 있지만, 이럴 경우 caption 이 표와 겹치는 결과를 불러일으킨다. 원인은 아직 밝혀지지 않았다.

또한, figure의 경우 보통 그림 아래에 caption 이 붙지만, table의 경우 caption을 내용 위쪽에 붙여야 한다. 

\section{Test Section}

Title : 21pt, bold face.

\subsection{Test Subsection}

Title : 16pt, bold face.

\subsubsection{Test Subsubsection}

Title : 14pt, normal style.

앞서 언급하였듯이 subsubsection, paragraph 와 같은 하위 절은 사용을 자제하는 것이 좋다.

\paragraph{Test Paragraph}

Title : 12pt, italic style. No numbering.

%-----------------------------------------------------
% Conclusion
%-----------------------------------------------------

\section{결론}
글이라는 것은 개인의 개성이 담겨 있기 때문에 모든 사람들이 동일한 방식으로 표현하는 것은 아니다. 그러나 고대로부터 개인의 연구 내용을 글로써 타인에게 전달할 때, 효율적인 방법이라고 공감대를 형성하며 다듬어져 온 것이 지금의 논문 형태이다. 그러므로 처음 논문을 작성하는 학생들은 이 문서에서 지시하는 논문 작성 방식을 따르는 것을 권한다. 하지만 여기서는 다양한 논문들에 대해 일일이 사례를 들어 올바른 논문 작성법을 설명하기에는 한계가 있기에 간략하게만 소개를 했다. 여기서 설명되지 않은 부분들은 다른 사람들의 논문을 참고하자. 이미 서론을 작성하면서 많은 선행 연구 논문들을 읽어 봤을 것이다. 그 논문들에서는 데이터를 어떤 방식으로 표현하는지, 서론은 어떤 흐름으로 구성하는지 등을 살펴보자. 논문을 잘 쓰는 비결의 첫 번째는 논문을 많이 읽어 보는 것이다.


%-----------------------------------------------------
% Appendix  (Appendix가 없는 경우 이 파트 전체를 제거하시오)
%-----------------------------------------------------
\clearpage  %%% Appendix를 새 페이지에서 시작
\appendix
\renewcommand{\thesection}{\Alph{section}} %%% TOC에 appendix numbering 재설정
\renewcommand{\thesubsection}{\arabic{subsection}}
\renewcommand{\thesubsubsection}{\arabic{subsubsection}}
\titleformat{\section}[hang] {\normalfont\LARGE\bfseries}{\Alph{section}.}{1em}{} %%% Appendix section title의 재설정
\titleformat{\subsection}[hang] {\normalfont\Large\bfseries}{\Alph{section}.\arabic{subsection}.}{1em}{}
\titleformat{\subsubsection}[hang] {\normalfont\bfseries}{\Alph{section}.\arabic{subsection}.\arabic{subsubsection}.}{1em}{}
\renewcommand{\theequation}{\thesection.\arabic{equation}} %%% Appendix equation numbering 의 재설정
\renewcommand{\thefigure}{\thesection-\arabic{figure}} %%% Appendix figure numbering 의 재설정
\renewcommand{\thetable}{\thesection-\arabic{table}} %%% Appendix table numbering 의 재설정
\setcounter{equation}{0} %%% Appendix equation starting number의 초기화
\setcounter{figure}{0} %%% Appendix figure starting number의 초기화
\setcounter{table}{0} %%% Appendix table starting number의 초기화
\section{Appendix title}
논문의 구조 상 본문에 넣지는 못했지만 논문에 수록하고 싶은 부록(appendix)이 있다면 여기에 작성한다. Appendix는 반드시 있어야 하는 건 아니다. 만약 본문에서 연구 내용을 충분히 작성했다면 이 Appendix 파트 전체를 제거한다. 


%-----------------------------------------------------
%   References 
%   현재 연도가 가장 뒤에 오도록 되어 있지만, 현행 양식은 APA입니다. 아래 예시들은 MLA로 작성되어 있습니다.
%-----------------------------------------------------
\clearpage
\begin{thebibliography}{99}\begin{onehalfspace}

\bibitem{Mok10}C. Mok, C.-M. Ryu, P. H. Yoon, and A. T. Y. Lui, ``Obliquely propagating electromagnetic drift ion cyclotron instability'', J. Geophys. Res., {\bf 115}, A04218 (2010).
\bibitem{Schindler74} K. Schindler, ``A theory of the substorm mechanism'', J. Geophys. Res., {\bf 79}, 2803 (1974).
\bibitem{Sitnov97} M. I. Stinov, H. V. Malova, and A. T. Y. Lui, ``Quasi-neutral sheet tearing instability induced by electron preferential acceleration from stochasticity'', J. Geophys. Res., {\bf 102}, 163 (1997).
\bibitem{Zelenyi08} L. Zelenyi, A. Artemiev, H. Malova, and V. Popov, ``Marginal stability of thin current sheets in the Earth's magnetotail'', J. Atmos. Sol. Terr. Phys., {\bf 70}, 325 (2008).
\bibitem{Cheng98} C. Z. Cheng and A. T. Y. Lui, ``Kinetic ballooning instability for substorm onset and current disruption observed by AMPTE/CCE'', Geophys. Res. Lett., {\bf 25}, 4091 (1998).
\bibitem{Bhattacharjee98} A. Bhattacharjee, Z. W. Ma, and X. Wang, ``Dynamics of thin current sheets and their disruption by ballooning instabilities: A mechanism for magnetospheric substorms'', Phys. Plasmas, {\bf 5}, 2001 (1998).
\bibitem{Dobias04} P. Dobias, I. O. Voronkov, and J. C. Samson, ``On linear plasma instabilities during the substorm expansive phase onset'', Phys. Plasmas, {\bf 11}, 2046 (2004).
\bibitem{Zhu03} P. Zhu, A. Bhattacharjee, and Z. W. Ma, ``Hall magnetohydrodynamic ballooning instability in the magnetotail'', Phys. Plasmas, {\bf 10}, 249 (2003).
\bibitem{Saito08} M. H. Saito, {\it et. al.}, ``Ballooning mode waves prior to substorm-associated dipolarizations: Geotail observations'', Geophys. Res. Lett., {\bf 35}, L07103 (2008).
\bibitem{Friedrich01} E. Friedrich, J. C. Samson, and I. Voronkov, ``Ground-based observations and plasma instabilities in auroral substroms'', Phys. Plasmas, {\bf 8}, 1104 (2001).
\bibitem{Shinohara98} I. Shinohara, {\it et. al.}, ``Low-frequency electromagnetic turbulence observed near the substorm onset site'', J. Geophys. Res., {\bf 103}, 20,365 (1998).
\bibitem{Yoon02} P. H. Yoon, A. T. Y. Lui, and M. I. Sitnov, ``Generalized lower-hybrid drift instabilities in current-sheet equilibrium'', Phys. Plasmas, {\bf 9}, 1526 (2002).
\bibitem{Mok06}C. Mok, C.-M. Ryu, P. H. Yoon, and A. T. Y. Lui, ``Global two-fluid stability of bifurcated current sheets'', J. Geophys. Res., {\bf 111}, A03203 (2006).
\bibitem{Rostoker84} G. Rostoker and J. C. Samson, ``Can substorm expansive phase effects and low frequency Pc magnetic pulsations be attributed to the same source mechanism?'', Geophys. Res. Lett., {\bf 11}, 271 (1984).
\bibitem{Dovias06} P. Dovias, J. A. Wanliss, and J. C. Samson, ``Nonlinear stability of the near-Earth plasma sheet during substorms: 9 February 1995 event'', Can. J. Phys., {\bf 84}, 1029 (2006).
\bibitem{Yoon93} P. H. Yoon and A. T. Y. Lui, ``Nonlinear analysis of generalized cross-field current instability'', Phys. Fluids B, {\bf 5}, 836 (1993).
\bibitem{Sadovskii01} A. M. Sadovskii and A. A. Galeev, ``Quasilinear theory of the ion Weibel instability in the Earth's magnetospheric tail'', Plasma Phys. Rep., {\bf 27}, 490 (2001).
\bibitem{Yoon06} P. H. Yoon and A. T. Y. Lui, ``Quasi-linear theory of anomalous resistivity'', J. Geophys. Res., {\bf 111}, A02203 (2006).

\end{onehalfspace}\end{thebibliography}

%-----------------------------------------------------
%   Summary (영어로 작성한 학생은 이 부분 전체를 제거한다.)
%   2021: Summary 제거
%-----------------------------------------------------
%\begin{summary}
%\addcontentsline{toc}{section}{요약}  %%% TOC에 표시
%영어로 논문 작성 시 5페이지 이상의 한글 요약을 작성해야 합니다. 만약 아니라면 이 부분을 주석처리하세요.
%\end{summary}

%-----------------------------------------------------
%   감사의 글
%-----------------------------------------------------
\begin{acknowledgements}
\addcontentsline{toc}{section}{감사의 글}  %%% TOC에 표시
정말 감사합니다.
\end{acknowledgements}

%-----------------------------------------------------
%   연구활동 
%-----------------------------------------------------
\begin{researches}
\addcontentsline{toc}{section}{연구활동}  %%% TOC에 표시
\begin{itemize}
\item{2011학년도 교내 R\&E 발표대회에서 장려상 수상}
\item{2012학년도 교내 R\&E 발표대회에서 장려상 수상}
\item{2013학년도 교내 R\&E 발표대회에서 장려상 수상}
\item{2014학년도 교내 R\&E 발표대회에서 장려상 수상}
\item{2015학년도 교내 R\&E 발표대회에서 장려상 수상}
\item{2016학년도 교내 R\&E 발표대회에서 장려상 수상}
\item{2017학년도 교내 R\&E 발표대회에서 장려상 수상}
\item{2018학년도 교내 R\&E 발표대회에서 장려상 수상}
\item{2019년 노벨 물리학상 수상}
\end{itemize}
\end{researches}


\end{document} 
