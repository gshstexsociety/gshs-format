%[2015/10/13 GSHS 기초/심화 R&E report sample document]
% by GSHS LaTeX Intro.
% Website : http://gshslatexintro.github.io/gshslatexintro

\documentclass[12pt]{article}
\usepackage[left=25mm,right=25mm,top=30mm,bottom=30mm]{geometry}
\usepackage{indentfirst}
\setlength\parindent{2.0em}
\usepackage{amsmath} % math
\usepackage{amssymb} % math
\usepackage{graphicx} % to use \includegraphics{}
\usepackage{diagbox} % to make advanced tables
\usepackage{multirow} % to make advanced tables
\usepackage[hangul]{kotex} % hangul모드를 사용하면 'Contents'가 '차례'와 같이 나오는 등 한글화됨.
\usepackage{color}
\usepackage[hidelinks]{hyperref}


\linespread{1.6}
\newcommand{\makecover}{
	\thispagestyle{empty}
	\noindent
	2015학년도\\
	{\sf 경기과학고등학교 심화R\&E 결과보고서}\\
	\vskip 3cm
	\begin{center}
		\fontsize{21pt}{21pt}\selectfont 보고서 제목 \\[21pt]
	\end{center}
	\vskip 3cm
	\begin{center}
		\today
	\end{center}
	\vskip 2.5cm
	\begin{center}
		\large 연구참여자 : 가나다(hong@e-mail.address)\\
		\hskip 7pc 라마바(hong@e-mail.address)\\
	\end{center}
	\vskip 3pc
	\begin{center}
		\large 지도교사 : 홍판서
	\end{center}
	\vskip 5pc
	\begin{center}
		\large 과학영재학교 \Large 경기과학고등학교
	\end{center}
	\newpage
	\pagenumbering{roman} %% 표지가 끝난 이후로 번호 매기기.
	%% abstract 까지도 번호를 매기고 싶지 않다면 \thispagestyle{empty}를 사용하고, 그 이후에 \pagenumbering{arabic 또는 roman}을 써주면 됩니다.
}

\renewenvironment{abstract}{
	\vspace*{2cm}
	\begin{center}
		\textbf{Abstract}
	\end{center}
	\vspace*{4mm}
	\hspace*{5mm}
	\addcontentsline{toc}{section}{Abstract}
}

\newenvironment{abstractkor}{%
	\vspace*{2cm}
	\begin{center}
		\Large \textbf{초~~~~~록}
	\end{center}
	\vspace*{4mm}
	\hspace*{5mm}
	\addcontentsline{toc}{section}{초록}
}

%%% 여기까지 템플릿 (재)설정. %%%
%% 다음 주석까지는 abstract를 포함한 문서의 초반. 그 이후로는 줄간격 원상복귀%%
%%% 주의! chapter를 사용하면 안됩니다. 관례적으로 논문 작성시에는 section, subsection, subsubsection을 사용합니다.

\begin{document}
\makecover
\tableofcontents
\newpage
\begin{abstract}
	English abstract
	\newpage
\end{abstract}

\begin{abstractkor}
	한국어 초록
	\newpage
\end{abstractkor}

%% 여기까지 문서의 초반. 줄간격 복귀를 위해 \linespread{1.0}

\linespread{1.0}

\newcommand{\scin}{Scintillation}
\newcommand{\PM}{Photomultiplier}
\newcommand{\etal}{\textit{et al.}}

\pagenumbering{arabic}
\begin{center}
	\Large 관측계획서 : 시간에 따른 별의 섬동(\scin) 경향성
	
	\Large Obsevation Plan : Graphing the Amount of \scin about time
	\normalsize
	\begin{flushright}
		경기과학고등학교 14041 박승원
		
		\today
	\end{flushright}
\end{center}
\section{연구 개요}

관측천문을 하는 데에 있어서 별의 섬동(\scin)은 무시하지 못할 요소이다. \scin 이란 천체를 관측하는 데에 있어서 상층 대기의 요동에 의해 빛의 경로가 휘어 별이 반짝여 보이는 현상으로, 직경이 작은 망원경일수록 심하게 나타난다.

어떤 망원경을 이용하여 밝기 변화에 민감한 연구를 진행하기 위해서는 별의 섬동이 그 망원경이 설치된 지역에 맞게 수치적으로 DB화되어있어야 한다.
\footnote{그 예시로 남극에 위치한 Dome C에서의 turbulence profile을 제공하는 reference로서의 논문이 있다.\cite{ant_scin}}
하지만, 수원시 장안구 송죽동에 맞는 별의 섬동 DB에 관한 연구가 없었다.

나는 이 관측을 통해 경기도 수원시 장안구 송죽동
\footnote{위도 37.3도, 경도 127.0도}
에 위치한 경기과학고등학교 천문대에서의 시간에 따른 섬동(\scin)의 정도를 구하고자 한다. 또한, 가능하다면 J. Osborn \etal \cite{Osborn}에 의한 새로운 \scin 이론을 수치적으로나마 검증해 보고자 한다.

\section{사용기기}

\begin{itemize}
	\item 망원경 : Celestron C-14. F ; 3910mm. f/11 \cite{telescope}
	\item CCD : SBIG STX-16803 \cite{ccd}
\end{itemize}

\section{관측개요}
되도록 밝은 별을 촬영하고자 한다. 밝은 별을 촬영해야 짧은 노출 시간으로도 선명한 상을 얻을 수 있고, 찾기도 쉽기 때문이다. 노출 시간이 짧아야 \scin 이 더 잘 나타날 것으로 예상된다. 

\subsection{이론적 배경 - 시간에 따른 Scintillation의 정도} \label{exp_1}

Brian D.Warner의 \textquotedblleft A Practical Guide to Lightcurve Photometry and Analysis \textquotedblright 에 의하면 \scin 에 의한 분산 $\sigma$는 식 \ref{young}과 같이 계산된다.
\begin{equation} \label{young}
\sigma = 0.09 \cdot \frac{X^{1.5}}{D^{2/3}\sqrt{2\cdot t}} \cdot \exp{(-h/h_0)}
\end{equation}
\begin{itemize}
	\item $X$ = airmass = $\sec{\theta}$ where $\theta$ is the zenith angle of the observation
	\item $D$ = telescope aperture(cm)
	\item $t$= exposure duration(sec)
	\item $h$ = observatory elevation(cm)
	\item $h_0$ = atmospheric scale height
\end{itemize}

식 \ref{young}는 1967년에 발표된 Young's Equation이다. [citation needed] 이 식의 0.09는 경험적인 값으로, McDonald 천문대에서 측정된 수치이다. 하지만, 어떠한 상황에 대해서도 \scin 의 정도가 0.09와 같은 중간값과는 시간에 따라 차이가 있을 수 있다.\cite{Osborn} 1998년에 식 \ref{dravin}과 같은 Dravin's Equation이 발표되었다.\cite{AISS.1}

\begin{equation} \label{dravin}
\sigma = \sqrt{ 10.7\int_{0}^{\infty}{ \frac{C_n^2(h)h^2dh}{V_\perp} } } X^{7/4}D^{-2/3}t_{exp}^{-1/2}
\end{equation}

\subsection{시간에 따른 \scin 분산 $\sigma$그래프 얻기} \label{sigma-t}

식 \ref{dravin}에 의하면 $\sigma$는 $1/\sqrt{t}$에 비례할 것으로 예상된다. 따라서 나는 동일한 시간대에 같은 별을 서로 다른 노출시간을 주어 가며 촬영해 보고, $\sigma - t$그래프를 그려 볼 것이다.
어떤 임계치 $t_c$에 대해 $t>t_c$의 경우 \scin 이 더 이상 잘 나타나지 않을 것을 예상된다. 실험 데이터를 통해 $t_c$를 결정할 것이다.
\ref{exp_1}절의 방법대로 $t_c$를 결정한 후, $t_c$, $t_c$보다 짧은 시간, $t_c$보다 긴 시간의 노출시간을 주어 같은 시간대에 같은 별을 서로 다른 세 노출시간으로 촬영해 볼 것이다.

촬영할 별의 위치는 천정에 있는 것을 채택할 것이다. 천정에서의 대기에 의한 \scin 을 촬영해야 우리 지역의 대기에 의한 효과만을 고려할 수 있기 때문이다.
9월에 관측하게 될 경우 9월 23일 추분 기준으로 태양의 적경이 0h이며, 적위는 37.3도의 별들을 보게 된다.

\section{\scin 을 어떻게 구할 것인가?}

\subsection{촬영 방법}
기본적인 촬영 설정은 다음과 같다.
\begin{itemize}
	\item Exposure Preset : Find DSO 
	\item Filter Wheel : V
	\item Exposure Time : Manual
	\item Image Size : 4096*4096 (X Binning = 1)
\end{itemize}

\subsection{Simple Method - Data Reduction} \label{simple_reduction}

\ref{sigma-t}절에서도 논한 바와 같이 노출시간을 바꾸어 가며 특정 별을 촬영하는 것이 가장 간단한 방법이라 할 수 있다. 구체적인 방법은 2015년 9월 17일 관측일지에서 논한다.

하지만, 별이 떨리는 정도는 대기에 의한 효과 뿐만 아니라 망원경의 Tracking 오류 혹은 지표의 흔들림도 있기 때문에 여러 논문을 찾아보며 더 나은 방법을 찾고자 하였다.

\subsection{Speckle Imaging} \label{speckle_img}

Speckle Imaging은 천문학에서 뿐만 아니라 생체 내 혈관 조직 등을 촬영하기 위해서도 많이 쓰이는 방법으로, 광학 기기에서의 '번짐'을 줄이기 위한 촬영 기술 중의 하나이다. \cite{speckle_img}

\subsection{By using \PM} \label{photomultiplier}

\PM 을 사용하는 이유는 ㅁㄴㅇㄹ\cite{paterno}

Beam Splitter 한 개, 두 개의 광전자 증배관(\PM)과 Digital Correlator를 이용하여 Scintillation의 정도를 파악하는 QVANTOS 방법이 있다. \cite{AISS.1} 


\section{관측계획}
\subsection{날씨}

관측할 날을 정하기 위해 고려해야 할 가장 큰 점은, (당연하지만) 구름이 없어야 하며 달이 관측시간 내외에서 남중하는 날은 피해야 한다. 2015년 9~12월 중 보름달이 뜨는 날은 9월 28일, 10월 28일, 11월 26일, 12월 26일이다. 반면 달이 아예 뜨지 않는 날(삭)은 9월 13일, 10월 12일, 11월 11일, 12월 10일이다.

\subsection{관측 횟수}

관측 시 발생할 실수나 오류 등을 고려하고, 충분한 데이터를 얻기 위하여 관측은 3회정도 실시해야 할 것이다.

\subsection{관측하는 날의 구체적 일정}
\begin{enumerate}
	\item www.kma.go.kr\cite{kma}에서 날씨영상-위성-기본영상으로 구름이 적은지 확인. 적외영상으로 아시아 전체 모습도 확인할 것.
	
	\item 정오 전까지 김혁 선생님(010-5536-0743)께 문자드리기, 송죽학사에서 관측실 시설사용 신청
	
	\item 오후 5시에 온도 안정화를 위해 돔을 미리 오픈해두고, 망원경과 CCD cooler를 켜놓는다.
	
	\item 일몰 직후 Flat 보정용영상을 촬영한다. Flat은 Alt 60도 내외에서 SouthEast를 바라보며 25000$\sim$30000 ADU가 되도록 노출시간을 조정하며 동$\longrightarrow$서 1$\sim$2도 간격으로 5장 정도를 찍는다.
	
	\item Flat 촬영 후 저녁식사.
	
	\item 망원경의 한 지점을 잡아서 Sync.
	
	\item 관측.
	
	\item 마무리 : CCD warm up 이후 power off, dome close
	
\end{enumerate}

\subsection{일반적인 주의사항들}
- Scintillation을 관측하기 위해 별의 떨림을 측정해야 하므로, 망원경의 떨림을 최소화해야 한다. 바람이 많이 부는 날은 피해야 하며, 망원경에 노출을 주고 있는 동안에는 몸의 동작을 삼가야 한다.

\addcontentsline{toc}{section}{References}
\begin{thebibliography}{00}
	% APA style
	\bibitem{Osborn}{Osborn, J., Föhring, D., Dhillon, V. S., \& Wilson, R. W. (2015). Atmospheric scintillation in astronomical photometry. Monthly Notices of the Royal Astronomical Society, \textbf{452}(2), 1707-1716.}
	
	\bibitem{ant_scin}{Kenyon, S. L., Lawrence, J. S., Ashley, M. C., Storey, J. W., Tokovinin, A., \& Fossat, E. (2006). Atmospheric Scintillation at Dome C, Antarctica: Implications for Photometryand Astrometry. Publications of the Astronomical Society of the Pacific, \textbf{118}(844), 924-932.}
	
	\bibitem{speckle_img}{Beavers, W., Dudgeon, D. E., Beletic, J. W., \& Lane, M. T. (1989). Speckle imaging through the atmosphere. Unknown, 1.}
	
	\bibitem{AISS.1}{Dravins, D., Lindegren, L., Mezey, E., \& Young, A. T. (1997). Atmospheric intensity scintillation of stars. I. Statistical distributions and temporal properties. Publications of the Astronomical Society of the Pacific, 173-207.}
	
	\bibitem{paterno}{Paterno, L. (1976). Spectrum measurements of star atmospheric scintillation. Astronomy and Astrophysics, \textbf{47}, 437-441.}
	
	\bibitem{kma}{www.kma.go.kr (일출/일몰/월출/월몰시간 데이터)}
	
	\bibitem{telescope}{http://www.celestron.com/browse-shop/astronomy/optical-tubes/c14-a-xlt-\%28cge\%29-optical-tube-assembly}
	
	\bibitem{ccd}{https://www.sbig.com/products/cameras/stx/stx-16803/}
	
\end{thebibliography}
\end{document}
