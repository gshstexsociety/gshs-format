\makecover
%\maketitle  % command to print the title page with above variables

%%
%% 앞표지 생성
%%

\newcommand{\makecover}{  %
	\renewcommand{\baselinestretch}{1.3}%
	%	\renewcommand{\baselinestretch}{3.3}%
	\thispagestyle{empty} \noindent%
	%	\pagestyle{empty} \noindent%
	\@summitYear 년도  \textbf{창의연구 R\&E 연구 노트}
	%{\sf \textbf{창의연구 R\&E 연구 노트}}
	\vskip 2.5cm
	\noindent\begin{minipage}[t]{\textwidth}
		\begin{center}
			\fontsize{21pt}{21pt}\selectfont  \@researchTitle \\[21pt]
			\fontsize{21pt}{21pt}\selectfont  \@title \\[21pt]
		\end{center}
	\end{minipage}
	\vskip 3.2cm %3cm + 12pt*1.3*1.2
	\noindent\begin{minipage}[c]{\textwidth}
		\begin{center}
			\@summitYear. \@summitMonth. \@summitDate
		\end{center}
	\end{minipage}
	\vskip 3.2cm %3cm + 12pt*1.3*1.2
	\noindent\begin{minipage}[c]{\textwidth}
		\begin{center}
			\large 연구참여자 : \@firstAuthor(\@firstAuthorEmail)\\ 
			\ifthenelse{\equal{\@secondAuthor}{}}{} {\hskip 7pc 
				\@secondAuthor(\@secondAuthorEmail)\\ }%
			\ifthenelse{\equal{\@thirdAuthor}{}}{}{\hskip 7pc 
				\@thirdAuthor(\@thirdAuthorEmail)\\ }%
			\ifthenelse{\equal{\@forthAuthor}{}}{}{\hskip 7pc 
				\@forthAuthor(\@forthAuthorEmail)\\ }%
			\ifthenelse{\equal{\@fifthAuthor}{}}{}{\hskip 7pc 
				\@fifthAuthor(\@fifthAuthorEmail)\\ }%
		\end{center}
	\end{minipage}
	\vskip 3.5pc %3pc + 12pt*1.3*1.2
	\noindent\begin{minipage}[c]{\textwidth}
		\begin{center}
			\large 지도교사: \@advisor(\@advisorEmail)
		\end{center}
	\end{minipage}
	\vskip 2pc
	\noindent\begin{minipage}[c]{\textwidth}
		\begin{center}
			\large \ifthenelse{\equal{\@type}{심화}}{지도교수: 
				\@professor(\@professorEmail)}{}
		\end{center}
	\end{minipage}
	\vskip 5pc
	\noindent \begin{minipage}[c]{\textwidth}
		\begin{center}
			{\normalfont\fontsize{18}{18}\selectfont\bfseries 
				과학영재학교}
			~
			{\normalfont\fontsize{24}{24}\selectfont\bfseries 
				경기과학고등학교}
		\end{center}
	\end{minipage}
	\newpage
}




\pagenumbering{roman}                        % 로마자 페이지 시작
\setcounter{page}{1}
%---------------------------------------------------------------------
%                  영문 초록을 입력하시오
%---------------------------------------------------------------------
%\begin{abstracts}     %this creates the heading for the abstract page
%	\addcontentsline{toc}{section}{Abstract}  %%% TOC에 표시
%	\noindent{
%		Put your abstract here. 
%	}
%\end{abstracts}
%
%---------------------------------------------------------------------
%                  국문 초록을 입력하시오
%---------------------------------------------------------------------
%\begin{abstractskor}        %this creates the heading for the abstract page
%	\addcontentsline{toc}{section}{초록}  %%% TOC에 표시
%	\noindent{
%		초록
%	}
%\end{abstractskor}


%----------------------------------------------
%   Table of Contents (자동 작성됨)
%----------------------------------------------
\cleardoublepage
\addcontentsline{toc}{section}{Contents}
\setcounter{secnumdepth}{3} % organisational level that receives a numbers
\setcounter{tocdepth}{3}    % print table of contents for level 3
\baselineskip=2.2em
\tableofcontents


%----------------------------------------------
%     List of Figures/Tables (자동 작성됨)
%----------------------------------------------
\cleardoublepage
\clearpage
\listoftables
% 표 목록과 캡션을 출력한다. 만약 논문에 표가 없다면 이 위 줄의 맨 앞에 
% `%' 기호를 넣어서 주석 처리한다.

\cleardoublepage
\clearpage
\listoffigures
% 그림 목록과 캡션을 출력한다. 만약 논문에 그림이 없다면 이 위 줄의 맨 앞에 
% `%' 기호를 넣어서 주석 처리한다.

\cleardoublepage
\clearpage
\renewcommand{\thepage}{\arabic{page}}
\setcounter{page}{1}