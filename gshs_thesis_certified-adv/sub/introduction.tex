\section{Introduction}

서론은 연구를 진행하게 된 배경을 기술하는 곳으로 보통 다음과 같은 순서로 쓰는 편이다.
\begin{itemize}
\item{연구 주제의 전반적 관심을 조명.}
\item{연구 분야의 스페셜 이슈를 조명.}
\item{해당 이슈를 해결하기 위한 다양한 선행 연구들을 서술.}
\item{선행 연구들의 한계점을 기술.}
\item{한계를 극복하기 위한 본 연구의 목적을 밝힘.}
\item{논문의 구성을 서술 (optional).}
\end{itemize}
서론은 과거부터 현재까지 해당 분야의 연구 진행을 기술하기 때문에 선행 연구 논문들을 레퍼런스로 도입하는 경우가 빈번하게 나타난다. \LaTeX에서 참고문헌을 표기하는 방법을 알아보자. 먼저 이 문서의 후반부에 위치한 레퍼런스 부분을 찾아간다. 이 문서를 컴파일했을 때 생성된 PDF 파일에는 {\bf References}라고 나와 있지만 여기서는 {\textbackslash}begin\{thebibliography\}\{99\}로 시작에서 {\textbackslash}end\{thebibliography\}로 종료되는 그 사이에 참고문헌을 작성하면 된다. 여기서 숫자 99는 참고문헌이 100개 넘는 논문을 작성하는 것이 아니라면 그대로 놔둔다. 참고문헌 작성 예시는 다음과 같다.
\begin{lstlisting}
\bibitem{Mok06}C. Mok, C.-M. Ryu, P. H. Yoon, and A. T. Y. Lui, ``Global twofluid stability of bifurcated current sheets'', J. Goephys. Res., {\bf 111}, A03203
(2006).
\end{lstlisting}
{\textbackslash}bibitem 다음의 \{ \} 안에는 자신이 그 논문을 기억하기 쉬운 규칙을 정하여 작성하면 된다. 보통 논문 주저자의 last name과 논문 출판 년도를 사용하여 표기한다. 그리고 저자들, 논문 제목, 저널 이름, 권, 호, 페이지, 출판년도 순으로 입력한다. 저자는 3인 이하일 경우에는 모두 적도록 하고 4인 이상일 때는 주저자만 작성하고 그 외는 et al.이란 표기로 대체한다. 논문 제목은 큰 따옴표로 묶어준다. \LaTeX에서 시작하는 따옴표는 키보드에서 숫자 1 왼쪽 버튼, 마치는 따옴표는 키보드 엔터키 왼쪽 버튼을 사용함에 유의하라. 저널 이름은 경우에 따라 약어를 사용할 수 있다. 약어를 사용할 때는 정식으로 정해진 약어임을 확인한 후 사용한다. Volume(권)은 굵은 글자 처리한다. 위의 예시처럼 {\textbackslash}bf를 사용하면 된다. Number(호)는 경우에 따라 없는 저널도 있다. 위의 예시에는 `호'를 넣지 않았다. 만약 `호'를 넣고자 할 때는 둥근 괄호로 묶어준다. 마지막으로 페이지와 출판년도를 작성한다. 출판년도는 둥근 괄호로 묶어준다.

이제 서론에서 해당 논문을 인용할 준비 작업은 끝났다. 서론에서 필요한 부분에 이 논문을 인용 표기할 경우 {\textbackslash}cite 라고 입력한 후 \{ \} 안에 해당 논문을 표시하면 된다. 표시하는 방법은 바로 레퍼런스에서 {\textbackslash}bibitem 이후 \{ \} 안에 적었던 것을 넣어주면 된다. 논문 인용 표시가 문장 마지막에 등장할 때는 마침표의 위치는 인용 표시 다음이다. 아래 문장은 논문 인용 표시의 예로 C. Mok의 2010년 논문에서 인용하였다 \cite{Mok10}.
\begin{lstlisting}
Various plasma instabilities have been proposed as playing important roles during the substorm onset process. These include the tearing \cite{Schindler74, Sitnov97, Zelenyi08}, ballooning \cite{Cheng98, Bhattacharjee98, Dobias04, Zhu03, Saito08, Friedrich01}, lower hybrid drift \cite{Shinohara98, Yoon02, Mok06}, Kelvin--Helmholtz \cite{Rostoker84, Dovias06}, and the ion Weibel \cite{Yoon93, Sadovskii01} instabilities.
\end{lstlisting}
위와 같이 입력한 후 컴파일하면 pdf 파일에는 다음과 같이 나타날 것이다.
\begin{quote}
Various plasma instabilities have been proposed as playing important roles during the substorm onset process. These include the tearing \cite{Schindler74, Sitnov97, Zelenyi08}, ballooning \cite{Cheng98, Bhattacharjee98, Dobias04, Zhu03, Saito08, Friedrich01}, lower hybrid drift \cite{Shinohara98, Yoon02, Mok06}, Kelvin--Helmholtz \cite{Rostoker84, Dovias06}, and the ion Weibel \cite{Yoon93, Sadovskii01} instabilities.
\end{quote}
이 때 참고문헌이 번호 순서대로 나오도록 한다. 또한 세 개 이상의 문헌이 연속된 번호로 이어진 경우 자동으로 첫 번호와 마지막 번호가 hyphen으로 연결된 형태로 등장함을 확인할 수 있다.


참고문헌은 다음의 조건들을 만족해야 한다.
\begin{itemize}
\item{저자가 명시되어야 한다.}
\item{검증이 된 내용이어야 한다.}
\item{이미 출판되어 수정이 불가능해야 한다.}
\end{itemize}
전문 논문 저널에 수록된 논문들은 위 조건들을 만족하므로 되도록 논문을 참고문헌으로 삼도록 한다. 웹사이트는 위 조건들을 만족하지 못하므로 참고문헌으로 부적절하다. 또한 누구라도 책을 출판할 수 있으므로 전문 서적을 참고문헌으로 사용하는 경우에는 널리 받아들여지고 인정받는 서적만 사용해야 한다. 사실 전공 서적의 저자는 여러 연구 논문들을 참고로 하여 책을 집필하기 때문에 전공서적에도 참고 문헌(논문)이 명시되어 있다. 이 경우 전공 서적 대신에 책에서 지시하는 논문을 참고문헌으로 삼도록 한다.

