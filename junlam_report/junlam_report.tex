%%!TeX program = xelatex
%% 부득이하게 pdflatex을 사용해야 할 경우 위의 magic comment를 제거하십시오.

% Initiated by 정민석(2014년도 경기과학고 수학과전문교원)
% Continously being modified by 경기과학고 TeX 사용자협회
% Website : http://gshslatexintro.github.io 

\usepackage{graphicx}
\graphicspath{ {./figures/} } % Figure는 figures 폴더에 저장
\usepackage{wrapfig}
\usepackage{amsmath}
\usepackage[backend=bibtex]{biblatex}
\addbibresource{sub/ref.bib}
\usepackage{subcaption}
\usepackage{hyperref}
\usepackage{pgfplots}
\usepackage{pgfplotstable}
\usepackage{tikz}
\usepackage{tikzscale}
%\tikzset{>=stealth}
\tikzset{>=latex}
\usetikzlibrary{positioning,fadings,backgrounds,angles,quotes,patterns,calc,fit}
\usepackage{tabularx}
\usepackage{array}
\usepackage{setspace}
% \usepackage{booktabs}
% \newcolumntype{M}[1]{>{\centering\arraybackslash}m{#1}} % Vertical & horizontal centering cell.
% \usepackage{stackengine}
% \usepackage{siunitx}
% \sisetup{per-mode=symbol,inter-unit-product = \ensuremath{\cdot}}
% \def\(#1\){\addstackgap{$\displaystyle#1$}}
\usepackage{indentfirst}

%\setmonofont[AutoFakeBold]{Courier New} % usepackage 등의 명령어는 여기에.
\usepackage{cite}
\usepackage{textcomp}
\usepackage{tocloft}
\setlength{\cftbeforesecskip}{0pt}
\setlength{\cftbeforesubsecskip}{0pt}
\setlength{\cftbeforesubsubsecskip}{0pt}

% 본문 시작
\begin{document}

	%표지만들기
	%makecover 함수와 관련하여 "Underfull \hbox (badness 10000) in paragraph" 오류는 무시하십시오. (TeXstudio ver 2.9.4 오류 기준)
	%\makecover
	

	%\baselineskip=2.2em         % line spacing in the paragraph
	%\maketitle  % command to print the title page with above variables
\makecover  % command to print the title page with above variables

\setcounter{page}{1}
\renewcommand{\thepage}{\roman{page}}

%----------------------------------------------
%   Table of Contents (자동 작성됨)
%----------------------------------------------
\cleardoublepage
\addcontentsline{toc}{section}{Contents}
\setcounter{secnumdepth}{3} % organisational level that receives a numbers
\setcounter{tocdepth}{3}    % print table of contents for level 3
\baselineskip=2.2em
\tableofcontents


%----------------------------------------------
%     List of Figures/Tables (자동 작성됨)
%----------------------------------------------
\cleardoublepage
\clearpage
\listoffigures	% 그림 목록과 캡션을 출력한다. 만약 논문에 그림이 없다면 이 줄의 맨 앞에 %기호를 넣어서 코멘트 처리한다.

\cleardoublepage
\clearpage
\listoftables  % 표 목록과 캡션을 출력한다. 만약 논문에 표가 없다면 이 줄의 맨 앞에 %기호를 넣어서 코멘트 처리한다.


\cleardoublepage
\clearpage

%---------------------------------------------------------------------
%                  영문 초록을 입력하시오
%---------------------------------------------------------------------
%\begin{abstracts}     %this creates the heading for the abstract page
%	\addcontentsline{toc}{section}{Abstract}  %%% TOC에 표시
%	\noindent{
%			Put your abstract here. Once upon a time, \gshs said : `The first, and the best.'
%	}
%\end{abstracts}

%\cleardoublepage
%\clearpage

\begin{abstractskor}
	\addcontentsline{toc}{section}{초록}  %%% TOC에 표시
	\noindent{
		
	여기에 국문 초록을 입력하십시오
	
	}
\end{abstractskor}






 % Abstract
	
	%%%%%%%%%%%%%%%%%%%%%%%%%%%%%%%%%%%%%%%%%%%%%%%%%%%%%%%%%%%
	%%%% Main Document %%%%%%%%%%%%%%%%%%%%%%%%%%%%%%%%%%%%%%%%
	%%%%%%%%%%%%%%%%%%%%%%%%%%%%%%%%%%%%%%%%%%%%%%%%%%%%%%%%%%%
	\cleardoublepage
	\clearpage
	\renewcommand{\thepage}{\arabic{page}}
	\setcounter{page}{1}
	
	%각 장을 아래와 같이 sub 폴더 안에 만들어서 넣으면 편리하다.
	\section{Introduction}

서론은 연구를 진행하게 된 배경을 기술하는 곳으로 보통 다음과 같은 순서로 쓰는 편이다.
\begin{itemize}
\item{연구 주제의 전반적 관심을 조명.}
\item{연구 분야의 스페셜 이슈를 조명.}
\item{해당 이슈를 해결하기 위한 다양한 선행 연구들을 서술.}
\item{선행 연구들의 한계점을 기술.}
\item{한계를 극복하기 위한 본 연구의 목적을 밝힘.}
\item{논문의 구성을 서술 (optional).}
\end{itemize}
서론은 과거부터 현재까지 해당 분야의 연구 진행을 기술하기 때문에 선행 연구 논문들을 레퍼런스로 도입하는 경우가 빈번하게 나타난다. \LaTeX 에서 참고문헌을 표기하는 방법을 알아보자. 먼저 이 문서의 후반부에 위치한 레퍼런스 부분을 찾아간다. 이 문서를 컴파일했을 때 생성된 PDF 파일에는 {\bf References}라고 나와 있지만 여기서는 {\textbackslash}begin\{thebibliography\}\{99\}로 시작에서 {\textbackslash}end\{thebibliography\}로 종료되는 그 사이에 참고문헌을 작성하면 된다. 여기서 숫자 99는 참고문헌이 100개 넘는 논문을 작성하는 것이 아니라면 그대로 놔둔다. 참고문헌 작성 예시는 다음과 같다.
\begin{lstlisting}
\bibitem{Mok06}C. Mok, C.-M. Ryu, P. H. Yoon, and A. T. Y. Lui, ``Global twofluid stability of bifurcated current sheets'', J. Goephys. Res., {\bf 111}, A03203
(2006).
\end{lstlisting}
{\textbackslash}bibitem 다음의 \{ \} 안에는 자신이 그 논문을 기억하기 쉬운 규칙을 정하여 작성하면 된다. 보통 논문 주저자의 last name과 논문 출판 년도를 사용하여 표기한다. 그리고 저자들, 논문 제목, 저널 이름, 권, 호, 페이지, 출판년도 순으로 입력한다. 저자는 3인 이하일 경우에는 모두 적도록 하고 4인 이상일 때는 주저자만 작성하고 그 외는 et al.이란 표기로 대체한다. 논문 제목은 큰 따옴표로 묶어준다. \LaTeX에서 시작하는 따옴표는 키보드에서 숫자 1 왼쪽 버튼, 마치는 따옴표는 키보드 엔터키 왼쪽 버튼을 사용함에 유의하라. 저널 이름은 경우에 따라 약어를 사용할 수 있다. 약어를 사용할 때는 정식으로 정해진 약어임을 확인한 후 사용한다. Volume(권)은 굵은 글자 처리한다. 위의 예시처럼 {\textbackslash}bf를 사용하면 된다. Number(호)는 경우에 따라 없는 저널도 있다. 위의 예시에는 `호'를 넣지 않았다. 만약 `호'를 넣고자 할 때는 둥근 괄호로 묶어준다. 마지막으로 페이지와 출판년도를 작성한다. 출판년도는 둥근 괄호로 묶어준다.

이제 서론에서 해당 논문을 인용할 준비 작업은 끝났다. 서론에서 필요한 부분에 이 논문을 인용 표기할 경우 {\textbackslash}cite 라고 입력한 후 \{ \} 안에 해당 논문을 표시하면 된다. 표시하는 방법은 바로 레퍼런스에서 {\textbackslash}bibitem 이후 \{ \} 안에 적었던 것을 넣어주면 된다. 논문 인용 표시가 문장 마지막에 등장할 때는 마침표의 위치는 인용 표시 다음이다. 아래 문장은 논문 인용 표시의 예로 C. Mok의 2010년 논문에서 인용하였다 \cite{Mok10}.
\begin{lstlisting}
Various plasma instabilities have been proposed as playing important roles during the substorm onset process. These include the tearing \cite{Schindler74, Sitnov97, Zelenyi08}, ballooning \cite{Cheng98, Bhattacharjee98, Dobias04, Zhu03, Saito08, Friedrich01}, lower hybrid drift \cite{Shinohara98, Yoon02, Mok06}, Kelvin--Helmholtz \cite{Rostoker84, Dovias06}, and the ion Weibel \cite{Yoon93, Sadovskii01} instabilities.
\end{lstlisting}
위와 같이 입력한 후 컴파일하면 pdf 파일에는 다음과 같이 나타날 것이다.
\begin{quote}
Various plasma instabilities have been proposed as playing important roles during the substorm onset process. These include the tearing \cite{Schindler74, Sitnov97, Zelenyi08}, ballooning \cite{Cheng98, Bhattacharjee98, Dobias04, Zhu03, Saito08, Friedrich01}, lower hybrid drift \cite{Shinohara98, Yoon02, Mok06}, Kelvin--Helmholtz \cite{Rostoker84, Dovias06}, and the ion Weibel \cite{Yoon93, Sadovskii01} instabilities.
\end{quote}
이 때 참고문헌이 번호 순서대로 나오도록 한다. 또한 세 개 이상의 문헌이 연속된 번호로 이어진 경우 자동으로 첫 번호와 마지막 번호가 hyphen으로 연결된 형태로 등장함을 확인할 수 있다.


참고문헌은 다음의 조건들을 만족해야 한다.
\begin{itemize}
\item{저자가 명시되어야 한다.}
\item{검증이 된 내용이어야 한다.}
\item{이미 출판되어 수정이 불가능해야 한다.}
\end{itemize}
전문 논문 저널에 수록된 논문들은 위 조건들을 만족하므로 되도록 논문을 참고문헌으로 삼도록 한다. 웹사이트는 위 조건들을 만족하지 못하므로 참고문헌으로 부적절하다. 또한 누구라도 책을 출판할 수 있으므로 전문 서적을 참고문헌으로 사용하는 경우에는 널리 받아들여지고 인정받는 서적만 사용해야 한다. 사실 전공 서적의 저자는 여러 연구 논문들을 참고로 하여 책을 집필하기 때문에 전공서적에도 참고 문헌(논문)이 명시되어 있다. 이 경우 전공 서적 대신에 책에서 지시하는 논문을 참고문헌으로 삼도록 한다.

\paragraph{BibTeX 사용 방법}
BibTeX 은 \TeX 에서 참고문헌을 쉽게 관리하기 위한 도구이다. 보통의 경우 thebibliography 환경을 사용하여 참고문헌을 넣는데, BibTeX을 사용할 경우 단순히 다음과 같은 두 줄의 코드로 사용할 수 있다. 
\begin{lstlisting}
\bibliographystyle{ieetr}
\bibliography{bibfile}
\end{lstlisting}
위에서 `bibfile'은 단순히 참고문헌 데이터베이스가 기록되어 있는 BibTeX 파일의 이름이다. 또한, `ieetr'은 마치 MLA, APA style 처럼 BibTeX 에서 사용 가능한 bibliography 스타일 중 하나로, 가장 많이 쓰이는 스타일 중 하나이다. 인터넷 검색을 통해 TeXLive에서 기본적으로 제공되는 BibTeX style의 종류에 관해 알아볼 수 있다. 하지만 경기과학고 \TeX 사용자협회에서는 ieetr 스타일 대신 APA 스타일을 선호하였으나, APA 스타일을 완벽히 구현하는 알려진 스타일 파일이 없어서 직접 `mynewapa.bst' 파일을 제작하였다. 만약 당신도 이렇게 BibTeX 스타일을 직접 제작하고 싶다면 command line interface에서 `latex makebst' 를 치면 된다.

BibTeX이 단순한 thebibliography 환경에 비해 갖는 장점들은 다음과 같다.
\begin{itemize}
	\item 참고문헌들이 본문 내의 인용 순으로 {\bf 자동 정렬}된다.
	\item JabRef 와 같은 BibTeX 관리 프로그램을 이용하여 참고문헌들을 효율적으로 관리할 수 있다.
\end{itemize}

BibTeX 을 사용하기 위해서는 `bibfile.bib' 파일에 각각의 논문의 코드를 쌓아놓기만 하면 된다. 인용 방식은 평소와 같이 \textbackslash cite\{Mok10\} 과 같이 하면 되며, BibTeX 코드는 직접 작성할 수도 있으나 쉽게 얻는 방법은 다음과 같다.
\begin{enumerate}
	\item Google Scholar 에서 검색한 결과에서 `인용'을 클릭한다.
	\item APA, MLA style 등이 나온다. 보통 여기에서 텍스트를 얻어오곤 했을 것이다.
	\item 여기에서 BibTeX 코드를 얻고자 한다면, 하단의 `BibTeX' 을 클릭.
	\item 코드가 나온다. Ctrl+A, Ctrl+C로 복사, bibfile에 붙여넣기.
\end{enumerate}
Google Scholar 외에도 doi2bib 와 같이 DOI 만 갖고 있으면 bibtex 파일을 제공하는 사이트도 있으나, Google Scholar 에 비해서는 적은 양의 정보를 제공하는 것으로 보인다. 다음은 동일한 논문에 대한 thebibliography에서 사용할 코드와 BibTeX 코드이다. 어떤 차이점을 갖는지 비교해 보라.
\begin{quote}
	\verb+\+bibitem\{Mok10\}C. Mok, C.-M. Ryu, P. H. Yoon, and A. T. Y. Lui, ``Obliquely propagating electromagnetic drift ion cyclotron instability'', J. Geophys. Res., \{\verb+\+bf 115\}, A04218 (2010).
\end{quote}
\begin{lstlisting}
@article{Mok10,
title={Obliquely propagating electromagnetic drift ion cyclotron instability},
author={Mok, Chinook and Ryu, Chang-Mo and Yoon, PH and Lui, ATY},
journal={Journal of Geophysical Research: Space Physics},
volume={115},
number={A4},
year={2010},
publisher={Wiley Online Library}
}
\end{lstlisting} % 서론
	\section{Theoretical Background}

\subsection{줄 바꿈}

\LaTeX 에서 줄 바꿈을 하는 방법은 한컴이나 워드와는 다르다.
줄 바꿈을 하기 위해서는 Enter 키를 2번 연속으로 누르거나, 
Back slash 키를 두 번 연속 누르는 방법(\textbackslash \textbackslash)을 사용한다. 
또한 여러 줄을 띄우고 싶다면, \textbackslash vskip 명령어를 사용한다. 
예를 들어, \textbackslash vskip 2pc을 사용 시
\vskip 2pc
위와 같이 한 번에 두 줄을 띄운다. (pc 대신 cm나 mm 등도 사용 가능하다)


그런데 \textbackslash \textbackslash 를 제외한 나머지 방법 사용 시 첫 단어에 들여쓰기가 되는 것을 볼 수 있다. 들여쓰기를 하지 않으려면 문단의 첫부분에 \textbackslash noindent 명령어를 사용한다.

\subsection{그림 삽입}

\LaTeX 에서 그림을 넣는 명령어는 아래와 같다.
\begin{lstlisting}
	\begin{figure}[t]
		\begin{center}
			\includegraphics[width=.3\textwidth]{figure.png}
			\caption{Figure Caption}
			\label{figlabel}
		\end{center}
	\end{figure}
\end{lstlisting}
그림을 넣기 위해서는 images 폴더에 그림이 있어야 한다.
위 명령어에서 width는 그림의 너비, caption은 그림 밑에 표시될 설명, label은 그림을 언급할 때 사용하는 그림의 코드이다. 그림의 언급은 \textbackslash ref\{figlabel\}과 같은 방법으로 한다.

\noindent center 옆의 [t]는 그림을 넣을 위치이다. h, t, b, p 총 네 가지의 설정이 가능하며 각각 here(여기), top(페이지 맨 위), bottom(페이지 맨 아래), page(새로운 페이지)를 의미한다.
일반적으로 htbp를 사용하지만 논문과 같이 페이지 상단에 그림을 위치시켜야 하는 경우 [t]를 사용한다.

\newpage

\begin{figure}[t]
	\begin{center}
		\includegraphics[width=.3\textwidth]{figure.png}
		\caption{Figure Caption}
		\label{figlabel}
	\end{center}
\end{figure}

\noindent 예시 코드 실행 시 위와 같이 페이지 상단에 그림 \ref{figlabel}이 나타난다.

\subsection{표 삽입}

\LaTeX 에서 직접 표를 만들 수 있으나 명령어가 길고 복잡하며, 표의 모습을 보면서 작업할 수 없기에 본인이 원하는 모양대로 만드는 것이 불편하다는 단점이 있다.
그러므로 표의 경우 다른 조판 프로그램에서 제작한 후 이미지 형태로 저장해 \LaTeX 문서에 넣거나, 인터넷 사이트인 Tables Generator 등을 사용하는 것이 더 효율적이다. 다음은 Tables Generator를 사용해 만든 표이다.
\begin{lstlisting}
\begin{table}[htbp]
	\centering
	\caption{렌즈의 종류}
	\label{lens}
	\resizebox{.5\textwidth}{!}{%
		\begin{tabular}{|l|l|l|}
			\hline
			\multicolumn{1}{|c|}{} & 광각렌즈 & 망원렌즈 \\ \hline
			초점거리                   & 짧다   & 길다   \\ \hline
			화각                     & 크다   & 작다   \\ \hline
		\end{tabular}%
	}
\end{table}
\end{lstlisting}

\begin{table}[htbp]
	\centering
	\caption{렌즈의 종류}
	\label{lens}
	\resizebox{.5\textwidth}{!}{%
		\begin{tabular}{|l|l|l|}
			\hline
			\multicolumn{1}{|c|}{} & 광각렌즈 & 망원렌즈 \\ \hline
			초점거리                   & 짧다   & 길다   \\ \hline
			화각                     & 크다   & 작다   \\ \hline
		\end{tabular}%
	}
\end{table}

\subsection{수식 삽입}

문장 내부에서 수식을 사용하기 위해서는 \$ 수식 \$이나 \textbackslash( 수식 \textbackslash)와 같은 방법을 사용한다.
예를 들어 '소비 전력 \$P=I\^{}2 R\$이다.'라는 코드를 작성 후 컴파일하면
\begin{quote}
'소비 전력 $P=I^2 R$이다.'
\end{quote}
\noindent 와 같이 수식이 나타난다. \\
이때 수식과 문장의 높이를 맞추기 위해 분수나 시그마 등의 수학 기호의 크기가 조절된다. 수식의 가독성을 키우기 위해서는 수식 앞에 \textbackslash displaystyle을 붙이면 된다.
\begin{itemize}
	\item displaystyle을 사용하지 않은 경우 : $\sum_{i=1}^{n} = \frac{n(n+1)}{2}$
	\item displaystyle을 사용한 경우 : \(\displaystyle \sum_{i=1}^{n} = \frac{n(n+1)}{2}\)
\end{itemize}

\vskip 2pc

\noindent 수식이 문장 안에 있는 것이 아니라 한 줄을 통째로 차지하는 경우 \textbackslash[ 수식 \textbackslash]이나 명령어 \textbackslash equation을 사용한다. 차이점이라면 전자는 수식에 번호가 붙지 않고, 후자는 붙는다는 것이다. 각각을 사용한 명령어와 실행 모습은 다음과 같다.

\begin{lstlisting}
\[ K + U = \frac{1}{2} mv^2 - \frac{GMm}{r^2}\]
\end{lstlisting}
\[ K + U = \frac{1}{2} mv^2 - \frac{GMm}{r^2}\]

\vskip 2pc

\noindent \begin{lstlisting}
\begin{equation}
S = k_B \ln \Omega \label{equ1}
\end{equation}
\end{lstlisting}
\begin{equation}
	S = k_B \ln \Omega \label{equ1}
\end{equation}

\noindent \textbackslash equation을 사용해서 작성한 식을 언급하기 위해서는 \textbackslash ref\{equ1\}과 같이 ref 명령어를 사용한다. \\
예시 : 통계역학적 엔트로피는 식 \ref{equ1}(\textbackslash ref\{equ1\})과 같이 표현된다.

\vskip 2pc

\noindent 수식이 길어서 한 줄에 표현할 수 없는 경우에는 align 명령어를 사용한다.
\begin{lstlisting}
\begin{align}
F &= \frac{dp}{dt} \notag\\
&= \frac{d}{dt}(mv) \notag \\
&= \frac{dm}{dt}a + m\frac{dv}{dt} \notag \\
&= ma \textrm{ (if mass is constant)} \label{equ2}
\end{align}
\end{lstlisting}
위 명령어를 입력한 후 컴파일하면

\begin{align}
F &= \frac{dp}{dt} \notag\\
&= \frac{d}{dt}(mv) \notag \\
&= \frac{dm}{dt}a + m\frac{dv}{dt} \notag \\
&= ma \textrm{ (if mass is constant)} \label{equ2}
\end{align}

\noindent 위와 같이 여러 줄로 이루어진 수식 \ref{equ2}가 나온다. \\
등호 앞에 있는 \& 기호는 \&를 기준으로 줄을 맞추라는 의미이며 \\
중간 과정의 수식에는 번호를 붙이지 않게 하기 위해 마지막 줄을 제외한 각 줄의 끝에 \textbackslash notag를 붙인다.\\ % 이론적 배경
	\section{연구 과정}
%\section{Method of the study}

\subsection{제목}

내용 % 연구 과정
	\section{결과 및 토의}
%\section{Result and discussion}

\subsection{제목}

내용 % 결과 및 토의
	\section{결론}
글이라는 것은 개인의 개성이 담겨 있기 때문에 모든 사람들이 동일한 방식으로 표현하는 것은 아니다. 그러나 고대로부터 개인의 연구 내용을 글로써 타인에게 전달할 때, 효율적인 방법이라고 공감대를 형성하며 다듬어져 온 것이 지금의 논문 형태이다. 그러므로 처음 논문을 작성하는 학생들은 이 문서에서 지시하는 논문 작성 방식을 따르는 것을 권한다. 하지만 여기서는 다양한 논문들에 대해 일일이 사례를 들어 올바른 논문 작성법을 설명하기에는 한계가 있기에 간략하게만 소개를 했다. 여기서 설명되지 않은 부분들은 다른 사람들의 논문을 참고하자. 이미 서론을 작성하면서 많은 선행 연구 논문들을 읽어 봤을 것이다. 그 논문들에서는 데이터를 어떤 방식으로 표현하는지, 서론은 어떤 흐름으로 구성하는지 등을 살펴보자. 논문을 잘 쓰는 비결의 첫 번째는 논문을 많이 읽어 보는 것이다. 

+ 첨언을 하자면, 본교의 영어논문작성법 수업에 사용되는 `Science Research Writing for Non-Native Speakers of English' 를 참고하면 많은 도움이 될 것이다. % 결론
	% \section{추후 연구}
%\section{Further research}

\subsection{제목}

내용 % 추후 연구
	\section{부록}
%\section{Appendix} % 부록
	
	\bibliography{bibfile} % 참고문헌
	% BibTeX 코드 쉽게 얻어오는 방법 %
	% Google Scholar 에서 검색한 결과에서 `인용'을 클릭한다.
	% BibTeX 코드를 얻고자 한다면, 하단의 `BibTeX' 을 클릭.
	% 코드가 나온다. Ctrl+A, Ctrl+C로 복사, bibfile에 붙여넣기.


\end{document}
