\section{서론}

서론은 연구를 진행하게 된 배경을 기술하는 곳으로 보통 다음과 같은 순서로 쓰는 편이다.
\begin{itemize}
\item{연구 주제의 전반적 관심을 조명.}
\item{연구 분야의 스페셜 이슈를 조명.}
\item{해당 이슈를 해결하기 위한 다양한 선행 연구들을 서술.}
\item{선행 연구들의 한계점을 기술.}
\item{한계를 극복하기 위한 본 연구의 목적을 밝힘.}
\item{논문의 구성을 서술 (optional).}
\end{itemize}
서론은 과거부터 현재까지 해당 분야의 연구 진행을 기술하기 때문에 선행 연구 논문들을 레퍼런스로 도입하는 경우가 빈번하게 나타난다. \LaTeX에서 참고문헌을 표기하는 방법을 알아보자. 먼저 이 문서의 후반부에 위치한 레퍼런스 부분을 찾아간다. 이 문서를 컴파일했을 때 생성된 PDF 파일에는 {\bf References}라고 나와 있지만 여기서는 {\textbackslash}begin\{thebibliography\}\{99\}로 시작에서 {\textbackslash}end\{thebibliography\}로 종료되는 그 사이에 참고문헌을 작성하면 된다. 여기서 숫자 99는 참고문헌이 100개 넘는 논문을 작성하는 것이 아니라면 그대로 놔둔다. 참고문헌 작성 예시는 다음과 같다.
\begin{lstlisting}[breaklines=false]
\bibitem{Mok10}Mok, C., Ryu, C.-M., Yoon, P., \& Lui, A. (2010). Obliquely 
propagating electromagnetic drift ion cyclotron instability. \textit{Journal of 
Geophysical Research: Space Physics}, \textit{115}(A4).
\end{lstlisting}
{\textbackslash}bibitem 다음의 \{ \} 안에는 자신이 그 논문을 기억하기 쉬운 
규칙을 정하여 작성하면 된다. 보통 논문 주저자의 last name과 논문 출판 년도를 
사용하여 표기한다.
참고 문헌의 입력은 표준 APA 양식을 따른다. Author, Co-authors, (Publication 
Year), Article title, Journal title, Volume(Issue), pp.-pp. 순으로 작성한다. 
저자는 family name, personal name 순으로 입력하며 Family name 다음에 쉼표를 
찍는다. 저자는 7명까지 표시하며 7명이 넘을 경우 앞의 6명까지 표시한 다음 .~.~. 
뒤에 마지막 저자를 표시한다. 마지막 저자 앞에는 \& 기호를 붙여야 한다. 저널명은 
이탤릭체로 표기하며 관용적으로 약어로 표기 가능하다. 단 약어는 저널에서 
공식으로 인정된 표준 약어를 사용한다. 약어 뒤에는 마침표를 넣는다. 권 
(volume)은 기울임체로 숫자만 입력한다. 호 (issue)와 출판년도는 괄호 안에 
입력한다. 참고 문헌이 전문 서적인 경우, Author, (Publication Year), Title of 
work, Publisher City, State: Publisher, pp.-pp. 순서로 입력한다. 더 자세한 것은 
본 양식 참고 문헌 예시를 참고한다.

이제 서론에서 해당 논문을 인용할 준비 작업은 끝났다. 서론에서 필요한 부분에 이 논문을 인용 표기할 경우 {\textbackslash}cite 라고 입력한 후 \{ \} 안에 해당 논문을 표시하면 된다. 표시하는 방법은 바로 레퍼런스에서 {\textbackslash}bibitem 이후 \{ \} 안에 적었던 것을 넣어주면 된다. 논문 인용 표시가 문장 마지막에 등장할 때는 마침표의 위치는 인용 표시 다음이다. 아래 문장은 논문 인용 표시의 예로 C. Mok의 2010년 논문에서 인용하였다 \cite{Mok10}.
\begin{lstlisting}
Various plasma instabilities have been proposed as playing important roles during the substorm onset process. These include the tearing \cite{Schindler74, Sitnov97, Zelenyi08}, ballooning \cite{Cheng98, Bhattacharjee98, Dobias04, Zhu03, Saito08, Friedrich01}, lower hybrid drift \cite{Shinohara98, Yoon02, Mok06}, Kelvin--Helmholtz \cite{Rostoker84, Dovias06}, and the ion Weibel \cite{Yoon93, Sadovskii01} instabilities.
\end{lstlisting}
위와 같이 입력한 후 컴파일하면 pdf 파일에는 다음과 같이 나타날 것이다.
\begin{quote}
Various plasma instabilities have been proposed as playing important roles during the substorm onset process. These include the tearing \cite{Schindler74, Sitnov97, Zelenyi08}, ballooning \cite{Cheng98, Bhattacharjee98, Dobias04, Zhu03, Saito08, Friedrich01}, lower hybrid drift \cite{Shinohara98, Yoon02, Mok06}, Kelvin--Helmholtz \cite{Rostoker84, Dovias06}, and the ion Weibel \cite{Yoon93, Sadovskii01} instabilities.
\end{quote}
이 때 참고 문헌은 번호 순서대로 나오도록 한다. 또한 세 개 이상의 문헌이 연속된 
번호로 이어진 경우 자동으로 첫 번호와 마지막 번호가 hyphen으로 연결된 형태로 
등장함을 확인할 수 있다.


참고문헌은 다음의 조건들을 만족해야 한다.
\begin{itemize}
\item{저자가 명시되어야 한다.}
\item{검증이 된 내용이어야 한다.}
\item{이미 출판되어 수정이 불가능해야 한다.}
\end{itemize}
전문 논문 저널에 수록된 논문들은 위 조건들을 만족하므로 되도록 논문을 참고문헌으로 삼도록 한다. 웹사이트는 위 조건들을 만족하지 못하므로 참고문헌으로 부적절하다. 또한 누구라도 책을 출판할 수 있으므로 전문 서적을 참고문헌으로 사용하는 경우에는 널리 받아들여지고 인정받는 서적만 사용해야 한다. 사실 전공 서적의 저자는 여러 연구 논문들을 참고로 하여 책을 집필하기 때문에 전공서적에도 참고 문헌(논문)이 명시되어 있다. 이 경우 전공 서적 대신에 책에서 지시하는 논문을 참고문헌으로 삼도록 한다.

\paragraph{\hologo{BibTeX} 사용 방법}
\hologo{BibTeX} 은 \TeX 에서 참고문헌을 쉽게 관리하기 위한 도구이다. 보통의 
경우 thebibliography 환경을 사용하여 참고문헌을 넣는데, \hologo{BibTeX}을 
사용할 경우 단순히 다음과 같은 두 줄의 코드로 사용할 수 있다. 
\begin{lstlisting}
\bibliographystyle{ieeetr}
\bibliography{bibfile}
\end{lstlisting}
위에서 `bibfile'은 단순히 참고문헌 데이터베이스가 기록되어 있는 BibTeX 파일의 
이름이다. 또한, `ieeetr'은 MLA, APA style 처럼 \hologo{BibTeX} 에서 사용 
가능한 bibliography 스타일 중 하나로, 가장 많이 쓰이는 스타일 중 하나이다. 
인터넷 검색을 통해 TeXLive에서 기본적으로 제공되는 \hologo{BibTeX} style의 
종류에 관해 알아볼 수 있다. 2016학년도 경기과학고 졸업논문 양식은 표준 APA 
방식을 따른다고 되어 있어 경기과학고 \TeX 사용자협회에서도 ieeetr 스타일 대신 
APA 스타일을 사용하려 했으나, APA 스타일을 완벽히 구현하는 알려진 스타일 파일이 
없어서 `newapa.bst' 파일을 수정하여 `gshs\_thesis.bst' 파일을 제작하였다.
본 양식에는 이미 cls 파일에 bibliographystyle 명령이 지정되어 있으므로 실제 
문서를 작성할 때에는 bibliography 명령만 사용하면 된다.

\hologo{BibTeX}이 단순한 thebibliography 환경에 비해 갖는 장점들은 다음과 같다.
\begin{itemize}
	\item 참고문헌들이 본문 내의 인용 순으로 {\bf 자동 정렬}된다. (.bst 
	파일의 설정에 따라 인용 순 외 여러 정렬 방법을 이용할 수 있다.)
	\item JabRef 와 같은 \hologo{BibTeX} 관리 프로그램을 이용하여 
	참고문헌들을 효율적으로 관리할 수 있다.
\end{itemize}

\hologo{BibTeX} 을 사용하기 위해서는 `bibfile.bib' 파일에 각각의 논문의 코드를 
쌓아놓기만 하면 된다. 인용 방식은 평소와 같이 \textbackslash cite\{Mok10\} 과 
같이 하면 되며, \hologo{BibTeX} 코드는 직접 작성할 수도 있으나 쉽게 얻는 방법은 
다음과 같다.
\begin{enumerate}
	\item Google Scholar 에서 검색한 결과에서 `인용'을 클릭한다.
	\item APA, MLA style 등이 나온다. 보통 여기에서 텍스트를 얻어오곤 했을 
	것이다.
	\item 여기에서 BibTeX 코드를 얻고자 한다면, 하단의 `BibTeX' 을 클릭.
	\item 코드가 나온다. Ctrl+A, Ctrl+C로 복사, bibfile에 붙여넣기.
\end{enumerate}
Google Scholar 외에도 doi2bib 와 같이 DOI 만 갖고 있으면 \hologo{BibTeX} 파일을 
제공하는 사이트도 있으나, Google Scholar 에 비해서는 적은 양의 정보를 제공하는 
것으로 보인다. 다음은 동일한 논문에 대한 thebibliography에서 사용할 코드와 
\hologo{BibTeX} 코드이다. 어떤 차이점을 갖는지 비교해 보라.
\begin{lstlisting}[breaklines=false]
\bibitem{Mok10}Mok, C., Ryu, C.-M., Yoon, P., \& Lui, A. (2010). Obliquely 
propagating electromagnetic drift ion cyclotron instability. \textit{Journal of 
Geophysical Research: Space Physics}, \textit{115}(A4).
\end{lstlisting}

\begin{lstlisting}
@article{Mok10,
title={Obliquely propagating electromagnetic drift ion cyclotron instability},
author={Mok, Chinook and Ryu, Chang-Mo and Yoon, PH and Lui, ATY},
journal={Journal of Geophysical Research: Space Physics},
volume={115},
number={A4},
year={2010},
publisher={Wiley Online Library}
}
\end{lstlisting}
