\title{졸업논문 제목}
\engtitle{GSHS Thesis Title}

% 한글 이름, 한자 이름, 영어 성, 영어 이름 형식으로 입력.
\author{홍길동, 洪吉童, Hong, Gil-dong}
% 학번
\studid{23000}

% 지도 교사
\advisor{Hong, Pan-seo}

% 심사위원장, 심사위원 순서로 입력. 확인란에 입력할 문자나 사진을 코드로 직접 작성. 사진 삽입 시: img:/image/approval1 와 같이 img:사진 경로 입력. img:[height=2em]/path/to/image 와 같이 includegraphics에 옵션 지정 가능.
\judge{박ㅇㅇ, 김ㅇㅇ, 홍ㅇㅇ}
\degreeyear{2026} % 졸업 수료 연도  기준.
\date{2025.00.00.}

% 제목 설정
% kor: 목차, 그림 차례, 표 차례, 감사의 글, 연구 활동
% eng: Contents, List of Figures, List of Tables, Acknowledgments, Research Activities
\usetitle{kor}

% 캡션 설정
% kor: 그림, 표
% eng: Fig., Table
\usecaption{kor}


%% 논문 제출 전 체크리스트
%% 각 항목에 대하여 사실인 경우 \checklist{#}{check}로 uncheck를 check로 변경.

% 1. 이 논문은 내가 직접 연구하고 작성한 것이다.
\checklist{1}{uncheck}
% 2. 인용한 모든 자료(책·논문·인터넷자료 등)의 인용표시를 바르게 하였다.
\checklist{2}{uncheck}
% 3. 인용한 자료의 표현이나 내용을 왜곡하지 않았다.
\checklist{3}{uncheck}
% 4. 정확한 출처제시 없이 다른 사람의 글이나 아이디어를 가져오지 않았다.
\checklist{4}{uncheck}
% 5. 논문 작성 중 도표나 데이터를 조작(위조 혹은 변조)하지 않았다.
\checklist{5}{uncheck}
% 6. 다른 친구와 같은 내용의 논문을 제출하지 않았다.
\checklist{6}{uncheck}