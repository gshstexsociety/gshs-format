\begin{abstract}[eng]
	한글로 논문을 작성하더라고 논문의 영문 제목이 있어야 한다. 또한 영문 초록은 한글 초록을 영어로 번역하여 반드시 작성해야 한다.\\
	Put your abstract here. It is completely consistent with 한글 초록.
\end{abstract}

\begin{abstract}[kor]
	초록(요약문)은 본문을 모두 작성한 후 가장 마지막에 작성한다. 연구한 내용, 즉 본론부터 요약하며 서론 요약은 하지 않는다. 대개 첫 문장은 연구 주제 + 연구 방법을 핵심적으로 나타낼 수 있는 문구: 예) 실험적으로, 이론적으로, 수치적으로)를 쓴다. 다음으로 연구 방법을 요약한다. 선행 연구들과 구별되는 특징을 중심으로 쓴다. 뚜렷한 특징이 없다면 연구 방법은 생략할 수 있다. 다음으로 연구 결과를 요약한다. 연구 결과는 추론을 담지 않고 객관적으로 서술한다. 마지막으로 이 연구를 통해 주장하고자 하는 바를 간략히 정리하여 결론을 쓴다. 요약문 전체에서 연구 결과와 결론이 차지하는 비율이 절반이 넘도록 한다. 읽는 이가 요약문으로부터 얻으려는 정보는 연구 결과와 결론이기 때문이다. 연구 결과만 보고하는 논문인 경우, 결론을 쓰지 않는 경우도 있다.
\end{abstract}