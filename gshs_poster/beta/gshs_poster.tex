\documentclass[draft,b0]{sciposter}
%	draft 	% 문서 편집 중, 빠른 조판을 위한 기능.(그림을 제외하고 조판한다.) 
%% 최종본은 draft 대신 final 을 사용하면 된다.
%	b0		% 포스터 크기 : a0 - a3, b0 - b3, ansiE, ansiD, ansiC, tabloid, ra0 - ra2 가능
\usepackage{amsmath,amssymb,multicol,amsrefs}
\usepackage{lipsum}

% TODO : Make poster size customizable via custom.cfg
%% Custom, % 사용자 정의 포스터 크기. 크기를 변경하고 싶다면 custom.cfg 에서 값들을 수정.

\usepackage{kotex}
\usepackage{amssymb,amsmath}
\usepackage{caption,graphicx}
%\usepackage{subcaption} % sciposter class 에서 사용불가

\title{Title}
\author{14000 아무개, 15000 박대감}
\email{gshstexsociety@gmail.com}

\conference{제 62회 과학전람회}
\institute{과학영재학교 경기과학고등학교 \\ Gyeonggi Science High School for the Gifted}
\leftlogo[1.5]{gshslogo.png}

\newcommand{\imsize}{0.45\columnwidth}
% Color box 지정

\usepackage{tikz}
\usepackage{tcolorbox}
\usepackage{pgfplots}

\usetikzlibrary{shadows,calc}
\tcbuselibrary{skins,theorems,breakable}
\newtcolorbox{block}[2][\linewidth]{mybox,width=#1,title=#2}
\pgfplotsset{compat=1.4}
\definecolor{myblue}{RGB}{40,96,139}
\tcbset{ % to define box style
	mybox/.style={
		breakable,
		freelance,
		boxrule=0.4pt,
		width=\linewidth,
		frame code={%
			\path[draw=black,rounded corners,fill=white,drop shadow]
			(frame.south west) rectangle (frame.north east);
		},
		title code={
			\path[top color=myblue!30,bottom color=myblue!0.5,rounded corners,draw=none]
			([xshift=\pgflinewidth,yshift=-\pgflinewidth]frame.north west) rectangle ([xshift=-\pgflinewidth]frame.south east|-title.south east);
			\path[fill=myblue]
			([xshift=5pt,yshift=-\pgflinewidth]frame.north west) to[out=0,in=180] ([xshift=50pt,yshift=-5pt]title.south west) -- ([xshift=-50pt,yshift=-5pt]title.south east) to[out=0,in=180] ([xshift=-5pt,yshift=-\pgflinewidth]frame.north east) -- cycle;
		},
		fonttitle=\Large\bfseries\sffamily,
		fontupper=\sffamily,
		fontlower=\sffamily,
		before=\par\medskip,
		after=\par\medskip,
		center title,
		toptitle=3pt,
		top=11pt,
		colback=white
	}
}

% TODO : move it to cls file

\begin{document}
	\pagecolor{blue!5!white} % 경기과학고 교색 : 하늘색
	\maketitle
	\begin{abstract}
		\lipsum[1]
	\end{abstract}
	\begin{multicols}{2}
		\begin{block}{Introduction}
			\PARstart{경}{기과학}고등학교는 최근 중요성이 부각되고 있다. \cite{ashley,fab_coil} 정말 중요하다.\cite{mirvakili}
			\lipsum[1]
			
			\begin{figure}
				\begin{center}
					\begin{tabular}{c c} % subfigure environment is not recommended. subref doens't work since we can't use `subcaption' package...
						{\resizebox{\imsize}{!}{\includegraphics{example-image-a}}} &
						{\resizebox{\imsize}{!}{\includegraphics{example-image-b}}}\\
						(a) & (b) \\
						{\resizebox{\imsize}{!}{\includegraphics{example-image-c}}} &
						{\resizebox{\imsize}{!}{\includegraphics{example-image-a}}}\\
						(c) & (d) \\
					\end{tabular}
				\end{center}
				\caption{ Parts (a) through (c) show three images consisting of squares of
					different sizes; (d) shows the pattern spectra, denoting the number of foreground pixels removed by openings by reconstruction by $\lambda \times \lambda$ squares. No granulometry is capable of separating the patterns, because the only 	differences between the images lie in the distributions of the connected components. }\label{fig:blocks}
			\end{figure}
		\end{block}
		
		\begin{block}{Title of a block}
			\lipsum[3]
		\end{block}
		\begin{block}{Title of a block}
			\textbf{\scshape Needed Improvements}:
			\begin{itemize}
				\item higher intensity within
				\item small specimen area
				\item[$\Rightarrow$] focussed beams $\to$ STEM?!
			\end{itemize}
		\end{block}
		
		\begin{block}{Title of a block}
			\lipsum[4]
		\end{block}
		\begin{block}{Title of a block}
			\lipsum[2]
		\end{block}
		
		\begin{block}{Title of a block}
			\lipsum[3]
		\end{block}
		\begin{block}{Title of a block}
			\lipsum[4]
		\end{block}
		
		\begin{block}{Conclusion}
			\lipsum[1]
		\end{block}
		
	\end{multicols}
	
	% using bibtex
	\bibliographystyle{ieeetr}
	\bibliography{bibfile}

\end{document}