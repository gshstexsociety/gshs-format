%  The AAU Poster Theme.
%  2013-05-08 v. 1.1.0
%  Copyright 2013 by Jesper Kjær Nielsen <jkn@es.aau.dk>
%
%  You can find the GNU General Public License at <http://www.gnu.org/licenses/>.


% 경기과학고등학교 정보교사 채상미 선생님께서 2021년 포스터 양식에 맞게 수정함.


\documentclass[a0paper,portrait]{baposter}
\usepackage{kotex}
\usepackage[english]{babel}
\usepackage{helvet}
\renewcommand{\familydefault}{\sfdefault} % for text
\usepackage[helvet]{sfmath} % for math
\usepackage[T1]{fontenc}

\usepackage{caption}
\captionsetup{
  font=small,% set font size to footnotesize
  labelfont=bf % bold label (e.g., Figure 3.2) font
}
% Make the standard latex tables look so much better
\usepackage{array,booktabs}
% For creating beautiful plots
\usepackage{pgfplots}

\usepackage{amsmath}
% Adds new math symbols
\usepackage{amssymb}

%%%%%%%%%%%%%%%%%%%%%%%%%%%%%%%%%%%%%%%%%%%%%%%%
% Colours
% http://en.wikibooks.org/wiki/LaTeX/Colors
%%%%%%%%%%%%%%%%%%%%%%%%%%%%%%%%%%%%%%%%%%%%%%%%
\selectcolormodel{RGB}
% define the three aau colors : blue version
\definecolor{aaublue1}{RGB}{0,176,244}% dark blue
\definecolor{aaublue2}{RGB}{113,109,143} % light blue
\definecolor{aaublue3}{RGB}{194,193,204} % lighter blue
\definecolor{aauwhite}{RGB}{255,255,255}
\definecolor{aaublack}{RGB}{0,0,0}

%%%%%%%%%%%%%%%%%%%%%%%%%%%%%%%%%%%%%%%%%%%%%%%%
% Lists
% http://en.wikibooks.org/wiki/LaTeX/List_Structures
%%%%%%%%%%%%%%%%%%%%%%%%%%%%%%%%%%%%%%%%%%%%%%%%
% Easier configuration of lists
\usepackage{enumitem}
%configure itemize
\setlist{%
  topsep=0pt,% set space before and after list
  noitemsep,% remove space between items
  labelindent=\parindent,% set the label indentation to the paragraph indentation
  leftmargin=*,% remove the left margin
  font=\color{aaublack}\normalfont, %set the colour of all bullets, numbers and descriptions to aaublue1
}
% use set<itemize,enumerate,description> if you have an older latex distribution
\setitemize[1]{label={\raise1.25pt\hbox{$\blacktriangleright$}}}
\setitemize[2]{label={\scriptsize\raise1.25pt\hbox{$\blacktriangleright$}}}
\setitemize[3]{label={\raise1.25pt\hbox{$\star$}}}
\setitemize[4]{label={-}}
%\setenumerate[1]{label={\theenumi.}}
%\setenumerate[2]{label={(\theenumii)}}
%\setenumerate[3]{label={\theenumiii.}}
%\setenumerate[4]{label={\theenumiv.}}
%\setdescription{font=\color{aaublue1}\normalfont\bfseries}

% use setlist[<itemize,enumerate,description>,<level>] if you have a newer latex distribution
%\setlist[itemize,1]{label={\raise1.25pt\hbox{$\blacktriangleright$}}}
%\setlist[itemize,2]{label={\scriptsize\raise1.25pt\hbox{$\blacktriangleright$}}}
%\setlist[itemize,3]{label={\raise1.25pt\hbox{$\star$}}}
%\setlist[itemize,4]{label={-}}
%\setlist[enumerate,1]{label={\theenumi.}}
%\setlist[enumerate,2]{label={(\theenumii)}}
%\setlist[enumerate,3]{label={\theenumiii.}}
%\setlist[enumerate,4]{label={\theenumiv.}}
%\setlist[description]{font=\color{aaublue1}\normalfont\bfseries}

%%%%%%%%%%%%%%%%%%%%%%%%%%%%%%%%%%%%%%%%%%%%%%%%
% Misc
%%%%%%%%%%%%%%%%%%%%%%%%%%%%%%%%%%%%%%%%%%%%%%%%
% change/remove some names
\addto{\captionsenglish}{
  %remove the title of the bibliograhpy
  \renewcommand{\refname}{\vspace{-0.7em}}
  %change Figure to Fig. in figure captions
  \renewcommand{\figurename}{Fig.}
}
% create links
\usepackage{url}
%note that the hyperref package is currently incompatible with the baposter class

%%%%%%%%%%%%%%%%%%%%%%%%%%%%%%%%%%%%%%%%%%%%%%%%
% Macros
%%%%%%%%%%%%%%%%%%%%%%%%%%%%%%%%%%%%%%%%%%%%%%%%
\newcommand{\alert}[1]{{\color{aaublue1}#1}}

%%%%%%%%%%%%%%%%%%%%%%%%%%%%%%%%%%%%%%%%%%%%%%%%
% Document Start 
%%%%%%%%%%%%%%%%%%%%%%%%%%%%%%%%%%%%%%%%%%%%%%%%
\begin{document}
%%%%%%%%%%%%%%%%%%%%%%%%%%%%%%%%%%%%%%%%%%%%%%%%
% Some changes that cannot be made in the preamble
%%%%%%%%%%%%%%%%%%%%%%%%%%%%%%%%%%%%%%%%%%%%%%%%
% set the background of the poster
\background{
  \begin{tikzpicture}[remember picture,overlay]%
    %the poster background color
    \fill[fill=aauwhite] (current page.north west) rectangle (current page.south east);
    %the header
    \fill [fill=aauwhite] (current page.north west) rectangle ([yshift=-\headerheight] current page.north east);

  \end{tikzpicture}
}
% if you want to reduce the space before and after equations, use and adjust
% the following lines
%\addtolength{\abovedisplayskip}{-2mm}
%\addtolength{\belowdisplayskip}{-2mm}

%%%%%%%%%%%%%%%%%%%%%%%%%%%%%%%%%%%%%%%%%%%%%%%%
% General poster setup
%%%%%%%%%%%%%%%%%%%%%%%%%%%%%%%%%%%%%%%%%%%%%%%%
\begin{poster}{
  %general options for the poster
  grid=false,
  columns=2,
%  colspacing=4.2mm,
  headerheight=0.1\textheight,
  background=user,
%  bgColorOne=red!42, %is used when background != user and none
%  bgColortwo=green!42, %is used when background is shaded
  eyecatcher=true,
  %posterbox options
  headerborder=closed,
  borderColor=aaublue1,
  headershape=rectangle,
  headershade=plain,
  headerColorOne=aaublue1,
%  headerColortwo=yellow!42, %is used when the header background is shaded
  textborder=rectangle,
  boxshade=plain,
  boxColorOne=white,
%  boxColorTwo=cyan!42,%is used when the text background is shaded
  headerFontColor=white,
  headerfont=\Large\sf\bf,
  linewidth=1pt
}
%the Eye Catcher (the logo on the left)
{
  %this can be commented out or replaced by a company/department logo
  \includegraphics[height=0.65\headerheight]{./logo/gshslogo_21.jpeg}

  \includegraphics[height=0.25\headerheight]{./logo/qrcode.png}

 
  \includegraphics[height=0.25\headerheight]{./logo/qrcode.png}

}
%the poster title
{\color{black}\LARGE
 제목이 길어질다면 조절이 필요함. 그외 잘 조절하면서 작업하길 바랍니다. 
}
%the author(s)
{\color{black}\small
  \vspace{0.2em} 
  \begin{flushright}
  \normalsize{
  \textbf{
  사   업   명 : 2021년 경기과학고등학교 기초 R\&E\\[0.2em]
  연   구   자 : 홍길동(1학년), 이순신(1학년), 장영실(1학년)\\[0.2em]
  지 도 교 사 : 채상미 }}
  \end{flushright}
}

%%%%%%%%%%%%%%%%%%%%%%%%%%%%%%%%%%%%%%%%%%%%%%%%
% the actual content of the poster begins here
%%%%%%%%%%%%%%%%%%%%%%%%%%%%%%%%%%%%%%%%%%%%%%%%

\begin{posterbox}[name=intro,span=2, column=0,row=0, ]{ABSTRACT}

  Posters are often used at conferences for presenting exciting new research results.
  So far no official AAU poster theme is available to the researchers and students at Aalborg University (AAU).
 The present theme changes this.
  The theme is a particular configuration of the \alert{baposter} poster template \cite{baposter} which you can find here \url{http://www.brian-amberg.de/uni/poster/} and must have installed in order to use the AAU poster theme.

\end{posterbox}

\begin{posterbox}[name=usage,column=0,below=intro]{INTRODUCTION}
\begin{itemize}
  \item To use the AAU poster theme, place the {\tt aauposter.tex} file in your preferred folder and modify the file to your needs.
  \item You can read more about how you can modify the theme in the documentation for the baposter template which you can find here \url{http://www.brian-amberg.de/uni/poster/}.
  \item \footnotesize 한글 지원도 잘 됩니다. 폰트가 마음에 안들어요.
   \item To use the AAU poster theme, place the {\tt aauposter.tex} file in your preferred folder and modify the file to your needs.
  \item You can read more about how you can modify the theme in the documentation for the baposter template which you can find here \url{http://www.brian-amberg.de/uni/poster/}.
  \item \footnotesize 한글 지원도 잘 됩니다. 폰트가 마음에 안들어요.
\end{itemize}
\end{posterbox}

\begin{posterbox}[name=lists,column=0,below=usage]{THEORETICAL BACKGROUND}
Itemize
\begin{itemize}
  \item item 1
    \begin{itemize}
      \item subitem 1
        \begin{itemize}
          \item subsubitem 1
            \begin{itemize}
              \item subsubsubitem 1
              \item subsubsubitem 2
            \end{itemize}
          \item subsubitem 2
        \end{itemize}
      \item subitem 2
    \end{itemize}
  \item item 2
\end{itemize}
Enumerate
\begin{enumerate}
  \item item 1
    \begin{enumerate}
      \item subitem 1
        \begin{enumerate}
          \item subsubitem 1
            \begin{enumerate}
              \item subsubsubitem 1
              \item subsubsubitem 2
            \end{enumerate}
          \item subsubitem 2
        \end{enumerate}
      \item subitem 2
    \end{enumerate}
  \item item 2
\end{enumerate}
Description
\begin{description}
  \item[desc 1] item 1
    \begin{description}
      \item[desc 1] subitem 1
        \begin{description}
          \item[desc 1] subsubitem 1
            \begin{description}
              \item[desc 1] subsubsubitem 1
              \item[desc 2] subsubsubitem 2
            \end{description}
          \item[desc 2] subsubitem 2
        \end{description}
      \item[desc 2] subitem 2
    \end{description}
  \item[desc 2] item 2
\end{description}
\end{posterbox}
\begin{posterbox}[name=install,column=0,below=lists]{MAIN CONTENTS}
You can either make a local or a global installation of the baposter poster template\cite{baposter}.
You can either make a local or a global installation of the baposter poster template\cite{baposter}.
\begin{description}
  \item[Local:] Place the {\tt baposter.cls} file in the same folder as the poster file {\tt aauposter.tex}
  \item[Global:] Place the {\tt baposter.cls} file in your local latex-directory tree. This is by default {\tt <somewhere>/textmf/tex/latex/baposter} where {\tt <somewhere>} is
 
\end{description}
 \begin{description}
    \item[GNU/Linux:] {\tt/home/<username>}
    \item[Windows XP:] {\tt c:\textbackslash Document and Settings\textbackslash<username>}
    \item[Windows Vista+:] {\tt c:\textbackslash Users\textbackslash<username>}
    \item[Mac OSX] {\tt/home/<username>/Library}
  \end{description}
\end{posterbox}

\begin{posterbox}[name=equation,column=1,below=intro]{MAIN CONTENTS}

  On GNU/Linux and Windows, you have to update the filename database after placing {\tt baposter.cls} in the correct folder. This is done by
  \begin{description}
    \item[GNU/Linux:] {\tt \$ texhash \textasciitilde /texmf}
    \item[Windows with MiKTeX] Open the MiKTeX Settings dialog and click 'Refresh FNDB'.
    \item[Windows with TeX Live] Open the TeX Live Manager dialog and select 'Update filename database' under 'Actions'.
  \end{description}
Here is an example of an equation
\begin{equation}
  f_X(x|\mu,\sigma^2) = \frac{1}{\sqrt{2\pi\sigma^2}}\exp\left\{\frac{1}{2\sigma^2}(x-\mu)^2\right\}
\end{equation}
\end{posterbox}


\begin{posterbox}[name=figures,column=1,below=equation]{FIGURES \& TABLES}
You cannot use floats in the baposter template. However, you can use figure captions by using {\tt \textbackslash captionof} instead of {\tt \textbackslash caption}. This is demonstrated in Fig.~\ref{fig:figlabel}. Moreover, you can also use {\tt \textbackslash label} and {\tt \textbackslash ref} to make references to your figures and/or tables.
\begin{center}
  \includegraphics[height=0.75\headerheight]{./logo/gshslogo.png}
  \captionof{figure}{Here is a figure caption}
  \label{fig:figlabel}
\end{center}

{\tt
f1 = figure(1);\\
set(f1,'Color','none');
}\par
You can also use {\tt pgfplots} \cite{pgfplots} for plotting your Matlab data. This is not that hard and the resulting plots are much nicer than Matlab plots, so I will strongly recommend that you have a look at {\tt pgfplots} right here \url{http://sourceforge.net/projects/pgfplots/}.
\begin{center}
  \begin{tabular}{c c c}
    \toprule
    header 1 & header 2 & header 3\\
    \midrule
    data (1,1) & data (1,2) & data (1,3)\\
    data (2,1) & data (2,2) & data (2,3)\\
    data (3,1) & data (3,2) & data (3,3)\\
    \bottomrule
  \end{tabular}
  \captionof{table}{A very simple table with booktabs}
  \label{tab:tablabel}
\end{center}
\end{posterbox}


\begin{posterbox}[name=feedback,column=1,below=figures]{CONCLUSION}
  \begin{itemize}
    \item The AAU poster theme v. 1.1.0 has been tested with baposter v. 2.0, and it can be downloaded from my AAU website \cite{jknaau} or my personal website \cite{jknsqrt-1}.
    \item If you find a bug in the AAU theme (and not in the baposter template), please do not hesitate to contact me. There is a FAQ at the baposter website \cite{baposter}, if you should have any problems with it.
    \item The AAU poster theme v. 1.1.0 has been tested with baposter v. 2.0, and it can be downloaded from my AAU website \cite{jknaau} or my personal website \cite{jknsqrt-1}.
  \end{itemize}
\end{posterbox}

\begin{posterbox}[name=refs,span=2,column=0,below=install,above=bottom]{REFERENCES}
% In the last box, you will usually have a list of references
% The bibliography automatically adds the title "References", but
% this have been removed in the preamble

% use either ......
\bibliographystyle{plain}
\begin{thebibliography}{1}% Simple bibliography with widest label of 1
\itemsep=-0.01em% Save space between the separation
\setlength{\baselineskip}{0.4em}% Save space with longer lines
\bibitem{baposter} Brian Amberg: \emph{LaTeX Poster Template}, \url{http://www.brian-amberg.de/uni/poster/} 
\bibitem{pgfplots} Christian Feuersänger: \emph{PGFPlots - A LaTeX Package to create normal/logarithmic plots in two and three dimensions}, \url{http://pgfplots.sourceforge.net/} 
\bibitem{jknaau} Jesper Kjær Nielsen: \emph{Official AAU Beamer Theme, Poster Theme, and Report Template}, \url{http://kom.aau.dk/~jkn/latex/latex.php}
\bibitem{jknsqrt-1} Jesper Kjær Nielsen: \emph{Official AAU Beamer Theme, Poster Theme, and Report Template}, \url{http://sqrt-1.dk/latex/latex.php}
\end{thebibliography}

% ...... or
%  \bibliographystyle{IEEEbib}
%  \bibliography{IEEEabrv,mybib,mybib2} 
\end{posterbox}

\end{poster}
\end{document}
