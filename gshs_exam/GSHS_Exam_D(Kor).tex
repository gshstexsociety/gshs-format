\documentclass[]{gshs_exam_D}

\usetikzlibrary{patterns,decorations.pathmorphing,arrows.meta}
\usepackage{bm}

\usepackage{tikz-3dplot}


\makeatletter
\myyear{2018}\let\MyYear\@myyear %학년도
\semester{1}\let\Semester\@semester %학기
\exams{1차 지필평가}\let\Exams\@exams %1차,2차 지필평가
\subject{정수론}\let\Subject\@subject %과목명
\credits{3} %학점
\pfscore{100} %만점
\examtime{80분} %시험시간
\examinerI{김가혜} %출제1
\examinerII{조영득} %출제2
\makeatother

%% 한글줄간격 %%
\renewcommand{\baselinestretch}{1.3}

\begin{document}

\maketitle

%%% page 1 %%%
\begin{multicols*}{2}
\noindent\fbox{\parbox{0.98\columnwidth}{\vspace*{-0.6em}
\begin{enumerate}[leftmargin=5.5mm,label=※]
\item 문항에 따라 배점이 다르므로 각 물음의 끝에 표시된 배점을 참고하시오.\\[-2.2em]
\item 단답형 ( 30 )점 포함, 서술형 ( 30 )점 포함, 논술형 ( 40 )점 포함
\end{enumerate}\vspace{-0.6em}}}\vspace{1em}

\begin{questions}
\extrawidth{8.1em}
%%% Problem 1 %%%
\addpoints
\question 다음 물음에 답하시오.\droptotalpoints
\vspace{0.3em}
\begin{parts}
\part[2] \ddh\ 정렬성의 원리를 기술하시오.\droppoints
\vspace{6em}
\part[6] \nsh\ 정렬성의 원리를 이용하여 유한귀납법의 기본 원리를 증명하시오.\droppoints
\noindent\fbox{
\parbox{0.95\linewidth}{
\textbf{[유한귀납법의 기본 원리]}\par
양의 정수들로 이루어진 집합 $S$가 다음 두 가지 성질
\begin{itemize}
\item[1)] 정수 $1$은 $S$에 속한다.
\item[2)] 정수 $k$가 $S$에 속하면, 다음 정수 $k+1$도 $S$에 속한다.
\end{itemize}
를 만족할 때, 집합 $S$는 모든 양의 정수를 가진다.
}}
\end{parts}


\vspace*{\fill}
\columnbreak

%%% Problem 2 %%%
\vspace*{-0.5em}
\question 다음 물음에 답하시오.\droptotalpoints
\vspace{0.3em}
\begin{parts}
\part \ddh\ 정수 $a$, $b$에 대하여 $\mathrm{gcd}(a,b)=1$일 때, 다음 최대공약수가 가질 수 있는 가능한 값을 모두 구하시오.
\begin{subparts}
\subpart[3] $\mathrm{gcd}(3a+b,a+3b)$ \droppoints
\vspace{6em}
\subpart[3] $\mathrm{gcd}(a-b,a^2 +b^2 )$ \droppoints
\end{subparts}
\vspace{6em}
\part[7] \nsh\ 정렬성의 원리와 나눗셈 정리를 이용하여 다음 정리를 증명하시오.\droppoints
\fbox{\parbox{0.95\linewidth}{
정수 $a$, $b$ 중 적어도 하나가 $0$이 아닐 때, $\mathrm{gcd}(a,b)=ax+by$를 만족하는 정수 $x$, $y$가 존재한다.}}
\end{parts}


\end{questions}
\end{multicols*}


%%% page 2 %%%
\begin{multicols*}{2}
\begin{questions}\extrawidth{8.1em}\setcounter{question}{2} %이전 페이지 마지막 문항 번호 입력

%%% Prob 3 %%%
\addpoints
\question 다음 물음에 답하시오.\droptotalpoints
\vspace{0.3em}
\begin{parts}
\part[3] \ddh\ 모든 항이 소수로 이루어진 길이가 $6$인 유한등차수열을 하나 찾으시오.\droppoints
\vspace{5em}
\part[3] \ddh\ 반복수 $R_{12}$의 소인수 $6$개 구하시오. (단, $R_n =(10^n -1)/9$이다.)\droppoints
\vspace{5em}
\part[2] \ddh\ 버트란드의 공준을 기술하시오.\droppoints
\vspace{5em}
\part[6] \nsh\ 버트란드의 공준을 이용하여 이항계수 $N=\displaystyle{\begin{pmatrix}2n\\n\end{pmatrix}}$에 대하여 $p\mid N$이면서 $p^2 \nmid N$인 $N$의 소인수 $p$가 존재함을 증명하시오. (단, $n\ge 2$이다.)\droppoints
\end{parts}

\vspace*{\fill}
\columnbreak

%%% Prob 4 %%%
\addpoints
\question 양의 정수 $n$과 정수 $a$, $b$, $c$, $d$에 대하여 다음 명제의 참$\cdot$거짓을 판단하고, 그 이유를 서술하시오.\droptotalpoints
\vspace{0.3em}
\begin{parts}
\part[3] \ssh\ $ab\equiv cd\pmod n$)이고 $a\equiv c\pmod n$이면 $b\equiv d\pmod n$이다. (단, $a$, $b$, $c$, $d$는 모두 $n$의 배수가 아니다.)\par\droppoints
\vspace{5em}
\part[3] \ssh\ 법 $n$에 대한 두 잉여류 $\bar{a}$, $\bar{b}$에 대하여 $\bar{a}\cap\bar{b}\ne\varnothing$이면 $\mathrm{gcd}(a,n)=\mathrm{gcd}(b,n)$이다.\droppoints
\vspace{5em}
\part[3] \ssh\ $\mathrm{gcd}(a,n)=\mathrm{gcd}(b,n)=1$이면 $\{ ba,\,ba^2 ,\,ba^3 ,\cdots,ba^n \}$은 법 $n$에 대한 완전잉여계이다.\droppoints
\vspace{5em}
\part[4] \ssh\ 양의 정수 $N$의 십진 표현이 $N=\displaystyle\sum_{i=0}^m a_i 10^i$, $0\le a_i \le 9$일 때, $N$이 $6$의 배수일 필요충분조건은 $a_0 +4\displaystyle\sum_{i=1}^m a_i \equiv 0\pmod 6$이다.\droppoints
\end{parts}


\end{questions}
\end{multicols*}


\end{document}